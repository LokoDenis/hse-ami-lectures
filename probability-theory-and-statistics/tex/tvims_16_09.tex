\documentclass[a4paper,12pt]{article}

%% Работа с русским языком
\usepackage{cmap}					% поиск в PDF
\usepackage{mathtext} 				% русские буквы в формулах
\usepackage[T2A]{fontenc}			% кодировка
\usepackage[utf8]{inputenc}			% кодировка исходного текста
\usepackage[english,russian]{babel}	% локализация и переносы

%% Отступы между абзацами и в начале абзаца 
\setlength{\parindent}{0pt}
\setlength{\parskip}{\medskipamount}

%% Изменяем размер полей
\usepackage[top=0.5in, bottom=0.75in, left=0.625in, right=0.625in]{geometry}

%% Графика
\usepackage[pdftex]{graphicx}
\graphicspath{{images/}}

%% Различные пакеты для работы с математикой
\usepackage{mathtools}				% Тот же amsmath, только с некоторыми поправками

\usepackage{amssymb}				% Математические символы
\usepackage{amsthm}					% Пакет для написания теорем
\usepackage{amstext}
\usepackage{array}
\usepackage{amsfonts}
\usepackage{icomma}					% "Умная" запятая: $0,2$ --- число, $0, 2$ --- перечисление
\usepackage{bbm}				    % Для красивого (!) \mathbb с  буквами и цифрами
\usepackage{enumitem}               % Для выравнивания itemise (\begin{itemize}[align=left])

% Номера формул
\mathtoolsset{showonlyrefs=true} % Показывать номера только у тех формул, на которые есть \eqref{} в тексте.

% Ссылки
\usepackage[colorlinks=true, urlcolor=blue]{hyperref}

% Шрифты
\usepackage{euscript}	 % Шрифт Евклид
\usepackage{mathrsfs}	 % Красивый матшрифт

% Свои команды\textbf{}
\DeclareMathOperator{\sgn}{\mathop{sgn}}

% Перенос знаков в формулах (по Львовскому)
\newcommand*{\hm}[1]{#1\nobreak\discretionary{}
{\hbox{$\mathsurround=0pt #1$}}{}}

% Графики
\usepackage{tikz}
\usepackage{pgfplots}
%\pgfplotsset{compat=1.12}

% Изменим формат \section и \subsection:
\usepackage{titlesec}
\titleformat{\section}
{\vspace{1cm}\centering\LARGE\bfseries}	% Стиль заголовка
{}										% префикс
{0pt}									% Расстояние между префиксом и заголовком
{} 										% Как отображается префикс
\titleformat{\subsection}				% Аналогично для \subsection
{\Large\bfseries}
{}
{0pt}
{}

% Информация об авторах
\author{Группа лектория ФКН ПМИ 2015-2016 \\
	Анастасия Иовлева \\
	Ксюша Закирова \\
	Руслан Хайдуров}
\title{Лекции по предмету \\
	\textbf{Линейная алгебра и геометрия}}
\date{2016 год}

\newtheorem*{Def}{Определение}
\newtheorem*{Lemma}{Лемма}
\newtheorem*{Suggestion}{Предложение}
\newtheorem*{Examples}{Пример}
\newtheorem*{Comment}{Замечание}
\newtheorem*{Consequence}{Следствие}
\newtheorem*{Theorem}{Теорема}
\newtheorem*{Statement}{Утверждение}
\newtheorem*{Task}{Упражнение}
\newtheorem*{Designation}{Обозначение}
\newtheorem*{Generalization}{Обобщение}
\newtheorem*{Thedream}{Предел мечтаний}
\newtheorem*{Properties}{Свойства}

\renewcommand{\mathbb}{\mathbbm}
\renewcommand{\Re}{\mathrm{Re\:}}
\renewcommand{\Im}{\mathrm{Im\:}}
\newcommand{\Arg}{\mathrm{Arg\:}}
\renewcommand{\arg}{\mathrm{arg\:}}
\newcommand{\Mat}{\mathrm{Mat}}
\newcommand{\id}{\mathrm{id}}
\newcommand{\isom}{\xrightarrow{\sim}} 
\newcommand{\leftisom}{\xleftarrow{\sim}}
\newcommand{\Hom}{\mathrm{Hom}}
\newcommand{\Ker}{\mathrm{Ker}\:}
\newcommand{\rk}{\mathrm{rk}\:}
\newcommand{\diag}{\mathrm{diag}}
\newcommand{\ort}{\mathrm{ort}}
\newcommand{\pr}{\mathrm{pr}}
\newcommand{\vol}{\mathrm{vol\:}}

\renewcommand{\epsilon}{\varepsilon}
\renewcommand{\phi}{\varphi}
\newcommand{\e}{\mathbb{e}}
\renewcommand{\l}{\lambda}
\renewcommand{\C}{\mathbb{C}}
\newcommand{\R}{\mathbb{R}}
\newcommand{\E}{\mathbb{E}}

\newcommand{\vvector}[1]{\begin{pmatrix}{#1}_1 \\\vdots\\{#1}_n\end{pmatrix}}
\renewcommand{\vector}[1]{({#1}_1, \ldots, {#1}_n)}
\newcommand\independent{\protect\mathpalette{\protect\independenT}{\perp}}
\def\independenT#1#2{\mathrel{\rlap{$#1#2$}\mkern2mu{#1#2}}}

\begin{document}

\section*{}

\subsection{Задача о сумасшедшей старушке}

\textbf{Условие:} Есть самолёт имеющий $n$ мест, в который садятся $n$ пассажиров. Первой в него заходит некоторая старушка, которая своего места не знает, и садится на случайное; каждый следующий садится на своё место, если оно свободно, и на случайное, если своё занято. 
\paragraph{а)} Какова вероятность, что последний человек сядет на своё место?
\paragraph{б)} Какова вероятность, что предпоследний человек сядет на своё место?
\paragraph{в)} Какова вероятность, что они оба сядут на своё место?

\paragraph{a)} Правильный ответ, как ни странно, $\frac{1}{2}$ (прямо как в задаче про динозавра --- либо сядет, либо не сядет). Кажется неверным, но если задуматься, становится ясно --- есть только два варианта того, куда он может сесть --- либо на своё место, либо на место старушки. Объяснение очень просто --- 

\paragraph{б)} Тут ответом являются $\frac{2}{3}$;

\paragraph{в)} Тут ответом является $\frac{1}{3}$.


\paragraph{}А теперь строже:

\paragraph{а)} Давайте рассмотрим несколько примеров:
\begin{itemize}
    \item $n = 2$; вероятность, очевидно, $\frac{1}{2}$;
    \item $n = 3$; $P(A) = \frac{1}{3} \text{(старушка села на своё место)} + \frac{1}{3}P(A\mid \text{старушка села на второе место}) = \frac{1}{3} + \frac{1}{6} = \frac{1}{2}$.
\end{itemize}

    Пока гипотеза подтверждается.

    Пусть $B_i = \{\text{старушка села на место под номером $i$}\}$; считаем, что её номер --- 1. Воспользуемся методом индукции:

    \textbf{База:} Для $n=2$ верно.
    \textbf{Переход:} Пусть верно для всех $n < k$; докажем для $k$:

    \[
    P(A) = \{\text{формула полной вероятности}\} = \sum\limits_{i=1}^n P(A\mid B_i)P(B_i)
    \]

    \[P(A\mid B_1) = 1\]
    \[P(A\mid B_n) = 0\]
    \[P(A\mid B_i) = \frac{1}{2} 2\leqslant i\leqslant n-1 \text{, т.к. теперь $i$-ый пассажир стал старушкой}\]

    \[\frac{n-2}{n}\times\frac{1}{2} + \frac{1}{n}\times 1 + \frac{1}{n}\times 0 = \frac{1}{2} \qed\]

\paragraph{б)}
    Опять же, примеры:
\begin{itemize}
    \item $n=3$: $P(C) = \frac{1}{3}$
\end{itemize}

    \[P(C\mid B_1) = 1\]
    \[P(C\mid B_n) = 1\]
    \[P(C\mid B_{n-1}) = 0\]
    \[P(C\mid B_i) = \frac{2}{3} 2\leqslant i\leqslant n-2\]

    Далее аналогично.

\paragraph{в)}

    $D = A\cap C \implies P(D) = P(A\cap B)$ 

    \[P(A) = \{\text{формула полной вероятности}\} = \sum\limits_{i=1}^n P(D\mid B_i)P(B_i)\]


    \[P(C\mid B_1) = 1\]
    \[P(C\mid B_n) = 0\]
    \[P(C\mid B_{n-1}) = 0\]
    \[P(C\mid B_i) = \frac{1}{3} 2\leqslant i\leqslant n-2\]

    \[\frac{n-2}{n}\times\frac{1}{3} + \frac{1}{n}\times 1 + \frac{2}{n}\times 0 = \frac{1}{3} \qed\]

    Кажется, что эта вероятность равна произведению двух прошлых; \emph{это \sout{ж-ж-ж} неспроста}

\subsection{Удачливый студент}

\textbf{Условие:}

Студент знает $k$ билетов из $n$. Каким ему нужно встать в очередь из $n$ студентов, чтобы вероятность вытянуть ``хороший'' билет была максимальнв?

\textbf{Решение:} 

Пусть $A_s = \{\text{студент вытянул хороший билет, стоя на $s$-ом месте}\}$;

$B_i = \{\text{до студента взяли ровно $i$ хороших билетов}\}$.

\[P(A_s) = \sum\limits_{c = 0}^{\min(k, \text{не разобрал})} P(A_s\mid B_i)P(B_i)\]

\[P(A_s\mid B_i) = \frac{k-i}{n-s+1}\]
\[P(B_i) = \frac{C_k^i C_{n-k}^{s-i-1}\text{\sout{$(s-1)!$}}}{C_n^{s-1}\text{\sout{$(s-1)!$}}}\]



\subsection{Формула Байеса}

Пусть $\{B_i, i = 1\ldots n\}$ --- разбиение $\Omega$, причём $P(B_i) > 0$. тогда для события $A$ такого что $P(A) > 0$ выполняется

\[P(B_i\mid A) = \frac{P(A\mid B_i)P(B_i)}{\sum\limits_{j=1}^nP(A\mid B_j)P(B_j)}\]

\subsection{Независимость}

\textbf{Определение:} события $A$ и $B$ называются независимыми тогда, когда $P(A\cap B) = P(a)P(B)$; обозначение  --- $A \independent B$

\textbf{Примеры:}
\begin{itemize}
    \item Задача про старушку; $A = \{\text{последний сел на своё место}\}$, \\ $B = \{\text{предпоследний сел на своё место}\}$
    \item Бросок игральной кости;  $A = \{\text{выпало чётное число}\}$, \\ $B = \{\text{выпало число, делящееся на три}\}$
\end{itemize}

События $A_1, \ldots, A_n$ называются \emph{попарно независимыми} если $\forall i, j$ $A_i$ и $A_j$ независимы.
События $A_1, \ldots, A_n$ называются \emph{независимыми в совокупности} если для любого набора выполняется $P(A_i\cap\ldots\cap A_k) = \prod P(A_j)$.

\textbf{Пример:} есть тетраэдр с раскрашенными гранями: К, С, З и КСЗ. Три события:
\begin{itemize}
    \item  $A = \{\text{на нижней грани есть красный цвет}\}$
    \item  $B = \{\text{на нижней грани есть синий цвет}\}$
    \item  $C = \{\text{на нижней грани есть зелёный цвет}\}$
\end{itemize}

Очевидно, что вероятность любого события --- $\frac{1}{2}$; любой пары событий --- $\frac{1}{4}$; однако вероятность всех трёх разом не равна $\frac{1}{8}$, значит, эти события независимы попарно, но не в совокупности.

\textbf{Упражнение:} привести такие события, что любой набор из $n-1$ события независим, а из $n$ --- зависимы.

\textbf{Утверждения:}
\begin{itemize}
    \item Если $A$ независимо с $A$, то его вероятность равна либо 0, либо 1 и оно также независимо с любым другим событием.
    \item $A \independent B \implies \overline{A}\independent B$
    \item ??? не смог разобрать
\end{itemize}

\section{Случайные величины в дискретных вероятностных пространствах}

\textbf{Определение:} пусть $(\Omega, P)$ --- вероятностное дискретное пространство; отображение \\ $\xi: \Omega \to \mathbb{R}$ называется случайной величиной.

\textbf{Примеры:} 
\begin{itemize}
    \item Индикаторы.

        Пусть $A\in \Omega$ --- событие. Тогда индикатором события $A$ называют

        \[
            I_A(w) = \begin{cases}
                1, w \in A; \\
                1, w \not\in A;
            \end{cases}
        \]

    \item Бросок игральной кости;

    \item Схема Бернулли
\end{itemize}

\end{document}
