\documentclass[a4paper,12pt]{article}

%% Работа с русским языком
\usepackage{cmap}					% поиск в PDF
\usepackage{mathtext} 				% русские буквы в формулах
\usepackage[T2A]{fontenc}			% кодировка
\usepackage[utf8]{inputenc}			% кодировка исходного текста
\usepackage[english,russian]{babel}	% локализация и переносы

%% Геометрия
\usepackage[left=2cm, right=2cm, top=1cm, bottom=2cm]{geometry}

%% Графика
\usepackage[pdftex]{graphicx}
\graphicspath{{images/}}

%% Дополнительная работа с математикой
\usepackage{amsmath,amsfonts,amssymb,amsthm,mathtools} % AMS
\usepackage{icomma} % "Умная" запятая: $0,2$ --- число, $0, 2$ --- перечисление

% Номера формул
\mathtoolsset{showonlyrefs=true} % Показывать номера только у тех формул, на которые есть \eqref{} в тексте.

% Шрифты
\usepackage{euscript}	 % Шрифт Евклид
\usepackage{mathrsfs}	 % Красивый матшрифт

% Свои команды\textbf{}
\DeclareMathOperator{\sgn}{\mathop{sgn}}

% Перенос знаков в формулах (по Львовскому)
\newcommand*{\hm}[1]{#1\nobreak\discretionary{}
{\hbox{$\mathsurround=0pt #1$}}{}}

% Графики
\usepackage{pgfplots}

%% Заголовок
\author{}
\title{Линейная алгебра, 3 модуль, лекция 15}
\date{11.01.2016}

\newtheorem*{Def}{Определение}
\newtheorem*{Lemma}{Лемма}
\newtheorem*{Suggestion}{Предложение}
\newtheorem*{Examples}{Примеры}
\newtheorem*{Comment}{Замечание}

\renewcommand{\Re}{\mathrm{Re\:}}
\renewcommand{\Im}{\mathrm{Im\:}}
\newcommand{\Arg}{\mathrm{Arg\:}}
\renewcommand{\arg}{\mathrm{arg\:}}

\begin{document}
\maketitle

Для начала вспомним, что такое \textit{векторное пространство} — это множество, на котором введены операции сложения, умножения на скаляр и в котором будут выполнятся восемь аксиом (см. 1 семестр). Но что такое скаляр?

\begin{Def}
Скаляры — это элементы некоторого фиксированного поля. 
\end{Def}

\begin{Def}
Полем называется множество $F$, на котором заданы две операции —  «сложение» $(+)$ и «умножение» $(\cdot)$,
\[
F \times F \rightarrow F \Rightarrow
\begin{aligned}
+\!:\:& (a, b) \mapsto a + b \\
\cdot:\:& (a, b) \mapsto a \cdot b
\end{aligned}
\]
удовлетворяющие следующим свойствам («аксиомам поля»): $\forall a, b, c \in F$
\begin{enumerate}
\item $a + b = b + a$ (коммутативность по сложению);
\item $(a + b) + c = a + (b + c)$ (ассоциативность по сложению);
\item $\exists\: 0 \in F \colon 0 + a = a + 0 = a$ (существование нулевого элемента);
\item $\exists\: {-a} \in F\colon a + (-a) = (-a) + a = 0$ (существование противоположного элемента);
\item $a(b + c) = ab + ac$ (дистрибутивность; связь между сложением и умножением);
\item $ab = ba$ (коммутативность по умножению);
\item $(ab)c = a(bc)$ (ассоциативность по умножению);
\item $\exists\: 1 \in F\setminus\{0\} : 1 \cdot a = a \cdot 1 = a$ (существование единицы);
\item $a \neq 0 \Rightarrow \exists a^{-1} \in F : a \cdot a^{-1} = a ^{-1} \cdot a = 1$ (существование обратного элемента).
\end{enumerate}
\end{Def}

\begin{Examples} \ 
\begin{itemize}
\item $\mathbb{Q}$ — рациональные числа; 
\item $\mathbb{R}$ — вещественные числа;
\item $\mathbb{C}$ — комплексные числа;
\item $F_2 = \{0, 1\}$, при сложении и умножении по модулю 2.
\end{itemize}
\end{Examples}

\section*{Поле комплексных чисел}
Поле вещественных чисел $\mathbb{R}$ плохо тем, что в нем уравнение $x^2 + 1 = 0$ не имеет решения. Отсюда возникает идея определить поле, удовлетворяющее следующим требованиям:
\begin{itemize}
\item[(T1)] новое поле содержит $\mathbb{R}$;
\item[(Т2)] уравнение $x^2 +1 = 0$ имеет решение.
\end{itemize}
Давайте формально простроим такое поле.

\begin{Def}
Полем $\mathbb{C}$ комплексных чисел называется множество $\{(a, b) \mid a, b \in \mathbb{R}\}$, на котором заданы операции сложения: $(a_1, b_1) + (a_2, b_2) = (a_1 + a_2, b_1 + b_2)$ и умножения: $(a_1, b_1) \cdot (a_2, b_2) = (a_1a_2 - b_1b_2, a_1b_2 + b_1a_2)$.
\end{Def}

\begin{Suggestion}
$\mathbb{C}$ и впрямь является полем.
\end{Suggestion}

\begin{proof}
Операции сложения и умножения введены, осталось только проверить выполнение всех аксиом.
\begin{enumerate}
\item очевидно, так как сложение идет поэлементно;
\item также очевидно;
\item $0 = (0, 0)$;
\item $-(a, b) = (-a, -b)$;
\item почти очевидно (т.е. прямая проверка);
\item ясно (тоже прямая проверка);
\item проверим:
\begin{gather*}
((a_1, b_1) (a_2, b_2)) (a_3, b_3) = (a_1a_2 - b_1b_2, a_1b_2 + b_1a_2) (a_3, b_3) = \\ 
= (a_1a_2a_3 - b_1b_2b_3 - a_1b_2b_3 - b_1a_2b_3, a_1a_2b_3 - b_1b_2b_3 + a_1b_2a_3 + b_1a_2a_3) = \\
= (a_1, b_1)  (a_2a_3 - b_2b_3, a_2b_3 + b_2a_3) = (a_1, b_1)((a_2, b_2) (a_3, b_3));
\end{gather*}
\item $1 = (1, 0)$;
\item $(a, b) \neq 0 \Leftrightarrow a^2 + b^2 \neq 0 \rightarrow (a, b)^{-1} = \left(\frac{a}{a^2 + b^2}, \frac{-b}{a^2 + b^2}\right)$.
\end{enumerate}
\end{proof}

Осталось только проверить, правда ли введенное поле $\mathbb{C}$ удовлетворяет нашим требованиям:
\begin{itemize}
\item[(Т1)] Заметим, что в подмножестве $\mathbb{C}$, состоящим из элементов вида $(a, 0)$ операции сложения и умножения будут работать как в поле вещественных чисел.
\begin{gather*}
(a, 0) + (b, 0) = (a + b, 0) \\
(a, 0) \cdot (b, 0) = (ab, 0)
\end{gather*} 
Следовательно, отображение $a \mapsto (a, 0)$ отождествляет $\mathbb{R}$ с этим подмножеством, то есть $\mathbb{R} \rightarrow \mathbb{C}$. Что нам и требуется.
\item[(Т2)] Примем $i = (0, 1)$. Тогда $i^2 = (0, 1) \cdot (0, 1) = (-1, 0) = -1$. Итого, требование выполнено.
\end{itemize}

Однако запись комплексных чисел в виде упорядоченной пары $(a, b)$ не очень удобна и громоздка. Поэтому преобразуем запись следующим образом:
\[
(a, b) = (a, 0) + (0, b) = (a, 0) + (b, 0) \cdot (0, 1) = a + bi.
\]

Тем самым мы получили реализацию поля $\mathbb{C}$ комплексных чисел как множества \\ $\{a + bi \mid a, b \in \mathbb{R},\ i^2 = -1\}$, с обычным сложением и умножением.

\begin{Def}
Запись $z = a + bi$ называется алгебраической формой комплексного числа $z \in \mathbb{C}$.


$a = \Re z$ — действительная часть числа $z$.


$b = \Im z$ — мнимая часть числа $z$.
\end{Def}

\begin{Def}
Числа вида $z = bi$ (т.е. $\Re z = 0$) называются чисто мнимыми.
\end{Def}

\begin{Def}
Отображение $\mathbb{C} \rightarrow \mathbb{C} : a + bi \mapsto a - bi$ называется (комплексным) сопряжением. Само число $\overline{z} = a - bi$ называется (комплексно) сопряженным к числу $z = a + bi$. 
\end{Def}

\begin{Lemma}
Для любых двух комплексных числе $z, w \in \mathbb{C}$ выполняется, что
\begin{enumerate}
\item $\overline{z + w} = \overline{z} + \overline{w}$;
\item $\overline{zw} = \overline{z} \cdot \overline{w}$.
\end{enumerate}
\end{Lemma}

\begin{proof}
Пусть $z = a + bi$, а $w = c + di$. 
\begin{enumerate}
\item $\overline{z} + \overline{w} = a - bi + c - di = (a + c) - (b + d)i = \overline{z+ w}$
\item $\overline{z} \cdot \overline{w} = (a - bi)(c - di) = ac - adi - bci + bdi^2 = (ac - bd) - (ad + bc)i = \overline{zw}$
\end{enumerate}
\end{proof}

\begin{Comment}
Равенство $z = \overline{z}$ равносильно равенству $\Im z = 0$, то есть $z \in \mathbb{R}$.
\end{Comment}

\subsection*{Геометрическая модель поля $\mathbb{C}$}

Заметим, что поле комплексных числе $\mathbb{C} = \{(a, b) \mid a, b \in \mathbb{R}\}$ равно $\mathbb{R}^2$. Следовательно, комплексные числа можно представить как точки на действительной плоскости $\mathbb{R}^2$, или сопоставить их векторам.

\begin{center}
%{\includegraphics{z-as-vector}}
\begin{tikzpicture}
\begin{axis}
    [
    axis lines = center,
    xtick={0, 3},
    xticklabels = {$0$, $a$},
    ytick = {2},
    yticklabels = {$b$},
    xlabel=$\mathrm{Re}\,z$,
    ylabel=$\mathrm{Im}\,z$,
    ymin=-1,
    ymax=+3,
    xmin=-1,
    xmax=+4
    ]
    \node [right, red] at (axis cs:  3, 2) {$a+bi$};
    \addplot[->] coordinates { (0,0) (3,2) };
    \addplot [dashed, black] coordinates { (3, 0) (3,2) };
    \addplot [dashed, black] coordinates { (0,2) (3,2) };
\end{axis}
\end{tikzpicture}
\end{center}

В таком представлении сложение комплексных чисел сопоставляется со сложением векторов, а сопряжение — с отражением относительно оси $Ox (\Re z)$.

\begin{Def}
Модулем комплексного числа $z = a + bi$ называется длина соответствующего вектора. Обозначение: $|z|; |z| = \sqrt{a^2 + b^2}$.
\end{Def}

Свойства модуля:
\begin{enumerate}
\item $|z| \geqslant 0$, причем $|z| = 0$ тогда и только тогда, когда $z = 0$;
\item $|z + w| \leqslant |z| + |w|$ — неравенство треугольника;
\item $z\cdot\overline{z} = |z|^2$;
\begin{proof}
$(a + bi)(a - bi) = a^2 - (bi)^2 = a^2 + b^2 = |z|^2$.
\end{proof}
\item $|zw| = |z| \cdot |w|$;
\begin{proof}
Возведем в квадрат.
\begin{gather*}
|z|^2 \cdot |w|^2 = z \overline{z} w \overline{w} = (zw)\overline{z}\overline{w} = zw\overline{zw} = |zw|^2
\end{gather*}
\end{proof}
\end{enumerate}

\begin{Comment}
Из свойства 3 следует, что при $z \neq 0$ выполняется: 
\begin{gather*}
z^{-1} = \frac{\overline{z}}{|z|^2}\\
(a + bi)^{-1} = \frac{1}{a + bi} = \frac{a - bi}{a^2 + b^2}.
\end{gather*}
\end{Comment}

\begin{Def}
Аргументом комплексного числа $z \neq 0$  называется всякий угол $\varphi$ такой что 
\[
\cos \varphi = \frac{a}{|z|} = \frac{a}{\sqrt{a^2 + b^2}}; \quad \sin \varphi = \frac{b}{|z|} = \frac{b}{\sqrt{a^2 + b^2}}.
\]
\end{Def}
Неформально говоря, аргумент $z$ — это угол между осью $Ox$ и соответствующим вектором.

\begin{Comment} \ 
\begin{enumerate}
\item Аргумент определен с точностью до $2\pi$.
\item Аргумент $z = 0$ не определен.
\end{enumerate}
\end{Comment}
Для $z \neq 0$ введем множество $\Arg z = \{\text{множество всех аргументов $z$}\}$ — \textit{большой аргумент}. Также введем \textit{малый аргумент} $\arg z$ — это такой $\varphi \in \Arg z$, который удовлетворяет условию $0 \leqslant \varphi < 2\pi$ и, следовательно, определен однозначно. 

Используя аргумент, можно представить комплексное число следующим образом:
\[
\left.
\begin{aligned}
&a = |z|\cos \varphi \\
&b = |z|\sin \varphi
\end{aligned} 
\right| \Rightarrow z = a + bi = |z| \cos \varphi + i |z \sin \varphi = |z|(\cos\varphi + i\sin\varphi)
\] 

\begin{Def}
Запись $z = |z|(\cos\varphi + i\sin\varphi)$ называется тригонометрической формой комплексного числа $z$.
\end{Def}

\begin{Comment}
\[
r_1(\cos\varphi_1 + i\sin\varphi_1) = r_2(\cos\varphi_2 + i\sin\varphi_2) \Leftrightarrow
\left\{
\begin{aligned}
&r_1 = r_2 \\
&\varphi_1 = \varphi_2 + 2\pi n, \quad n \in \mathbb{Z}
\end{aligned}
\right.
\]
\end{Comment}

\end{document}
