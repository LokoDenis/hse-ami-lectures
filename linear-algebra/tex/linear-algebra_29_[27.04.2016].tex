\documentclass[a4paper,12pt]{article}

%% Работа с русским языком
\usepackage{cmap}					% поиск в PDF
\usepackage{mathtext} 				% русские буквы в формулах
\usepackage[T2A]{fontenc}			% кодировка
\usepackage[utf8]{inputenc}			% кодировка исходного текста
\usepackage[english,russian]{babel}	% локализация и переносы

%% Отступы между абзацами и в начале абзаца 
\setlength{\parindent}{0pt}
\setlength{\parskip}{\medskipamount}

%% Изменяем размер полей
\usepackage[top=0.5in, bottom=0.75in, left=0.625in, right=0.625in]{geometry}

%% Графика
\usepackage[pdftex]{graphicx}
\graphicspath{{images/}}

%% Различные пакеты для работы с математикой
\usepackage{mathtools}				% Тот же amsmath, только с некоторыми поправками

\usepackage{amssymb}				% Математические символы
\usepackage{amsthm}					% Пакет для написания теорем
\usepackage{amstext}
\usepackage{array}
\usepackage{amsfonts}
\usepackage{icomma}					% "Умная" запятая: $0,2$ --- число, $0, 2$ --- перечисление
\usepackage{bbm}				    % Для красивого (!) \mathbb с  буквами и цифрами
\usepackage{enumitem}               % Для выравнивания itemise (\begin{itemize}[align=left])

% Номера формул
\mathtoolsset{showonlyrefs=true} % Показывать номера только у тех формул, на которые есть \eqref{} в тексте.

% Ссылки
\usepackage[colorlinks=true, urlcolor=blue]{hyperref}

% Шрифты
\usepackage{euscript}	 % Шрифт Евклид
\usepackage{mathrsfs}	 % Красивый матшрифт

% Свои команды\textbf{}
\DeclareMathOperator{\sgn}{\mathop{sgn}}

% Перенос знаков в формулах (по Львовскому)
\newcommand*{\hm}[1]{#1\nobreak\discretionary{}
{\hbox{$\mathsurround=0pt #1$}}{}}

% Графики
\usepackage{tikz}
\usepackage{pgfplots}
%\pgfplotsset{compat=1.12}

% Изменим формат \section и \subsection:
\usepackage{titlesec}
\titleformat{\section}
{\vspace{1cm}\centering\LARGE\bfseries}	% Стиль заголовка
{}										% префикс
{0pt}									% Расстояние между префиксом и заголовком
{} 										% Как отображается префикс
\titleformat{\subsection}				% Аналогично для \subsection
{\Large\bfseries}
{}
{0pt}
{}

% Информация об авторах
\author{Группа лектория ФКН ПМИ 2015-2016 \\
	Анастасия Иовлева \\
	Ксюша Закирова \\
	Руслан Хайдуров}
\title{Лекции по предмету \\
	\textbf{Линейная алгебра и геометрия}}
\date{2016 год}

\newtheorem*{Def}{Определение}
\newtheorem*{Lemma}{Лемма}
\newtheorem*{Suggestion}{Предложение}
\newtheorem*{Examples}{Пример}
\newtheorem*{Comment}{Замечание}
\newtheorem*{Consequence}{Следствие}
\newtheorem*{Theorem}{Теорема}
\newtheorem*{Statement}{Утверждение}
\newtheorem*{Task}{Упражнение}
\newtheorem*{Designation}{Обозначение}
\newtheorem*{Generalization}{Обобщение}
\newtheorem*{Thedream}{Предел мечтаний}
\newtheorem*{Properties}{Свойства}

\renewcommand{\mathbb}{\mathbbm}
\renewcommand{\Re}{\mathrm{Re\:}}
\renewcommand{\Im}{\mathrm{Im\:}}
\newcommand{\Arg}{\mathrm{Arg\:}}
\renewcommand{\arg}{\mathrm{arg\:}}
\newcommand{\Mat}{\mathrm{Mat}}
\newcommand{\id}{\mathrm{id}}
\newcommand{\isom}{\xrightarrow{\sim}} 
\newcommand{\leftisom}{\xleftarrow{\sim}}
\newcommand{\Hom}{\mathrm{Hom}}
\newcommand{\Ker}{\mathrm{Ker}\:}
\newcommand{\rk}{\mathrm{rk}\:}
\newcommand{\diag}{\mathrm{diag}}
\newcommand{\ort}{\mathrm{ort}}
\newcommand{\pr}{\mathrm{pr}}
\newcommand{\vol}{\mathrm{vol\:}}

\renewcommand{\epsilon}{\varepsilon}
\renewcommand{\phi}{\varphi}
\newcommand{\e}{\mathbb{e}}
\renewcommand{\l}{\lambda}
\renewcommand{\C}{\mathbb{C}}
\newcommand{\R}{\mathbb{R}}
\newcommand{\E}{\mathbb{E}}

\newcommand{\vvector}[1]{\begin{pmatrix}{#1}_1 \\\vdots\\{#1}_n\end{pmatrix}}
\renewcommand{\vector}[1]{({#1}_1, \ldots, {#1}_n)}

\begin{document}
\renewcommand{\f}{\mathbb{f}}
\newcommand{\ort}{\mathrm{ort}}
\renewcommand{\Pr}{\mathrm{Pr}}

\section{Лекция 29 от 27.04.2016}
Пусть $\mathbb{E}$ --- векторное пространство.
\begin{Def}
	Векторы $x,y$ называются ортогональными,если $(x,y)=0$. Обозначение $x\perp y$. 
\end{Def}
\begin{Def}
	\ \\
	Пусть $S \subset \mathbb{E}$ --- произвольное подпространство. Множество $S^{\perp} = \{x\in \mathbb{E}\; |\; (x,y) = 0\;\forall y \in S\}$ называется ортогональным дополнением к S
\end{Def}
\begin{Comment}
	\ 
	\begin{enumerate}
		\item $S^\perp$ --- подпространство
		\item $S^\perp = \langle S \rangle^\perp$
	\end{enumerate}
\end{Comment}
\begin{Suggestion}
	\ 
	\begin{enumerate}
		\item $\dim S^\perp = n - \dim S$
		\item $\mathbb{E} = S \oplus S^\perp$
		\item $(S^\perp)^\perp$
	\end{enumerate}
\end{Suggestion}

\begin{proof}
	\begin{enumerate}
	\item Выделим в $S$ базис $(e_1, \ldots, e_k)$ и дополним его векторами $(e_{k+1}, \ldots, e_n)$ до базиса $\mathbb{E}$.
	\begin{gather*}
		x = x_1e_1 + \ldots + x_ne_n \in\mathbb{E}\\
		x \in S^\perp \Leftrightarrow (x,e_i) = 0\; \forall i = 1,\ldots, k\\
		(x,e_i) = (e_1, e_i)x_1 + (e_2, e_i) x_2 + \ldots, (e_n, e_i) x_n= 0		
	\end{gather*}
	Получим однорудную СЛУ $G\begin{pmatrix}
	x_1\\x_2\\ \vdots\\ x_n
	\end{pmatrix} = 0$, причём $G \in Mat_{k\times n}(\mathbb{R})$. $\rk G = k$, поскольку левый верхний минор (совпадает с таковым в матрице Грама) $k \times k$ больше нуля. Размерность пространства решений $\dim S^\perp = n - \rk G = n - \dim S$.
	\item Поскольку $(x,x) = 0 \Leftrightarrow x = 0$, то $S \cap S^\perp = \{0\}$, а значит $\mathbb{E} = S \oplus S^\perp$
	\item $S \subset (S^\perp)^\perp$ --- всегда.
	\[
		\dim (S^\perp)^\perp = n - \dim S^\perp = k \Rightarrow S = (S^\perp)^\perp
	\]
\end{enumerate}
\end{proof}
Пусть $x \in \mathbb{E}$. Значит существует единственное представление его в виде $x = y + z$, где $y \in S,\; z \in S^\perp$.
\begin{Def}
	$y$ называется ортогональной проекцией вектора $x$ на подпространство $S$, обозначается $y = \Pr_S x$. $z$ называется ортогональной составляющей вектора $x$ вдоль $S$, обозначается $\ort_S x$
\end{Def}
\begin{Def}
	Базис $(e_1, \ldots, e_n)$ в $\mathbb{E}$ называется ортогональным, если $(e_i, e_j) = 0\; \forall i\neq j$. (Что то же самое, $G(e_1, \ldots, e_n)$ диагональна). Базис называется ортонормированным, если дополнительно $(e_i, e_i) = 1\; \forall i$. (Что то же самое, $G(e_1, \ldots, e_n) = E$).
\end{Def}
\begin{Comment}
	Если $(e_1,\ldots, e_n)$ ортогональный базис, то $\left(\cfrac{e_1}{|e_1|}, \ldots, \cfrac{e_n}{|e_n|}\right)$ ортонормированный.
\end{Comment}
\begin{Theorem}
	В любом конечномерном евклидовом пространстве существует ортонормированный базис.
\end{Theorem}
\begin{proof}
	Следует из того, что всякую квадратичную форму можно привести к нормальному виду. 
\end{proof}
Пусть $(e_1, \ldots, e_n)$ --- ортонормированный базис. Пусть также есть ещё один базис $(e_1', \ldots, e_n')$, причём $(e_1', \ldots, e_n') = (e_1, \ldots, e_n)C$. 
\begin{Suggestion}
	$(e_1', \ldots, e_n')$ --- ортонормированный тогда и только тогда, когда $C^{-1} = C^T$
\end{Suggestion}
\begin{proof}
	$(e_1', \ldots, e_n')$ ортонормированный, следовательно, $G(e_1', \ldots, e_n') = E$.
	\[
		G(e_1', \ldots, e_n') = C^TG(e_1, \ldots, e_n)C = C^TC = E
	\]
	Отсюда и получаем требуемое.
\end{proof}
\begin{Def}
	Матрица $C$ в таком случае называется ортогональной.
\end{Def}
\begin{Properties}
	\begin{gather*}
		\sum\limits_{k = 1}^{n}c_{ki}c_{kj} = \delta_{ij}\\
		\Updownarrow\\
		C^TC= E	\Leftrightarrow C^T = C^{-1}\Leftrightarrow CC^T = E\\
		\Updownarrow\\
		\sum\limits_{k = 1}^{n}c_{ik}c_{jk} = \delta_{ij}\\
	\end{gather*}
\end{Properties}
\begin{Examples}
	$C = \begin{pmatrix}
	\cos \varphi& -\sin \varphi\\
	\sin \varphi& \cos \varphi
	\end{pmatrix}$ --- матрица поворота на угол $\varphi$ в $\mathbb{R}^2$.
\end{Examples}
\par Ещё свойство: $\det C = \pm 1$.
\par Пусть $S \subset \mathbb{E}$ --- подпространство. $(e_1, \ldots, e_k)$ его ортогональный базис. $x \in \mathbb{E}$.
\begin{Suggestion}
	$\Pr_S x = \sum\limits_{i = 1}^{k}\cfrac{(x,e_i)}{(e_i, e_i)}e_i$. В частности, если базис ортонормированный, $\Pr_S x = \sum\limits_{i = 1}^{k}(x,e_i)e_i$
\end{Suggestion}
\begin{proof}
	\begin{gather*}
		\Pr_S x = \sum\limits_{i = 1}^{k}\lambda_ie_i\\
		x = \Pr_S x + \ort_S x\Leftrightarrow (x, e_i) = (\Pr_S x, e_i) + \underbrace{(\ort_S x, e_i)}_{=0}\\
		\Rightarrow (x, e_i) = \lambda (e_i, e_i)\\
		\lambda_i = \cfrac{(x_1, e_i)}{(e_i, e_i)}
	\end{gather*}
\end{proof}
Пусть есть базис $(e_1, \ldots, e_n)$ в $\mathbb{E}$. Процесс ортогонализации даёт ортогональный базис $(f_1, \ldots, f_n)$
\begin{gather*}
	f_1 = e_1\\
	f_2 \in e_2 + \langle e_1 \rangle\\
	\ldots\\
	f_n  \in e_n + \langle e_1, \ldots, e_{n - 1} \rangle\\
	\langle f_1, \ldots, f_i \rangle = \langle e_1, \ldots, e_i\rangle\; \forall i= 2, \ldots, n
\end{gather*}
\begin{Suggestion}
	$f_i = \ort_{\langle e_1, \ldots, e_{i-1}\rangle} e_i\; \forall i = 1, \ldots, n$
\end{Suggestion}
\begin{proof}
	\begin{gather*}
		e_i = f_i + \lambda_1f_1 + \ldots + \lambda_{i -1} f_{i -1}\\
		f_i \perp \langle e_1, \ldots, e_{i -1} \rangle = \langle f_1, \ldots, f_{i - 1} = f_i = \ort_{\langle e_1, \ldots, e_{i-1}\rangle} e_i
	\end{gather*}
\end{proof}
\begin{Theorem}[Пифагор]
	$|x+y| = |x|^2 + |y|^2$, если $x \perp y$
\end{Theorem}
\begin{proof}
	$|x + y|^2 = (x+y, x+y) = (x,x) + (y,y) + (x,y) + (y,x) = (x,x) + (y,y) = |x|^2 + |y|^2$
\end{proof}
\begin{Def}
	Расстоянием между $x,y$ называется число $\rho(x,y) = |x-y|$
\end{Def}
\begin{Suggestion}[Неравенство треугольника]
	$\rho(a,b) + \rho(b,c) \geqslant \rho(a,c)$.
\end{Suggestion}
\begin{proof}
	Пусть $x = a-b,\; y = b - c$. Тогда $a-c = x -y$. Достаточно доказать, что $|x| + |y| \geqslant |x + y|$
	\[
		|x+y|^2 = |x| + |y| \geqslant |x+y|
	\]
\end{proof}
Пусть $P$ и $Q$ --- два произвольных подмножества $\mathbb{E}$.
\begin{Def}
	Расстояние между $P$ и $Q$ определяется как $\rho (P,Q) = \inf \{\rho(x,y)\;|\; x\in P,\ y\in Q\}$.
\end{Def}
Пусть $x \in \mathbb{E}$, а $U \subset \mathbb{E}$ --- подпространство.
\begin{Theorem}
	$\rho (x,U) = |\ort_U x|$, причём $\Pr_U x$ --- единственный ближайший к $x$ вектор из $U$.
\end{Theorem}
\begin{proof}
	Пусть $y = \Pr_U x,\; z = \ort_U x$. Тогда, если $y' \in U\backslash \{0\}$, то 
	\[
		\rho(x+y, y') = |x + y - y'| = |z - y'| = \sqrt{|z|^2 + |y'|^2} > |z| = \rho(x,y)
	\]
\end{proof}
Пусть $U' \subset \mathbb{E}$ --- подпространство. $x \in \mathbb{E}$, $(e_1, \ldots, e_k)$ --- базис $U$.
\begin{Theorem}
	$(\rho(x,U))^2 = \cfrac{\det G(e_1, \ldots, e_k, x)}{\det G(e_1, \ldots, e_k)}$
\end{Theorem}
\begin{proof}
	Если $x \in U$, то $\rho (x,U) = 0$, но с другой стороны, $\det G(e_1, \ldots, e_k, x) = 0$, поскольку эти векторы линейно зависимы. Если же $x \notin U$, то $\rho(x,U) = |\ort_U x|$. Ортогонализация даёт нам $(f_1, \ldots, f_k, z)$, причём $|z|^2 = (z,z) = \cfrac{\delta_{k+1}}{\delta_{k}}$
\end{proof}
\end{document}