\documentclass[a4paper,12pt]{article}

%% Работа с русским языком
\usepackage{cmap}					% поиск в PDF
\usepackage{mathtext} 				% русские буквы в формулах
\usepackage[T2A]{fontenc}			% кодировка
\usepackage[utf8]{inputenc}			% кодировка исходного текста
\usepackage[english,russian]{babel}	% локализация и переносы

%% Отступы между абзацами и в начале абзаца 
\setlength{\parindent}{0pt}
\setlength{\parskip}{\medskipamount}

%% Изменяем размер полей
\usepackage[top=0.5in, bottom=0.75in, left=0.625in, right=0.625in]{geometry}

%% Графика
\usepackage[pdftex]{graphicx}
\graphicspath{{images/}}

%% Различные пакеты для работы с математикой
\usepackage{mathtools}				% Тот же amsmath, только с некоторыми поправками

\usepackage{amssymb}				% Математические символы
\usepackage{amsthm}					% Пакет для написания теорем
\usepackage{amstext}
\usepackage{array}
\usepackage{amsfonts}
\usepackage{icomma}					% "Умная" запятая: $0,2$ --- число, $0, 2$ --- перечисление
\usepackage{bbm}				    % Для красивого (!) \mathbb с  буквами и цифрами
\usepackage{enumitem}               % Для выравнивания itemise (\begin{itemize}[align=left])

% Номера формул
\mathtoolsset{showonlyrefs=true} % Показывать номера только у тех формул, на которые есть \eqref{} в тексте.

% Ссылки
\usepackage[colorlinks=true, urlcolor=blue]{hyperref}

% Шрифты
\usepackage{euscript}	 % Шрифт Евклид
\usepackage{mathrsfs}	 % Красивый матшрифт

% Свои команды\textbf{}
\DeclareMathOperator{\sgn}{\mathop{sgn}}

% Перенос знаков в формулах (по Львовскому)
\newcommand*{\hm}[1]{#1\nobreak\discretionary{}
{\hbox{$\mathsurround=0pt #1$}}{}}

% Графики
\usepackage{tikz}
\usepackage{pgfplots}
%\pgfplotsset{compat=1.12}

% Изменим формат \section и \subsection:
\usepackage{titlesec}
\titleformat{\section}
{\vspace{1cm}\centering\LARGE\bfseries}	% Стиль заголовка
{}										% префикс
{0pt}									% Расстояние между префиксом и заголовком
{} 										% Как отображается префикс
\titleformat{\subsection}				% Аналогично для \subsection
{\Large\bfseries}
{}
{0pt}
{}

% Информация об авторах
\author{Группа лектория ФКН ПМИ 2015-2016 \\
	Анастасия Иовлева \\
	Ксюша Закирова \\
	Руслан Хайдуров}
\title{Лекции по предмету \\
	\textbf{Линейная алгебра и геометрия}}
\date{2016 год}

\newtheorem*{Def}{Определение}
\newtheorem*{Lemma}{Лемма}
\newtheorem*{Suggestion}{Предложение}
\newtheorem*{Examples}{Пример}
\newtheorem*{Comment}{Замечание}
\newtheorem*{Consequence}{Следствие}
\newtheorem*{Theorem}{Теорема}
\newtheorem*{Statement}{Утверждение}
\newtheorem*{Task}{Упражнение}
\newtheorem*{Designation}{Обозначение}
\newtheorem*{Generalization}{Обобщение}
\newtheorem*{Thedream}{Предел мечтаний}
\newtheorem*{Properties}{Свойства}

\renewcommand{\mathbb}{\mathbbm}
\renewcommand{\Re}{\mathrm{Re\:}}
\renewcommand{\Im}{\mathrm{Im\:}}
\newcommand{\Arg}{\mathrm{Arg\:}}
\renewcommand{\arg}{\mathrm{arg\:}}
\newcommand{\Mat}{\mathrm{Mat}}
\newcommand{\id}{\mathrm{id}}
\newcommand{\isom}{\xrightarrow{\sim}} 
\newcommand{\leftisom}{\xleftarrow{\sim}}
\newcommand{\Hom}{\mathrm{Hom}}
\newcommand{\Ker}{\mathrm{Ker}\:}
\newcommand{\rk}{\mathrm{rk}\:}
\newcommand{\diag}{\mathrm{diag}}
\newcommand{\ort}{\mathrm{ort}}
\newcommand{\pr}{\mathrm{pr}}
\newcommand{\vol}{\mathrm{vol\:}}

\renewcommand{\epsilon}{\varepsilon}
\renewcommand{\phi}{\varphi}
\newcommand{\e}{\mathbb{e}}
\renewcommand{\l}{\lambda}
\renewcommand{\C}{\mathbb{C}}
\newcommand{\R}{\mathbb{R}}
\newcommand{\E}{\mathbb{E}}

\newcommand{\vvector}[1]{\begin{pmatrix}{#1}_1 \\\vdots\\{#1}_n\end{pmatrix}}
\renewcommand{\vector}[1]{({#1}_1, \ldots, {#1}_n)}

\begin{document}
\renewcommand{\f}{\mathbb{f}}
\newcommand{\ort}{\mathrm{ort}}
\newcommand{\pr}{\mathrm{pr}}

\section{Лекция 29 от 27.04.2016}
Пусть $\mathbb{E}$ --- евклидово пространство, $\dim \E = n$.
\begin{Def}
	Векторы $x,y$ называются ортогональными, если $(x,y)=0$. Обозначение: \\$x\perp y$. 
\end{Def}

\begin{Def}
	Пусть $S \subseteq \mathbb{E}$ --- произвольное подпространство. Ортогональным дополнением к $S$ называется множество $S^{\perp} = \{x\in \mathbb{E}\; |\; (x,y) = 0\;\forall y \in S\}$.
\end{Def}

\begin{Comment}\
	\begin{enumerate}
		\item $S^\perp$ --- подпространство в $\E$.
		\item $S^\perp = \langle S \rangle^\perp$.
	\end{enumerate}
\end{Comment}

\begin{Suggestion}
Пусть $S$ --- подпространство в $\E$. Тогда:
	\begin{enumerate}
		\item $\dim S^\perp = n - \dim S$;
		\item $\mathbb{E} = S \oplus S^\perp$;
		\item $(S^\perp)^\perp = S$.
	\end{enumerate}
\end{Suggestion}

\begin{proof}\
	\begin{enumerate}
	\item Выделим в $S$ базис $(e_1, \ldots, e_k)$ и дополним его векторами $(e_{k+1}, \ldots, e_n)$ до базиса $\mathbb{E}$. Рассмотрим вектор $x \in \E$ и представим его в виде $x_1e_1 + \ldots + x_ne_n$. Если $x \in S^\perp$, то это то же самое, если $(x, e_i) = 0$ для $i = 1 \ldots k$. Итого:
	\begin{gather*}
		(x,e_i) = (e_1, e_i)x_1 + (e_2, e_i) x_2 + \ldots, (e_n, e_i) x_n= 0, \quad i = 1 \ldots k		
	\end{gather*}
	Получим однородную СЛУ $G\begin{pmatrix}
	x_1\\x_2\\ \vdots\\ x_n
	\end{pmatrix} = 0$, где $G \in Mat_{k\times n}(\mathbb{R})$ и $g_{ij} = (e_i, e_j)$. Заметим, что $\rk G =  k$, так как это часть матрицы Грама, и ее левый верхний $k\times k$ минор больше нуля. Следовательно, размерность пространства решений $\dim S^\perp = n - \rk G = n - \dim S$.
	\item Из предыдущего пункта получаем, что $\dim S + \dim S^\perp = n$. Вместе с тем, поскольку $(x,x) = 0$ тогда и только тогда, когда $x = 0$, то $S \cap S^\perp = \{0\}$. Следовательно, $\mathbb{E} = S \oplus S^\perp$.
	\item $S \subset (S^\perp)^\perp$ --- всегда. Вместе с тем, $\dim (S^\perp)^\perp = n - \dim S^\perp = n - (n - k) =  k = \dim S$. И так как размерности совпадают, то $S = (S^\perp)^\perp$.
\end{enumerate}
\end{proof}

Итак, мы теперь знаем, что $\E = S \oplus S^\perp$. Значит, для $x \in \mathbb{E}$ существует единственное представление его в виде $x = y + z$, где $y \in S,\; z \in S^\perp$.
\begin{Def}
	Вектор $y$ называется ортогональной проекцией вектора $x$ на подпространство $S$. Обозначение: $\pr_S x$. \\ Вектор $z$ называется ортогональной составляющей вектора $x$ вдоль подпространства $S$. Обозначение: $\ort_S x$.
\end{Def}
\begin{Def}
	Базис $(e_1, \ldots, e_n)$ в $\mathbb{E}$ называется ортогональным, если $(e_i, e_j) = 0\; \forall i\neq j$. Это равносильно тому, что $G(e_1, \ldots, e_n)$ диагональна. \\Базис называется ортонормированным, если дополнительно $(e_i, e_i) = 1\; \forall i$. Это равносильно тому, что $G(e_1, \ldots, e_n) = E$.
\end{Def}
\begin{Comment}
	Если $(e_1,\ldots, e_n)$ ортогональный базис, то $\left(\cfrac{e_1}{|e_1|}, \ldots, \cfrac{e_n}{|e_n|}\right)$ ортонормированный.
\end{Comment}
\begin{Theorem}
	В любом конечномерном евклидовом пространстве существует ортонормированный базис.
\end{Theorem}
\begin{proof}
	Следует из того, что всякую положительно определенную квадратичную форму можно привести к нормальному виду. 
\end{proof}
Пусть $(e_1, \ldots, e_n)$ --- ортонормированный базис в $\E$. Пусть также есть ещё один базис $(e_1', \ldots, e_n')$, причём $(e_1', \ldots, e_n') = (e_1, \ldots, e_n)C$. 
\begin{Suggestion}
	$(e_1', \ldots, e_n')$ --- ортонормированный тогда и только тогда, когда $C^TC = E$ или, что то же самое, $C^{-1} = C^T$.
\end{Suggestion}
\begin{proof}
	Условие, что базис $(e_1', \ldots, e_n')$ является ортонормированным, равносильно тому, что $G(e_1', \ldots, e_n') = E$. С другой стороны, $G(e_1', \ldots, e_n') = C^TG(e_1, \ldots, e_n)C$, причем аналогично $G(e_1, \ldots, e_n) = E$. Откуда и следует, что $C^TC = E$.
\end{proof}
\begin{Def}
	Матрица $C$ в таком случае называется ортогональной.
\end{Def}

\begin{Properties}\
\begin{enumerate}
\item $C^TC = E$, значит, $C^T = C^{-1}$, и тогда $CC^T = E$. Итого, получаем:
$$
\sum\limits_{k = 1}^{n}c_{ki}c_{kj} = \delta_{ij} = \sum\limits_{k = 1}^{n}c_{ik}c_{jk}
$$
Напомним, что $\delta_{ij}$ это символ Кронекера.
\item $\det C = \pm 1$.
\end{enumerate}
\end{Properties}

\begin{Examples}
	$C = \begin{pmatrix}
	\cos \varphi& -\sin \varphi\\
	\sin \varphi& \cos \varphi
	\end{pmatrix}$ --- матрица поворота на угол $\varphi$ в $\mathbb{R}^2$.
\end{Examples}

Пусть $S \subseteq \mathbb{E}$ --- подпространство, $(e_1, \ldots, e_k)$ --- его ортогональный базис, $x \in \mathbb{E}$.
\begin{Suggestion}
	$\pr_S x = \sum\limits_{i = 1}^{k}\cfrac{(x,e_i)}{(e_i, e_i)}e_i$. В частности, если базис ортонормированный, \\$\pr_S x = \sum\limits_{i = 1}^{k}(x,e_i)e_i$
\end{Suggestion}
\begin{proof}
Представим вектор $x$ в виде суммы $x = \pr_Sx + \ort_Sx$. Тогда:
$$
(x, e_i) = (\pr_S x, e_i) + \underbrace{(\ort_S x, e_i)}_{=0} = (\pr_S x, e_i) \quad i = 1, \ldots, k.
$$
Вместе с тем, $\pr_S x = \sum\limits_{j = 1}^{k}\l_je_j$, следовательно, $(x, e_i) = \sum\limits_{j = 1}^k\l_j(e_j, e_i)$.
Но так как базис ортогональный, все слагаемые, кроме одного, занулятся, и останется только $(x, e_i) = \l_i(e_i, e_i)$. Откуда и следует, что $\l_i = \cfrac{(x, e_i)}{(e_i, e_i)}$.
\end{proof}

Пусть есть базис $(e_1, \ldots, e_n)$ в $\mathbb{E}$. Процесс ортогонализации Грама-Шмидта даёт ортогональный базис $(f_1, \ldots, f_n)$, причем:
\begin{align*}
	& f_1 = e_1\\
	& f_2 \in e_2 + \langle e_1 \rangle\\
	& \ldots\\
	& f_n  \in e_n + \langle e_1, \ldots, e_{n - 1} \rangle
\end{align*}
Точно так же можно заметить, что $\langle f_1, \ldots, f_i \rangle = \langle e_1, \ldots, e_i\rangle$ для всех $i= 1, \ldots, n$.
\begin{Suggestion}
	$f_i = \ort_{\langle e_1, \ldots, e_{i-1}\rangle} e_i$ для всех $i = 1, \ldots, n$.
\end{Suggestion}
\begin{proof}
Воспользовавшись равенством линейных оболочек, получаем, что \\$e_i \in f_i + \langle f_1, \ldots, f_{i-1}\rangle$. Следовательно, данный базисный вектор можно представить в виде $e_i = f_i + \lambda_1f_1 + \ldots + \lambda_{i -1} f_{i -1}$. И из того, что $f_i \perp \langle e_1, \ldots, e_{i -1} \rangle = \langle f_1, \ldots, f_{i - 1}\rangle$ как раз и получаем, что $f_i = \ort_{\langle e_1, \ldots, e_{i-1}\rangle} e_i$.
\end{proof}

\begin{Examples}
Данное рассуждение проще понять, если представить себе частный случай для $\E = \R^3$.

У нас зафиксированы векторы $e_1, e_2, e_3$, и мы их ортогонализируем. Для начала, $f_1 = e_1$. Вектор $f_2$ получается как проекция вектора $e_2$ на прямую, ортогональную $f_1$. А вектор $f_3$~--- как проекция $e_3$ на прямую, ортогональную плоскости, образованной векторами $f_1$ и $f_2$. Аналогично для пространств большей размерности.
\end{Examples}

\begin{Theorem}[Пифагора]
Если $x, y \in \E$ и $x \perp y$, то $|x+y| = |x|^2 + |y|^2$.
\end{Theorem}

\begin{proof}
	$$|x + y|^2 = (x+y, x+y) = (x, x) + (y, y) + \underbrace{(x, y)}_{=0} + \underbrace{(y, x)}_{=0} = (x, x) + (y, y) = |x|^2 + |y|^2$$
\end{proof}
Рассмотрим векторы $x, y \in \E$.
\begin{Def}
	Расстоянием между векторами $x$ и $y$ называется число $\rho(x,y) := |x-y|$.
\end{Def}

\begin{Suggestion}[Неравенство треугольника]
	$\rho(a,b) + \rho(b,c) \geqslant \rho(a,c)$ при $a, b, c \in \E$.
\end{Suggestion}

\begin{proof}
	Пусть $x = a-b,\; y = b - c$. Тогда $a-c = x +y$. Теперь достаточно доказать, что $|x| + |y| \geqslant |x + y|$. Для этого рассмотрим $|x + y|^2$.
	\begin{gather*}
	|x+y|^2 = (x, x) + 2(x, y) + (y, y) = |x|^2 + 2(x, y) + |y|^2 \leqslant |x|^2 + 2|x||y|+ |y|^2 = (|x| + |y|)^2
	\end{gather*}
	Сравнивая начало и конец неравенства, получаем, что $|x+y| \leqslant |x| + |y|$.
\end{proof}
Пусть $P$ и $Q$ --- два произвольных подмножества $\mathbb{E}$.
\begin{Def}
	Расстоянием между $P$ и $Q$ называют величину 
	$$
	\rho (P,Q) := \inf \{\rho(x,y) \mid x\in P,\ y\in Q\}.
	$$
\end{Def}
Пусть $x \in \mathbb{E}$ и $U \subseteq \mathbb{E}$ --- подпространство.
\begin{Theorem}
	$\rho (x,U) = |\ort_U x|$, причём $\pr_U x$ --- единственный ближайший к $x$ вектор из $U$.
\end{Theorem}
\begin{proof}
	Пусть $y = \pr_U x$ и $z = \ort_U x$. Пусть также $y' \in U\backslash \{0\}$, тогда:
	\[
		\rho(x, y + y') = |x - y - y'| = |z - y'| = \sqrt{|z|^2 + \underbrace{|y'|^2}_{>0}} > |z| = \rho(x,y).
	\]
	Из того, что вектор $z$, которым мы огранили снизу, определяется однозначно, и следует, что существует единственный ближайший вектор к $x$ из $U$.
\end{proof}
Пусть $U \subseteq \mathbb{E}$ --- подпространство, $x \in \mathbb{E}$, $(e_1, \ldots, e_k)$ --- базис $U$.
\begin{Theorem}
	$(\rho(x,U))^2 = \cfrac{\det G(e_1, \ldots, e_k, x)}{\det G(e_1, \ldots, e_k)}$
\end{Theorem}
\begin{proof}
	Разобьем на два случая: когда $x$ лежит в $U$ и когда не лежит.
	\begin{enumerate}
	\item $x \in U$. Тогда $\rho (x,U) = 0$. Но с другой стороны, $\det G(e_1, \ldots, e_k, x) = 0$, поскольку эти векторы линейно зависимы, и значит, равенство выполняется.
	\item $x \notin U$. Тогда $\rho(x,U) = |\ort_U x| = |z|$. Ортогонализация Грама-Шмидта к $(e_1, \ldots, e_k, x)$ даст нам $(f_1, \ldots, f_k, z)$, причём $|z|^2 = (z,z) = \cfrac{\delta_{k+1}}{\delta_{k}} = \cfrac{\det G(e_1, \ldots, e_k, x)}{\det G(e_1, \ldots, e_k)}$.
	\end{enumerate}
\end{proof}
\end{document}
