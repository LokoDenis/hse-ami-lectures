\documentclass[a4paper,12pt]{article}

%% Работа с русским языком
\usepackage{cmap}					% поиск в PDF
\usepackage{mathtext} 				% русские буквы в формулах
\usepackage[T2A]{fontenc}			% кодировка
\usepackage[utf8]{inputenc}			% кодировка исходного текста
\usepackage[english,russian]{babel}	% локализация и переносы

%% Отступы между абзацами и в начале абзаца 
\setlength{\parindent}{0pt}
\setlength{\parskip}{\medskipamount}

%% Изменяем размер полей
\usepackage[top=0.5in, bottom=0.75in, left=0.625in, right=0.625in]{geometry}

%% Графика
\usepackage[pdftex]{graphicx}
\graphicspath{{images/}}

%% Различные пакеты для работы с математикой
\usepackage{mathtools}				% Тот же amsmath, только с некоторыми поправками

\usepackage{amssymb}				% Математические символы
\usepackage{amsthm}					% Пакет для написания теорем
\usepackage{amstext}
\usepackage{array}
\usepackage{amsfonts}
\usepackage{icomma}					% "Умная" запятая: $0,2$ --- число, $0, 2$ --- перечисление
\usepackage{bbm}				    % Для красивого (!) \mathbb с  буквами и цифрами
\usepackage{enumitem}               % Для выравнивания itemise (\begin{itemize}[align=left])

% Номера формул
\mathtoolsset{showonlyrefs=true} % Показывать номера только у тех формул, на которые есть \eqref{} в тексте.

% Ссылки
\usepackage[colorlinks=true, urlcolor=blue]{hyperref}

% Шрифты
\usepackage{euscript}	 % Шрифт Евклид
\usepackage{mathrsfs}	 % Красивый матшрифт

% Свои команды\textbf{}
\DeclareMathOperator{\sgn}{\mathop{sgn}}

% Перенос знаков в формулах (по Львовскому)
\newcommand*{\hm}[1]{#1\nobreak\discretionary{}
{\hbox{$\mathsurround=0pt #1$}}{}}

% Графики
\usepackage{tikz}
\usepackage{pgfplots}
%\pgfplotsset{compat=1.12}

% Изменим формат \section и \subsection:
\usepackage{titlesec}
\titleformat{\section}
{\vspace{1cm}\centering\LARGE\bfseries}	% Стиль заголовка
{}										% префикс
{0pt}									% Расстояние между префиксом и заголовком
{} 										% Как отображается префикс
\titleformat{\subsection}				% Аналогично для \subsection
{\Large\bfseries}
{}
{0pt}
{}

% Информация об авторах
\author{Группа лектория ФКН ПМИ 2015-2016 \\
	Анастасия Иовлева \\
	Ксюша Закирова \\
	Руслан Хайдуров}
\title{Лекции по предмету \\
	\textbf{Линейная алгебра и геометрия}}
\date{2016 год}

\newtheorem*{Def}{Определение}
\newtheorem*{Lemma}{Лемма}
\newtheorem*{Suggestion}{Предложение}
\newtheorem*{Examples}{Пример}
\newtheorem*{Comment}{Замечание}
\newtheorem*{Consequence}{Следствие}
\newtheorem*{Theorem}{Теорема}
\newtheorem*{Statement}{Утверждение}
\newtheorem*{Task}{Упражнение}
\newtheorem*{Designation}{Обозначение}
\newtheorem*{Generalization}{Обобщение}
\newtheorem*{Thedream}{Предел мечтаний}
\newtheorem*{Properties}{Свойства}

\renewcommand{\mathbb}{\mathbbm}
\renewcommand{\Re}{\mathrm{Re\:}}
\renewcommand{\Im}{\mathrm{Im\:}}
\newcommand{\Arg}{\mathrm{Arg\:}}
\renewcommand{\arg}{\mathrm{arg\:}}
\newcommand{\Mat}{\mathrm{Mat}}
\newcommand{\id}{\mathrm{id}}
\newcommand{\isom}{\xrightarrow{\sim}} 
\newcommand{\leftisom}{\xleftarrow{\sim}}
\newcommand{\Hom}{\mathrm{Hom}}
\newcommand{\Ker}{\mathrm{Ker}\:}
\newcommand{\rk}{\mathrm{rk}\:}
\newcommand{\diag}{\mathrm{diag}}
\newcommand{\ort}{\mathrm{ort}}
\newcommand{\pr}{\mathrm{pr}}
\newcommand{\vol}{\mathrm{vol\:}}

\renewcommand{\epsilon}{\varepsilon}
\renewcommand{\phi}{\varphi}
\newcommand{\e}{\mathbb{e}}
\renewcommand{\l}{\lambda}
\renewcommand{\C}{\mathbb{C}}
\newcommand{\R}{\mathbb{R}}
\newcommand{\E}{\mathbb{E}}

\newcommand{\vvector}[1]{\begin{pmatrix}{#1}_1 \\\vdots\\{#1}_n\end{pmatrix}}
\renewcommand{\vector}[1]{({#1}_1, \ldots, {#1}_n)}

\begin{document}
\section{Лекция 30 от 11.05.2016}

\subsection*{Самосопряжённые линейные операторы (продолжение)}

Пусть $\mathbb{E}$ --- евклидово пространство, $\dim \mathbb{E} = n$, $\varphi \in L(\mathbb{E})$. Вспомним, что по определению сопряжённый линейный оператор это $\varphi^*\colon (x,\;\varphi(y)) = (\varphi^*(x),\; y)$.

\par Вспомним также, что такое самосопряжённый оператор, это такой оператор $\varphi$, что $\varphi = \varphi^*$.

\begin{Suggestion}
	Пусть $\varphi = \varphi^*$. Если $U \subseteq \mathbb{E}$ --- подпространство --- является $\varphi$-инвариантным, то $U^\perp$ тоже $\varphi$-инвариантно.
\end{Suggestion}

\begin{proof}
	 Посмотрим на матрицу $\varphi$. Поскольку $\mathbb{E} = U \oplus U^\perp$,то легко понять, что матрица линейного оператора будет выглядеть как $\begin{pmatrix}
		A& 0\\
		0& B
	\end{pmatrix}$, где $A$ --- матрица ограничения $\varphi$ на $U$, а $B$ --- на $U^\perp$.
	
	\par Пусть $\varphi(U) \subseteq U$. Хотим, чтобы $\varphi(U^\perp) \subseteq U^\perp$. 
	\[
		\forall x \in U,\; y \in U^\perp\colon (x,\varphi(y)) = (\varphi^*(x), y) = (\underbrace{\varphi(x)}_{\in U}, y) = 0
	\]
\end{proof}

\begin{Suggestion}
	У самосопряжённого оператора $\varphi$ есть собственный вектор над $\mathbb{R}$. 
\end{Suggestion}

\begin{proof}
	Знаем: у $\varphi$ есть одномерное(случай 1) или двумерное (случай 2) инвариантное подпространство.
	
\begin{enumerate}
	\item В случае одномерного инвариантного подпространства всё уже ок, потому что его порождающий вектор уже собственный
	
	\item Пусть $U \subseteq \mathbb{E}$ --- двумерное инвариантное подпространство, а $\mathbb{e} = (e_1, e_2)$ --- ортонормированный базис. Пусть $\psi \in L(U)$ --- ограничение $\varphi$ на $U$. В прошлый раз доказывали, что матрица $\psi$ --- симметричная. $A(\psi, \mathbb{e}) = \begin{pmatrix}
		a& b\\
		b& c
	\end{pmatrix}$.
	Её характеристический многочлен \begin{gather*}
	\chi_\psi(t) = (-1)^2\begin{vmatrix}
	a-t& b\\
	b& c-t
	\end{vmatrix} = t^2 - (a + c)t + ac - b^2 = 0\\
	D = (a - c)^2 + 4b^2 \geqslant 0
	\end{gather*}
	Значит у $\chi_\psi(t)$ есть корни, а у $\psi$ есть собственный вектор, а значит и у $\varphi$.
\end{enumerate}

\end{proof}

\begin{Theorem}
	У всякого самосопряжённого линейного оператора есть ортонормированный базис из собственных векторов. В частности, $\varphi$ диагонализуем над $\mathbb{R}$ и характеристический многочлен разлагается в произведение линейных сомножителей.
\end{Theorem}

\begin{Consequence}
	Всякая симметричная подобна диагональной над $\mathbb{R}$.
\end{Consequence}

\begin{proof}
	Индукцией по $n$. Для $n = 1$ всё очевидно.
	\par Если $n > 1$, то у $\varphi$ есть собственный вектор $v$. Положим $e_1 = \cfrac{v}{|v|}$. Положим $U = \langle e_1\rangle^\perp$. Тогда $\dim U = n - 1$.
	
	\par $U$ --- $\varphi$-инвариантное подпространство. По предположеню индукции в $U$ есть ортонормированный базис из собственных векторов $(e_2,\ldots, e_n)$. Тогда $(e_1,\ldots, e_n)$ --- искомый базис.
\end{proof}

\begin{Consequence}
	Пусть $\varphi = \varphi^*$; $\lambda, \mu$ --- собственные значения. Тогда из того, что $\lambda \neq \mu$, следует, что $V_\lambda(\varphi)\perp V_\mu(\varphi)$.
\end{Consequence}

\begin{proof}\ 
	\begin{enumerate} 
		\item Координатный способ. Пусть $\mathbb{e} = (e_1, \ldots, e_n)$ --- ортонормированный базис из собственных векторов. $x = x_1e_1 + \ldots + x_ne_n\in V_\lambda(\varphi)$, причём $\varphi(e_i) = \lambda_ie_i$.
		\begin{gather*}
			\varphi(x) = x_1\lambda_1 e_1 + \ldots x_n \lambda_n e_n\\
			x \in V_\lambda(\varphi) \Leftrightarrow \varphi(x) = \lambda x \Leftrightarrow x \in \langle e_i\;|\; \lambda_i = \lambda\rangle\\
			\Rightarrow V_\lambda(\varphi) \perp V_\mu (\varphi),\text{ если } \lambda \neq \mu
		\end{gather*}
		\item Бескоординатный способ. 
		\begin{gather*}
			x \in V_\lambda(\varphi)\\
			y \in V_\mu (\varphi)\\
			\lambda(x,y) = (\lambda x, y) = (\varphi(x), y) = (x, \varphi(y)) = (x, \mu y) = \mu (x, y).
		\end{gather*}
		А поскольку $\lambda \neq \mu$, то $(x,y) = 0$.
	\end{enumerate}
\end{proof}

\begin{Consequence}[Приведение квадратичной формы к главным осям]
	Для любой квадратичной формы $Q$ над $\mathbb{E}$ существует ортонормированный базис, в котором $Q$ имеет канонический вид. $Q(x_1, \ldots, x_n) = \lambda_1 x_1^2 + \ldots + \lambda_n x_n^2$.\\
	\par Числа $\lambda_1, \ldots, \lambda_n$ определены однозначно с точностью до перестановки.
\end{Consequence}

\begin{proof}
	Существует единственный самосопряжённый линейный оператор в $\mathbb{E}$ такой, что $Q(v) = (v, \varphi(v))$. Если $\mathbb{e}$ --- ортонормированный базис, то матрица $Q$ в базисе $\mathbb{e}$ будет равна матрице $\varphi$ в базисе $\mathbb{e}$.
	\par $\lambda_1, \ldots, \lambda_n$ --- собственные значения $\varphi$. 
\end{proof}

\begin{Consequence}
	Пусть $A\in M_n(\mathbb{R}), A = A^T$. Тогда существует ортогональная матрица $C$ такая, что $C^TAC = C^{-1} AC = D = \mathrm{diag}(\lambda_1, \ldots, \lambda_n)$
\end{Consequence}

\subsection{Ортогональные линейные операторы}

\begin{Def}
	Линейный оператор $\varphi \in L(\mathbb{E})$ называется ортогональным, если $(\varphi(x), \varphi(y)) = (x,y)$. Другими словами, $\varphi$ сохраняет скалярное произведение.
\end{Def}

\end{document}
