\documentclass[a4paper,12pt]{article}

%% Работа с русским языком
\usepackage{cmap}					% поиск в PDF
\usepackage{mathtext} 				% русские буквы в формулах
\usepackage[T2A]{fontenc}			% кодировка
\usepackage[utf8]{inputenc}			% кодировка исходного текста
\usepackage[english,russian]{babel}	% локализация и переносы

%% Отступы между абзацами и в начале абзаца 
\setlength{\parindent}{0pt}
\setlength{\parskip}{\medskipamount}

%% Изменяем размер полей
\usepackage[top=0.5in, bottom=0.75in, left=0.625in, right=0.625in]{geometry}

%% Графика
\usepackage[pdftex]{graphicx}
\graphicspath{{images/}}

%% Различные пакеты для работы с математикой
\usepackage{mathtools}				% Тот же amsmath, только с некоторыми поправками

\usepackage{amssymb}				% Математические символы
\usepackage{amsthm}					% Пакет для написания теорем
\usepackage{amstext}
\usepackage{array}
\usepackage{amsfonts}
\usepackage{icomma}					% "Умная" запятая: $0,2$ --- число, $0, 2$ --- перечисление
\usepackage{bbm}				    % Для красивого (!) \mathbb с  буквами и цифрами
\usepackage{enumitem}               % Для выравнивания itemise (\begin{itemize}[align=left])

% Номера формул
\mathtoolsset{showonlyrefs=true} % Показывать номера только у тех формул, на которые есть \eqref{} в тексте.

% Ссылки
\usepackage[colorlinks=true, urlcolor=blue]{hyperref}

% Шрифты
\usepackage{euscript}	 % Шрифт Евклид
\usepackage{mathrsfs}	 % Красивый матшрифт

% Свои команды\textbf{}
\DeclareMathOperator{\sgn}{\mathop{sgn}}

% Перенос знаков в формулах (по Львовскому)
\newcommand*{\hm}[1]{#1\nobreak\discretionary{}
{\hbox{$\mathsurround=0pt #1$}}{}}

% Графики
\usepackage{tikz}
\usepackage{pgfplots}
%\pgfplotsset{compat=1.12}

% Изменим формат \section и \subsection:
\usepackage{titlesec}
\titleformat{\section}
{\vspace{1cm}\centering\LARGE\bfseries}	% Стиль заголовка
{}										% префикс
{0pt}									% Расстояние между префиксом и заголовком
{} 										% Как отображается префикс
\titleformat{\subsection}				% Аналогично для \subsection
{\Large\bfseries}
{}
{0pt}
{}

% Информация об авторах
\author{Группа лектория ФКН ПМИ 2015-2016 \\
	Анастасия Иовлева \\
	Ксюша Закирова \\
	Руслан Хайдуров}
\title{Лекции по предмету \\
	\textbf{Линейная алгебра и геометрия}}
\date{2016 год}

\newtheorem*{Def}{Определение}
\newtheorem*{Lemma}{Лемма}
\newtheorem*{Suggestion}{Предложение}
\newtheorem*{Examples}{Пример}
\newtheorem*{Comment}{Замечание}
\newtheorem*{Consequence}{Следствие}
\newtheorem*{Theorem}{Теорема}
\newtheorem*{Statement}{Утверждение}
\newtheorem*{Task}{Упражнение}
\newtheorem*{Designation}{Обозначение}
\newtheorem*{Generalization}{Обобщение}
\newtheorem*{Thedream}{Предел мечтаний}
\newtheorem*{Properties}{Свойства}

\renewcommand{\mathbb}{\mathbbm}
\renewcommand{\Re}{\mathrm{Re\:}}
\renewcommand{\Im}{\mathrm{Im\:}}
\newcommand{\Arg}{\mathrm{Arg\:}}
\renewcommand{\arg}{\mathrm{arg\:}}
\newcommand{\Mat}{\mathrm{Mat}}
\newcommand{\id}{\mathrm{id}}
\newcommand{\isom}{\xrightarrow{\sim}} 
\newcommand{\leftisom}{\xleftarrow{\sim}}
\newcommand{\Hom}{\mathrm{Hom}}
\newcommand{\Ker}{\mathrm{Ker}\:}
\newcommand{\rk}{\mathrm{rk}\:}
\newcommand{\diag}{\mathrm{diag}}
\newcommand{\ort}{\mathrm{ort}}
\newcommand{\pr}{\mathrm{pr}}
\newcommand{\vol}{\mathrm{vol\:}}

\renewcommand{\epsilon}{\varepsilon}
\renewcommand{\phi}{\varphi}
\newcommand{\e}{\mathbb{e}}
\renewcommand{\l}{\lambda}
\renewcommand{\C}{\mathbb{C}}
\newcommand{\R}{\mathbb{R}}
\newcommand{\E}{\mathbb{E}}

\newcommand{\vvector}[1]{\begin{pmatrix}{#1}_1 \\\vdots\\{#1}_n\end{pmatrix}}
\renewcommand{\vector}[1]{({#1}_1, \ldots, {#1}_n)}

\begin{document}
\renewcommand{\phi}{\varphi}
\section{Лекция 18 от 29.01.2016}

\subsection{Матрица перехода и переход к новому базису}
Пусть $V$ --- векторное пространство, $\dim V = n$, вектора $e_1, \ldots, e_n$ --- базис, а $e'_1, \ldots, e'_n$ --- некий набор из $n$ векторов. Тогда каждый вектор из этого набора линейно выражается через базис.
\begin{gather*}
e'_j = \sum_{i = 1}^{n} c_{ij}e_i, \quad c_{ij} \in F \\
(e'_1, \ldots, e'_n) = (e_1, \ldots, e_n) \cdot C, \quad C = (c_{ij})
\end{gather*}
То есть мы получили матрицу, где в $j$-ом столбце стоят коэффициенты линейного разложения вектора $e'_j$ в базисе $(e_1, \ldots, e_n)$.

Теперь пусть $e'_1, \ldots, e'_n$ --- тоже базис в $V$. Вспомним, что на прошлой лекции уже было сказано, что в этом случае $\det C \neq 0$.

\begin{Def}
Матрица $C$ называется матрицей перехода от базиса $(e_1, \ldots, e_n)$ к базису $(e'_1, \ldots, e'_n)$.
\end{Def}

\begin{Comment}
Матрица перехода от $(e'_1, \ldots, e'_n)$ к $(e_1, \ldots, e_n)$ есть $C^{-1}$.
\end{Comment}

И небольшое замечание касательно записи: когда базис записан в скобках, то есть $(e_1, \ldots, e_n)$, то нам важен порядок векторов в нем, в противном случае, при записи $e_1, \ldots, e_n$, порядок не важен.

Итого, имеем два базиса пространства $V$, $(e_1, \ldots, e_n)$ и $(e'_1, \ldots, e'_n)$, и матрицу перехода $C$ такую, что $(e'_1, \ldots, e'_n) = (e_1, \ldots, e_n) \cdot C$. Возьмем некий вектор $v$ и разложим его по обоим базисам.
\begin{gather*}
v \in V \Rightarrow 
\begin{aligned}
& v = x_1e_1 + \ldots + x_ne_n, \quad & x_i \in F \\
& v = x'_1e'_1 + \ldots + x'_ne'_n, \quad & x'_i \in F
\end{aligned}
\end{gather*}

\begin{Suggestion}
Формула преобразования координат при переходе к другому базису:
\begin{gather*}
\begin{pmatrix*}
x_1 \\
\vdots \\
x_n
\end{pmatrix*}
= C 
\begin{pmatrix*}
x'_1 \\
\vdots \\
x'_n
\end{pmatrix*}
\qquad \text{или} \qquad
x_i = \sum_{j = 1}^{n}c_{ij}x'_j
\end{gather*}
\end{Suggestion}

\begin{proof}
С одной стороны:
\begin{gather*}
v = x'_1 e'_1 + \ldots + x'_ne'_n = 
\begin{pmatrix*}
e'_1 & \ldots & e'_n
\end{pmatrix*} 
\begin{pmatrix*}
x'_1 \\
\vdots \\
x'_n
\end{pmatrix*} = 
\begin{pmatrix*}
e_1 & \ldots & e_n
\end{pmatrix*} C
\begin{pmatrix*}
x'_1 \\
\vdots \\
x'_n
\end{pmatrix*}.
\end{gather*}

Однако с другой стороны:
\begin{gather*}
v = x_1e_1 + \ldots + x_ne_n = 
\begin{pmatrix*}
e_1 & \ldots & e_n
\end{pmatrix*}
\begin{pmatrix*}
x_1 \\
\vdots \\
x_n
\end{pmatrix*}.
\end{gather*}

Сравнивая одно с другим, получаем, что:
\[
\begin{pmatrix*}
x_1 \\
\vdots \\
x_n
\end{pmatrix*} = C
\begin{pmatrix*}
x'_1 \\
\vdots \\
x'_n
\end{pmatrix*}.
\]
\end{proof}

\subsection{Линейные отображения}

Пусть $V$ и $W$ --- два векторных пространства над полем $F$.

\begin{Def}
Отображение $f : V \rightarrow W$ называется линейным, если:
\begin{enumerate}
\item $f(u_1 + u_2) = f(u_1) + f(u_2), \quad \forall u_1, u_2 \in V$;
\item $f(\alpha u) = \alpha f(u), \quad \forall u \in V,\ \forall \alpha \in F$.
\end{enumerate}
\end{Def}

\begin{Comment}
Свойства 1--2 эквивалентны тому, что 
\[
f(\alpha_1 u_1 + \alpha_2 u_2) = \alpha_1f(u_1) + \alpha_2f(u_2), \quad \forall u_1, u_2 \in V,\ \forall \alpha_1, \alpha_2 \in F.
\]
\end{Comment}
Здесь важно понимать, что сначала сложение векторов и умножение на скаляр происходит в пространстве $V$, а потом в пространстве $W$.

\vspace{0.3cm}
\textbf{Простейшие свойства.}
\begin{enumerate}
\item $f(\vec{0}_V) = \vec{0}_W$
\begin{proof}
$f(\vec{0}_V) = f(0 \cdot \vec{0}_V) = 0f(\vec{0}_V) = \vec{0}_W$
\end{proof}
\item $\phi(-u) = -\phi(u)$, где $(-u)$ --- обратный элемент к $u$.
\begin{proof}
$\phi(-u) + \phi(u) = \phi(-u+u) = \phi(\vec{0}_V) = \vec{0}_W \Rightarrow \phi(-u) = -\phi(u)$
\end{proof}
\end{enumerate}

\vspace{0.3cm}
\textbf{Примеры}
\begin{itemize}
\item[\textbf{(0)}] $V \rightarrow V: v \mapsto v$ --- тождественное отображение.
\item[\textbf{(1)}] $f: \mathbb{R} \rightarrow \mathbb{R}$ линейно $\Leftrightarrow \exists k \in \mathbb{R}: f(x) = kx, \quad \forall x \in \mathbb{R}$
\begin{proof} \
\begin{description}
\item[$\Rightarrow$] $f(x) = f(x \cdot 1) = xf(1) = kx$, где $k = f(1)$
\item[$\Leftarrow$] Проверим необходимые условия линейности.
\begin{enumerate}
\item $f(x) = kx \Rightarrow f(x_1 + x_2) = k(x_1 + x_2) = kx_1 + kx_2 = f(x_1) + f(x_2)$
\item $f(\alpha x) = k\alpha x = \alpha k x = \alpha f(x)$
\end{enumerate}
\end{description}
\end{proof}
\item[\textbf{(2)}] $f: \mathbb{R}^2 \rightarrow \mathbb{R}^2$ --- декартова система координат.
\begin{itemize}
\item[2.1] Поворот вокруг 0 на угол $\alpha$ линеен.
\item[2.2] Проекция на прямую, проходящую через 0, линейна.
\end{itemize}
\item[\textbf{(3)}] $P_n = R[x]_{\leqslant n}$ --- пространство всех многочленов от $x$ степени не больше $n$.
\begin{gather*}
\Delta : f \mapsto f' \text{ (производная)} \\
\left.
\begin{aligned}
(f+g)' &= f' + g' \\
(\alpha f)' &= \alpha f'
\end{aligned}
\right| \Rightarrow \Delta \text{ --- линейное отображение из $P_n$ в $P_{n-1}$}
\end{gather*}
\item[\textbf{(4)}] Векторное пространство $V$, $\dim V = n$, $e_1, \ldots, e_n$ --- базис.
\begin{gather*}
V \mapsto \mathbb{R}^n \\
x_1e_1 + \ldots + x_ne_n \mapsto 
\begin{pmatrix*}
x_1 \\
\vdots \\
x_n
\end{pmatrix*} \text{ --- тоже линейное отображение.}
\end{gather*}
\item[\textbf{(5)}] $A \in \Mat_{m\times n}$, $k \geqslant 1$ --- любое, $\phi: \Mat_{n\times k} \rightarrow \Mat_{m\times k}$.
\begin{gather*}
\phi(X) = A \cdot X \\
A(X_1 + X_2) = AX_1 + AX_2 \\
A(\alpha X) = \alpha(AX) 
\end{gather*}
Частный случай, при $k = 1$ --- $\phi: F^n \rightarrow F^m$.
\end{itemize}

\subsection{Изоморфизм}
\begin{Def}
Отображение $\phi: V \rightarrow W$ называется изоморфизмом, если $\phi$ линейно и биективно. Обозначение: $\phi : V \isom  W$.
\end{Def}
\vspace{0.3cm}
Рассмотрим те же примеры:
\begin{itemize}
\item[\textbf{(0)}] Изоморфизм.
\item[\textbf{(1)}] Изоморфизм, при $k \neq 0$.
\item[\textbf{(2)}]
\begin{itemize}
\item[2.1] Изоморфизм.
\item[2.2] Не изоморфизм.
\end{itemize}
\item[\textbf{(3)}] Не изоморфизм.
\item[\textbf{(4)}] Изоморфизм.
\item[\textbf{(5)}] \underline{Задача:} доказать, что $\phi$ --- изоморфизм тогда и только тогда, когда $n = m$ и $\det A \neq 0$.
\end{itemize}

\begin{Suggestion}
Пусть $\phi: V \rightarrow W$ --- изоморфизм. Тогда $\phi^{-1}: W \rightarrow V$ --- тоже изоморфизм.
\end{Suggestion}
\begin{proof}
Так как $\phi$ --- биекция, то $\phi^{-1}$ --- тоже биекция.
\begin{gather*}
w_1, w_2 \in W \Rightarrow \exists v_1, v_2 \in V:
\begin{aligned} 
&\phi(v_1) = w_1 & v_1 = \phi^{-1}(w_1)\\ 
&\phi(v_2) = w_2 & v_2 = \phi^{-1}(w_2)
\end{aligned} 
\end{gather*}
Тогда осталось только доказать линейность обратного отображения. Для этого проверим выполнение необходимых условий линейности.
\begin{enumerate}
\item $ \phi^{-1}(w_1 + w_2) = \phi^{-1}\left(\phi(v_1) + \phi(v_2)\right) = \phi^{-1}\left(\phi(v_1 + v_2)\right) = \id (v_1 + v_2) = v_1 + v_2$
\item $\alpha \in F, \quad \phi^{-1}(\alpha w_1) = \phi^{-1}(\alpha\phi(v_1)) = \phi^{-1}(\phi(\alpha v_1)) = \id (\alpha v_1) = \alpha v_1$.
\end{enumerate}
\end{proof}

\begin{Def}
Два векторных пространства $V$ и $W$ называются изоморфными, если существует изоморфизм $\phi: V \isom W$ (и тогда существует изоморфизм $V \leftisom W$ по предположению). Обозначение: $V \simeq W$ или $V \cong W$.
\end{Def}

Отображения можно соединять в композиции:
\begin{gather*}
\left.
\begin{aligned}
\phi&: U \rightarrow V \\
\psi&: V \rightarrow W
\end{aligned}
\right|\Rightarrow \psi \circ \phi : U \rightarrow W \quad \psi \circ \phi(u) = \psi(\phi(u))
\end{gather*}

\begin{Suggestion}\ 
\begin{enumerate}
\item Если $\phi$ и $\psi$ линейны, то $\psi \circ \phi$ тоже линейно.
\item Если $\phi$ и $\psi$ изоморфизмы, то $\psi \circ \phi$ тоже изоморфизм.
\end{enumerate}
\end{Suggestion}

\begin{proof} \ 
\begin{enumerate}
\item Опять-таки, просто проверим необходимые условия линейности.
\begin{enumerate}
\item $(\psi \circ \phi)(u_1 + u_2) = \psi(\phi(u_1 + u_2)) = \psi(\phi(u_1) + \phi(u_2)) = \psi(\phi(u_1)) + \psi(\phi(u_2)) = \\ = (\psi \circ\phi)(u_1) + (\psi \circ \phi)(u_2)$
\item $(\psi \circ \phi)(\alpha u) = \psi(\phi(\alpha u)) = \psi(\alpha\phi(u)) = \alpha \psi(\phi(u)) = \alpha (\psi \circ \phi) (u)$
\end{enumerate}
\item Следует из сохранения линейности и того, что композиция биекций тоже биекция.
\end{enumerate}
\end{proof}

\begin{Consequence}
Изоморфизм это отношение эквивалентности на множестве всех векторных пространств над фиксированным полем $F$.
\end{Consequence}
\begin{proof} \ 
\begin{description}
\item[Рефлексивность] $V \simeq V$.
\item[Симметричность] $V \simeq W \Rightarrow W \simeq V$.
\item[Транзитивность] $(V \simeq U)\ \land\ (U \simeq W) \Rightarrow V \simeq W$.
\end{description}
\end{proof}

То есть множество всех векторных пространств над фиксированным полем $F$ разбивается на попарно непересекающиеся классы, причем внутри одного класса любые два пространства изоморфны. Такие классы называются \textit{классами эквивалентности}.

\begin{Theorem}
Если два конечномерных векторных пространства $V$ и $W$ над полем $F$ изоморфны, то $\dim V = \dim W$.
\end{Theorem}

Но для начала докажем следующую лемму.

\begin{Lemma}[1]
Для векторного пространства $V$ над полем $F$ размерности $n$ верно, что $V \simeq F^n$.
\end{Lemma}
\begin{proof}
Рассмотрим отображение $\phi: V \rightarrow F^n$ из примера 4. Пусть $(e_1, \ldots, e_n)$ --- базис пространства $V$. Тогда:
\[
x_1e_1 + \ldots + x_ne_n \mapsto 
\begin{pmatrix*}
x_1 \\
\vdots \\
x_n
\end{pmatrix*}, \quad x_i \in F.
\]
Отображение $\phi$ линейно и биективно, следовательно $\phi$ --- изоморфизм. А раз существует изоморфное отображение между пространствами $V$ и $F^n$, то они изоморфны.
\end{proof}

\end{document}
