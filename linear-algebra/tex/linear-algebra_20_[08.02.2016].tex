\documentclass[a4paper,12pt]{article}

%% Работа с русским языком
\usepackage{cmap}					% поиск в PDF
\usepackage{mathtext} 				% русские буквы в формулах
\usepackage[T2A]{fontenc}			% кодировка
\usepackage[utf8]{inputenc}			% кодировка исходного текста
\usepackage[english,russian]{babel}	% локализация и переносы

%% Отступы между абзацами и в начале абзаца 
\setlength{\parindent}{0pt}
\setlength{\parskip}{\medskipamount}

%% Изменяем размер полей
\usepackage[top=0.5in, bottom=0.75in, left=0.625in, right=0.625in]{geometry}

%% Графика
\usepackage[pdftex]{graphicx}
\graphicspath{{images/}}

%% Различные пакеты для работы с математикой
\usepackage{mathtools}				% Тот же amsmath, только с некоторыми поправками

\usepackage{amssymb}				% Математические символы
\usepackage{amsthm}					% Пакет для написания теорем
\usepackage{amstext}
\usepackage{array}
\usepackage{amsfonts}
\usepackage{icomma}					% "Умная" запятая: $0,2$ --- число, $0, 2$ --- перечисление
\usepackage{bbm}				    % Для красивого (!) \mathbb с  буквами и цифрами
\usepackage{enumitem}               % Для выравнивания itemise (\begin{itemize}[align=left])

% Номера формул
\mathtoolsset{showonlyrefs=true} % Показывать номера только у тех формул, на которые есть \eqref{} в тексте.

% Ссылки
\usepackage[colorlinks=true, urlcolor=blue]{hyperref}

% Шрифты
\usepackage{euscript}	 % Шрифт Евклид
\usepackage{mathrsfs}	 % Красивый матшрифт

% Свои команды\textbf{}
\DeclareMathOperator{\sgn}{\mathop{sgn}}

% Перенос знаков в формулах (по Львовскому)
\newcommand*{\hm}[1]{#1\nobreak\discretionary{}
{\hbox{$\mathsurround=0pt #1$}}{}}

% Графики
\usepackage{tikz}
\usepackage{pgfplots}
%\pgfplotsset{compat=1.12}

% Изменим формат \section и \subsection:
\usepackage{titlesec}
\titleformat{\section}
{\vspace{1cm}\centering\LARGE\bfseries}	% Стиль заголовка
{}										% префикс
{0pt}									% Расстояние между префиксом и заголовком
{} 										% Как отображается префикс
\titleformat{\subsection}				% Аналогично для \subsection
{\Large\bfseries}
{}
{0pt}
{}

% Информация об авторах
\author{Группа лектория ФКН ПМИ 2015-2016 \\
	Анастасия Иовлева \\
	Ксюша Закирова \\
	Руслан Хайдуров}
\title{Лекции по предмету \\
	\textbf{Линейная алгебра и геометрия}}
\date{2016 год}

\newtheorem*{Def}{Определение}
\newtheorem*{Lemma}{Лемма}
\newtheorem*{Suggestion}{Предложение}
\newtheorem*{Examples}{Пример}
\newtheorem*{Comment}{Замечание}
\newtheorem*{Consequence}{Следствие}
\newtheorem*{Theorem}{Теорема}
\newtheorem*{Statement}{Утверждение}
\newtheorem*{Task}{Упражнение}
\newtheorem*{Designation}{Обозначение}
\newtheorem*{Generalization}{Обобщение}
\newtheorem*{Thedream}{Предел мечтаний}
\newtheorem*{Properties}{Свойства}

\renewcommand{\mathbb}{\mathbbm}
\renewcommand{\Re}{\mathrm{Re\:}}
\renewcommand{\Im}{\mathrm{Im\:}}
\newcommand{\Arg}{\mathrm{Arg\:}}
\renewcommand{\arg}{\mathrm{arg\:}}
\newcommand{\Mat}{\mathrm{Mat}}
\newcommand{\id}{\mathrm{id}}
\newcommand{\isom}{\xrightarrow{\sim}} 
\newcommand{\leftisom}{\xleftarrow{\sim}}
\newcommand{\Hom}{\mathrm{Hom}}
\newcommand{\Ker}{\mathrm{Ker}\:}
\newcommand{\rk}{\mathrm{rk}\:}
\newcommand{\diag}{\mathrm{diag}}
\newcommand{\ort}{\mathrm{ort}}
\newcommand{\pr}{\mathrm{pr}}
\newcommand{\vol}{\mathrm{vol\:}}

\renewcommand{\epsilon}{\varepsilon}
\renewcommand{\phi}{\varphi}
\newcommand{\e}{\mathbb{e}}
\renewcommand{\l}{\lambda}
\renewcommand{\C}{\mathbb{C}}
\newcommand{\R}{\mathbb{R}}
\newcommand{\E}{\mathbb{E}}

\newcommand{\vvector}[1]{\begin{pmatrix}{#1}_1 \\\vdots\\{#1}_n\end{pmatrix}}
\renewcommand{\vector}[1]{({#1}_1, \ldots, {#1}_n)}

\begin{document}
\renewcommand{\phi}{\varphi}

\section{Лекция 20 от 08.02.2016}

\subsection{Линейные отображения (продолжение)}
Пусть $\phi\colon V \rightarrow W$ --- линейное отображение.

\begin{Suggestion}\
	\begin{enumerate}
		\item $\Ker \phi$ --- подпространство в $V$.
		\item $\Im \phi$ --- подпространство в $W$.
		
	\end{enumerate}
\end{Suggestion}

\begin{proof}
    Проверим по определению.
    \begin{enumerate}
        \item \begin{itemize}
            \item $\phi(0_v) = 0_w$ --- этот факт мы уже доказали.
            \item $v_1, v_2 \in \Ker \phi \Rightarrow \phi(v_1+v_2) = \phi(v_1) + \phi(v_2) = 0_w + 0_w = 0_w \Rightarrow v_1+v_2 \in \Ker \phi$.
            \item $v \in \Ker \phi, \alpha \in F \Rightarrow \phi(\alpha v) = \alpha \phi(v) = \alpha 0 = 0$, то есть $\alpha v$ тоже лежит в ядре.
        \end{itemize}
        \item \begin{itemize}
            \item $0_w = \phi(0_v) \Rightarrow 0_w \in \Im(\phi)$.
            \item $w_1, w_2 \in \Im \phi \Rightarrow \exists v_1, v_2 \in V\colon w_1 = \phi(v_1), w_2 = \phi(v_2) \Rightarrow w_1 + w_2 = \phi(v_1) + \phi(v_2) = \phi(v_1 + v_2) \Rightarrow w_1 + w_2 \in \Im \phi$.
            \item $w \in \Im \phi, \alpha \in F \Rightarrow \exists v \in V \colon \phi(v) = w \Rightarrow \alpha w = \alpha \phi(v) = \phi(\alpha v) \Rightarrow \alpha w \in \Im \phi$.
            
        \end{itemize}
    \end{enumerate}
    То есть все условия подпространства по определению выполнены и предложение доказано.
\end{proof}

\begin{Suggestion}\
    \begin{enumerate}
        \item Отображение $\phi$ инъективно тогда и только тогда, когда $\Ker \phi = \{0\}$.
        \item Отображение $\phi$ сюръективно тогда и только тогда, когда $\Im \phi = W$.
    \end{enumerate} 
\end{Suggestion}

\begin{proof}\
    \begin{enumerate}
        \item \begin{itemize}
            \item $[\Rightarrow]$ Очевидно.
            \item $[\Leftarrow]$ $v_1, v_2 \in V :\ \phi(v_1)=\phi(v_2) \Rightarrow \phi (v_1 - v_2) = 0 \Rightarrow v_1 - v_2 = 0 \Rightarrow v_1 = v_2 $.
        \end{itemize}
        \item Очевидно из определения образа.
    \end{enumerate}
\end{proof}

\begin{Consequence}
    Отображение $\phi$ является изоморфизмом тогда и только тогда, когда $\Ker \phi~=~\{0\}$ и $\Im \phi = W$.
\end{Consequence}

\begin{Suggestion}
    Пусть $U \subset V$ --- подпространство и $e_1, \ldots, e_k$ --- его базис. Тогда:
    \begin{enumerate}
        \item $\phi(U)$ --- подпространство, $\phi(U) = \langle \phi(e_1), \ldots, \phi(e_k)\rangle$;
        \item $\dim \phi(U) \leqslant \dim U$.
    \end{enumerate}
\end{Suggestion}

\begin{proof}\
    \begin{enumerate}
        \item $\phi(x_1e_1 + x_2e_2 + \ldots + x_ke_k) = x_1\phi(e_1) + \ldots + x_k\phi(e_k) \in \langle \phi(e_1), \ldots, \phi(e_k)\rangle$.
        \item $\phi(U) = \langle\phi(e_1), \ldots, \phi(e_k)\rangle \Rightarrow \dim \phi(U) \leqslant \dim U$ по основной лемме о линейной зависимости.
    \end{enumerate}
\end{proof}

Пусть $V, W$ --- векторные пространства, $\mathbb{e} = (e_1, \ldots, e_n)$ --- базис $V$, $\mathbb{f} = (f_1, \ldots, f_m)$ --- базис $W$, $A$ --- матрица $\phi$ по отношению к $\mathbb{e},\ \mathbb{f}$.

\begin{Suggestion}
    $\dim \Im \phi = \rk A$.
\end{Suggestion}

\begin{proof}
    \begin{gather*}
        v \in V,\ v = x_1e_1 + \ldots x_ne_n\\
        \phi(v) = y_1f_1 + \ldots y_mf_m 
    \end{gather*}
    Тогда:
    $$
    \begin{pmatrix}
        y_1\\
        \vdots\\
        y_m
    \end{pmatrix}
    = A \begin{pmatrix}
        x_1\\
        \vdots\\
        x_n
    \end{pmatrix}.
    $$
   \par  $A^{\left(j\right)}$ --- столбец координат в базисе $\mathbb{f}$, $\alpha_1, \ldots, \alpha_n \in F$.
   \[
        \alpha_1 \phi(e_1) + \ldots + \alpha_n \phi(e_n) = 0 \Leftrightarrow \alpha_1 A^{\left(1\right)} + \ldots + \alpha_n A^{\left(n\right)} = 0
   \]
   Отсюда следует, что:
   \[
        \rk A = \rk\{\phi(e_1), \ldots, \phi(e_n)\} = \dim \underbrace{\langle\phi(e_1), \ldots, \phi(e_n)\rangle}_{\Im\phi} = \dim \Im \phi.
   \]
\end{proof}

\begin{Consequence}
    Величина $\rk A$ не зависит от выбора базисов $\mathbb{e}$ и $\mathbb{f}$.
\end{Consequence}

\begin{Def}
    Величина $\rk A$ называется рангом линейного отображения $\phi$. Обозначение:~$\rk \phi$.
\end{Def}

\begin{Consequence}
    Если $\dim V = \dim W = n$, то $\phi$--- изоморфизм тогда и только тогда, когда $\det A \neq 0$. Тогда $A$ --- квадратная.
\end{Consequence}

\begin{proof}
\ 
\par $[\Rightarrow]$ $\phi$ --- изоморфизм, следовательно:
    \[
        \Im \phi = W \Rightarrow \dim \Im \phi = n \Rightarrow \rk A = n
        \Rightarrow \det A \neq 0.
    \]
\par $[\Leftarrow]$ $\det A \neq 0 \Rightarrow \exists A^{-1}$.
\[ \begin{pmatrix} 
    y_1\\
    \vdots\\
    y_m
\end{pmatrix} = A 
\begin{pmatrix}
x_1\\
\vdots\\
x_n
\end{pmatrix}
\Rightarrow 
\begin{pmatrix}
x_1\\
\vdots\\
x_n
\end{pmatrix} = A^{-1} \begin{pmatrix} 
    y_1\\
    \vdots\\
    y_m
\end{pmatrix}
\]
Таким образом, линейное отображение $\phi$ является биекцией, а значит, и изоморфизмом.
\end{proof}

\begin{Consequence}
    Пусть $A \in \Mat_{k\times m},\ B\in \Mat_{m\times n}$. Тогда $\rk AB \leqslant \min\{\rk A, \rk B\}$.
\end{Consequence}

\begin{proof}
    Реализуем $A$ и $B$ как матрицы линейных отображений, то есть $\phi_A\colon F^m \rightarrow~F^k,\\ \phi_B\colon F^n \rightarrow F^m$. Тогда $AB$ будет матрицей отображения $\phi_A \circ \phi_B$.
    \[
        \rk(AB) = \rk (\phi_A\circ\phi_B)
        \begin{cases}
            \leqslant \dim \Im \phi_A = \rk A \\
            \leqslant \dim \Im \phi_B = \rk B 

        \end{cases}
    \]
Первое неравенство следует из того, что $\Im(\phi_A\circ \phi_B) \subset \Im \phi_A$, откуда в свою очередь следует, что $\dim \Im (\phi_A\circ \phi_B) \leqslant \dim \Im \phi_A$. Рассматривая второе неравенство, получаем: 
    \[
        \Im (\phi_A\circ \phi_B) = \phi_A(\Im \phi_B) \Rightarrow \dim \Im (\phi_A\circ \phi_B)= \dim(\phi_A(\Im\phi_B)) \leqslant \dim\Im \phi_B.
    \]
\end{proof}

\begin{Task}\
    \begin{itemize}
        \item Если $A$ квадратна и $\det A \neq 0$, то $\rk AB = \rk B$.
        \item Если $B \in M_n$ и $\det B \neq 0$, то $\rk AB = \rk A$.
    \end{itemize}
\end{Task}

\begin{Theorem}
    $\dim \Im V = \dim \phi - \dim \Ker \phi$.
\end{Theorem}

Существует 2 способа доказательства. Рассмотрим оба.

\begin{proof}[Бескоординатный способ]
     Пусть $\dim \Ker \phi = k$ и $e_1, \ldots, e_k$ --- базис в $\Ker \phi$. Дополним его до базиса $V$ векторами $e_k, \ldots, e_n$. Тогда:
    \[
        \Im \phi = \langle\phi(e_1), \ldots, \phi(e_k), \ldots, \phi(e_n)\rangle = \langle0, 0, \ldots, 0, \phi(e_{k+1}), \ldots, \phi(e_n)\rangle = \langle\phi(e_{k+1}), \ldots, \phi(e_{n})\rangle
    \]
    
Пусть $\alpha_{k+1}\phi(e_{k+1}) + \ldots + \alpha_n\phi(e_n) = 0$ для некоторых $\alpha_1, \ldots, \alpha_n \in F$. Тогда:
\begin{gather*}
    \phi(\alpha_{k+1}e_{k+1} + \ldots + \alpha_ne_n) = 0 \\
    \alpha_{k+1}e_{k+1} + \ldots + \alpha_ne_n \in \Ker \phi \\
    \alpha_{k+1}e_{k+1} + \ldots + \alpha_ne_n = \beta_1e_1 + \ldots \beta_ke_k,
\end{gather*}
для некоторых $\beta_1, \ldots, \beta_k \in F$.

Но так как $e_1, \ldots, e_n$ --- базис в $V$, то $\alpha_{k+1} =  \ldots = \alpha_n = \beta_1 = \ldots = \beta_k = 0$. То есть векторы $\phi(e_1), \ldots, \phi(e_n)$ линейно независимы, а значит, образуют базис $\Im \phi$. Что и означает, что $\dim\Im\phi = n - k = \dim V - \dim\Ker\phi$.
\end{proof}

\begin{proof}[Координатный способ]
    Зафиксируем базис $\mathbb{e} = (e_1, \ldots, e_n)$ в $V$ и базис $\mathbb{f} = (f_1, \ldots, f_m)$ в $W$. Пусть $A$ --- матрица $\phi$ в базисе $\mathbb{f}$. Тогда $v = x_1e_1 + \ldots + x_ne_n$, $\phi(v) = y_1f_1 + \ldots + y_mf_m$. Получим, что $\begin{pmatrix} y_1\\ \vdots \\ y_m \end{pmatrix} = A \begin{pmatrix} x_1\\ \vdots \\ x_n\end{pmatrix}$. 
    
    $\Ker \phi$ состоит из векторов, координаты которых удовлетворяют СЛУ $ A \begin{pmatrix} x_1\\ \vdots \\ x_n\end{pmatrix} = 0$. Ранее в курсе мы уже доказали, что размерность пространства решений равна $n - \rk A$, то есть $\dim \Im \phi = n - \rk A = \dim V - \dim \Ker \phi$.
\end{proof}

\subsection{Линейные операторы}

Пусть $V$ --- конечномерное векторное пространство.

\begin{Def}
    Линейным оператором (или линейным преобразованием) называется всякое линейное отображение $\phi \colon V \rightarrow V$, то есть из $V$ в себя. Обозначение: $L(V) = \Hom(V, V)$.
\end{Def}

\par Пусть $\mathbb{e} = (e_1, \ldots, e_n)$ --- базис в $V$ и $\phi \in L(V)$. Тогда:
$$
\left(\phi(e_1), \ldots, \phi(e_n)\right) = \left(e_1, \ldots, e_n\right)A,
$$
где $A$ --- матрица линейного оператора в базисе $\mathbb{e}$. В столбце $A^{\left( j\right)}$ стоят координаты $\phi(e_j)$ в базисе $\mathbb{e}$. Матрица $A$ --- квадратная. 
\begin{Examples}\
    \begin{enumerate}
        \item $\forall v \in V : \phi(v) = 0$ --- нулевая матрица.
        \item Тождественный оператор: $\forall v \in V : \id(v) = v$ --- единичная матрица.
        \item Скалярный оператор $\lambda \id(v) = \lambda V$ --- матрица $\lambda E$ в любом базисе.
    \end{enumerate}
\end{Examples}

\begin{Consequence}[Следствия из общих фактов о линейных отображениях]\
    \begin{enumerate}
        \item Всякий линейный оператор однозначно определяется своей матрицей в любом фиксированном базисе.
        \item Для всякой квадратной матрицы существует, причем единственный, линейный оператор $\phi$ такой, что матрица $\phi$ есть $A$.
        \item Пусть $\phi \in L(V)$, $A$ --- матрица $\phi$ в базисе $\mathbb{e}$. Тогда:
        \begin{gather*}
            v = x_1e_1 + \ldots + x_ne_n\\ \phi(v) = y_1e_1 + \ldots + y_n e_n \\
            \begin{pmatrix}
                y_1\\
                \vdots \\
                y_n
            \end{pmatrix} = A \begin{pmatrix}
                x_1\\
                \vdots \\
                x_n
            \end{pmatrix}
        \end{gather*}
    \end{enumerate}
\end{Consequence}
Пусть $\phi \in L(V)$, $A$ --- матрица $\phi$ в базисе $\mathbb{e} = (e_1, \ldots, e_n)$. Пусть $\mathbb{e}' = (e_1', \ldots, e_n')$ --- другой базис, причём $(e_1', \ldots, e_n') = (e_1, \ldots, e_n)C$, где $C$ --- матрица перехода, и $A'$ --- матрица $\phi$ в базисе $\mathbb{e}'$.
\begin{Suggestion}
    $A' = C^{-1}AC$.
\end{Suggestion}
\begin{proof}
    \begin{gather}
        (e_1', \ldots, e_n') = (e_1, \ldots, e_n)C  \\
        e_j' = \sum\limits_{i=1}^{n} c_{ij}e_j \\
        \phi(e_j') = \phi\left(\sum\limits_{i=1}^{n} c_{ij}e_j\right) = \sum\limits_{i=1}^nc_{ij}\phi(e_j)\\
        (\phi(e_1'), \ldots, \phi(e_n')) = (\phi(e_1),\ldots,\phi(e_n))C = (e_1, \ldots, e_n)AC= (e_1', \ldots, e_n')\underbrace{C^{-1}AC}_{A'}
    \end{gather}
\end{proof}
\end{document}
