\documentclass[a4paper,12pt]{article}

%% Работа с русским языком
\usepackage{cmap}					% поиск в PDF
\usepackage{mathtext} 				% русские буквы в формулах
\usepackage[T2A]{fontenc}			% кодировка
\usepackage[utf8]{inputenc}			% кодировка исходного текста
\usepackage[english,russian]{babel}	% локализация и переносы

%% Отступы между абзацами и в начале абзаца 
\setlength{\parindent}{0pt}
\setlength{\parskip}{\medskipamount}

%% Изменяем размер полей
\usepackage[top=0.5in, bottom=0.75in, left=0.625in, right=0.625in]{geometry}

%% Графика
\usepackage[pdftex]{graphicx}
\graphicspath{{images/}}

%% Различные пакеты для работы с математикой
\usepackage{mathtools}				% Тот же amsmath, только с некоторыми поправками

\usepackage{amssymb}				% Математические символы
\usepackage{amsthm}					% Пакет для написания теорем
\usepackage{amstext}
\usepackage{array}
\usepackage{amsfonts}
\usepackage{icomma}					% "Умная" запятая: $0,2$ --- число, $0, 2$ --- перечисление
\usepackage{bbm}				    % Для красивого (!) \mathbb с  буквами и цифрами
\usepackage{enumitem}               % Для выравнивания itemise (\begin{itemize}[align=left])

% Номера формул
\mathtoolsset{showonlyrefs=true} % Показывать номера только у тех формул, на которые есть \eqref{} в тексте.

% Ссылки
\usepackage[colorlinks=true, urlcolor=blue]{hyperref}

% Шрифты
\usepackage{euscript}	 % Шрифт Евклид
\usepackage{mathrsfs}	 % Красивый матшрифт

% Свои команды\textbf{}
\DeclareMathOperator{\sgn}{\mathop{sgn}}

% Перенос знаков в формулах (по Львовскому)
\newcommand*{\hm}[1]{#1\nobreak\discretionary{}
{\hbox{$\mathsurround=0pt #1$}}{}}

% Графики
\usepackage{tikz}
\usepackage{pgfplots}
%\pgfplotsset{compat=1.12}

% Изменим формат \section и \subsection:
\usepackage{titlesec}
\titleformat{\section}
{\vspace{1cm}\centering\LARGE\bfseries}	% Стиль заголовка
{}										% префикс
{0pt}									% Расстояние между префиксом и заголовком
{} 										% Как отображается префикс
\titleformat{\subsection}				% Аналогично для \subsection
{\Large\bfseries}
{}
{0pt}
{}

% Информация об авторах
\author{Группа лектория ФКН ПМИ 2015-2016 \\
	Анастасия Иовлева \\
	Ксюша Закирова \\
	Руслан Хайдуров}
\title{Лекции по предмету \\
	\textbf{Линейная алгебра и геометрия}}
\date{2016 год}

\newtheorem*{Def}{Определение}
\newtheorem*{Lemma}{Лемма}
\newtheorem*{Suggestion}{Предложение}
\newtheorem*{Examples}{Пример}
\newtheorem*{Comment}{Замечание}
\newtheorem*{Consequence}{Следствие}
\newtheorem*{Theorem}{Теорема}
\newtheorem*{Statement}{Утверждение}
\newtheorem*{Task}{Упражнение}
\newtheorem*{Designation}{Обозначение}
\newtheorem*{Generalization}{Обобщение}
\newtheorem*{Thedream}{Предел мечтаний}
\newtheorem*{Properties}{Свойства}

\renewcommand{\mathbb}{\mathbbm}
\renewcommand{\Re}{\mathrm{Re\:}}
\renewcommand{\Im}{\mathrm{Im\:}}
\newcommand{\Arg}{\mathrm{Arg\:}}
\renewcommand{\arg}{\mathrm{arg\:}}
\newcommand{\Mat}{\mathrm{Mat}}
\newcommand{\id}{\mathrm{id}}
\newcommand{\isom}{\xrightarrow{\sim}} 
\newcommand{\leftisom}{\xleftarrow{\sim}}
\newcommand{\Hom}{\mathrm{Hom}}
\newcommand{\Ker}{\mathrm{Ker}\:}
\newcommand{\rk}{\mathrm{rk}\:}
\newcommand{\diag}{\mathrm{diag}}
\newcommand{\ort}{\mathrm{ort}}
\newcommand{\pr}{\mathrm{pr}}
\newcommand{\vol}{\mathrm{vol\:}}

\renewcommand{\epsilon}{\varepsilon}
\renewcommand{\phi}{\varphi}
\newcommand{\e}{\mathbb{e}}
\renewcommand{\l}{\lambda}
\renewcommand{\C}{\mathbb{C}}
\newcommand{\R}{\mathbb{R}}
\newcommand{\E}{\mathbb{E}}

\newcommand{\vvector}[1]{\begin{pmatrix}{#1}_1 \\\vdots\\{#1}_n\end{pmatrix}}
\renewcommand{\vector}[1]{({#1}_1, \ldots, {#1}_n)}

\begin{document}
\renewcommand{\phi}{\varphi}

\section{Лекция 19 от 01.02.2016}

\subsection{Изоморфизм (продолжение)}
На прошлой лекции мы ввели теорему и доказали одну лемму. Напомним их.

\begin{Theorem}
Если два конечномерных векторных пространства $V$ и $W$ изоморфны, то $\dim V \hm= \dim W$.
\end{Theorem}

\begin{Lemma}[1]
Если $\dim V = n$, то $V \simeq F^n$.
\end{Lemma}

\begin{Comment}
Говорят, что функция $\phi$ \textit{отождествляет} пространство $V$ с пространством $F^n$, если $\phi: V \isom F^n$.
\end{Comment}

Но перед тем, как доказывать эту теорему, докажем лучше еще одну лемму.
\begin{Lemma}[2]
Пусть $\phi: V \isom W$ --- изоморфизм векторных пространств, а $e_1, \ldots, e_n$--- базис $V$. Тогда $\phi(e_1), \ldots, \phi(e_n)$ --- базис $W$.
\end{Lemma}

\begin{proof}
Пусть $w \in W$ --- произвольный вектор. Положим $v \in V$ таковым, что $v \hm= \phi^{-1}(w)$.
\begin{gather*}
\begin{aligned}
v &= x_1e_1 + \ldots + x_ne_n, \quad x_i \in F \\
w &= \phi(v) = \phi(x_1e_1 + \ldots + x_ne_n) = x_1\phi(e_1) + \ldots + x_n\phi(e_n)
\end{aligned} \Rightarrow W = \langle \phi(e_1), \ldots, \phi(e_n)\rangle
\end{gather*}

Покажем, что $\phi(e_1), \ldots, \phi(e_n)$ --- линейно независимые вектора.

Пусть $\alpha_1, \ldots, \alpha_n \in F$ таковы, что $\alpha_1\phi(e_1) + \ldots + \alpha_n\phi(e_n) = 0$. Это то же самое, что $\phi(\alpha_1 e_1 \hm+ \ldots + \alpha_n e_n) = 0$. Применяя $\phi^{-1}$, получаем $\alpha_1 e_1 + \ldots \alpha_n e_n = \phi^{-1}(0) = 0$. Но так как $e_1, \ldots, e_n$~--- базис в $V$, то $\alpha_1 = \ldots= \alpha_n = 0$, и потому вектора $\phi(e_1), \ldots, \phi(e_n)$ линейно независимы. Следовательно, этот набор векторов --- базис в $W$.
\end{proof}

Теперь приступим наконец к доказательству теоремы.
\begin{proof} \ 
\begin{description}
\item[$\Rightarrow$] $V \simeq W \Rightarrow \exists \phi: V \isom W$. Тогда по лемме 2, если $e_1, \ldots, e_n$ --- базис $V$, то $\phi(e_1), \ldots, \phi(e_n)$ --- базис $W$, и тогда $\dim V = \dim W$.  
\item[$\Leftarrow$] Пусть $\dim V = \dim W = n$. Тогда по лемме 1 существуют изоморфизмы $\phi: V \isom F^n$ и $\psi : W \isom F^n$. Следовательно, $\psi^{-1}\circ\phi: V \rightarrow W$ --- изоморфизм.
\end{description}
\end{proof}

То есть получается, что с точностью до изоморфизма существует только одно векторное пространство размерности $n$. Однако не стоит заканчивать на этом курс линейной алгебры. Теперь главная наша проблема --- это как из бесконечного множества базисов в каждом векторном пространстве выбрать тот, который будет наиболее простым и удобным для каждой конкретной задачи. 

Например, рассмотрим вектор $v \in F^n$ с координатами $v = \begin{pmatrix*}x_1 \\ \vdots \\ x_n\end{pmatrix*}$. Пусть $v \neq 0$. Тогда существует такой базис $e_1, \ldots, e_n$, что $v = e_1$, то есть в этом базисе вектор имеет координаты $v = \begin{pmatrix*}1 \\ 0\\ \vdots\\0\end{pmatrix*}$.

Пусть $V$, $W$ --- векторные пространства над $F$, и $e_1, \ldots, e_n$ --- базис $V$.
\begin{Suggestion} \ 
\begin{enumerate}
\item Всякое линейное отображение $\phi : V \rightarrow W$ однозначно определяется векторами $\phi(e_1), \ldots, \phi(e_n)$.
\item Для всякого набора векторов $f_1, \ldots, f_n \in W$ существует единственное линейное отображение $\phi: V \rightarrow W$ такое, что $\phi(e_1) = f_1, \ldots, \phi(e_n) = f_n$.
\end{enumerate}
\end{Suggestion}
\begin{proof} \ 
\begin{enumerate}
\item Пусть $v \in V$, $v = x_1 e_1 + \ldots + x_ne_n$, где $x_i \in F$. Тогда $\phi(v) = x_1\phi(e_1) + \ldots + x_n \phi(e_n)$, то есть если мы знаем  вектора $\phi(e_i)$, то сможем задать $\phi(v)$ для любого $v \in V$.
\item Определим отображение $\phi: V \rightarrow W$ по формуле $\phi(x_1 e_1 + \ldots + x_n e_n) = x_1f_1 + \ldots + x_nf_n$. Прямая проверка показывает, что $\phi$ линейна, а единственность следует из пункта 1.
\end{enumerate}
\end{proof}

\begin{Consequence}
Если $\dim V = \dim W = n$, то для всякого базиса $e_1, \ldots, e_n$ пространства $V$ и всякого базиса $f_1, \ldots , f_n$ пространства $W$ существует единственный изоморфизм $\phi: V \isom W$ такой, что $\phi(e_1) = f_1, \ldots, \phi(e_n) = f_n$.
\end{Consequence}
\begin{proof}
Из пункта 2. предложения следует, что существует единственное линейное отображение $\phi: V \rightarrow W$ такое, что $\phi(e_1) = f_1, \ldots, \phi(e_n) = f_n$. Но тогда $\phi(x_1e_1 + \ldots + x_ne_n) \hm= x_1\phi(e_1) + \ldots + x_n\phi(e_n) = x_1f_1 + \ldots + x_nf_n$ для любых $x_i \in F$. Отсюда следует, что $\phi$ --- биекция. 
\end{proof}

\subsection{Матрицы линейных отображений}

Пусть $V$ и $W$ --- векторные пространства, $\mathbb{e} = (e_1, \ldots, e_n)$ --- базис $V$, $\mathbb{f} = (f_1, \ldots, f_m)$ --- базис $W$, $\phi: V \rightarrow W$ --- линейное отображение. Тогда:
\[
\phi(e_j) = a_{1j}f_1 + \ldots + a_{mj}f_m = \sum_{i = 1}^{m}a_{ij}f_i.
\]

\begin{Def}
Матрица $A = (a_{ij}) \in Mat_{m \times n}(F)$ называется \textit{матрицей линейного отображения $\phi$} в базисах $\mathbb{e}$ и $\mathbb{f}$ (или по отношению к базисам $\mathbb{e}$ и $\mathbb{f}$).
\end{Def}

\begin{Comment}
Существует биекция $\{\text{линейные отображения } V \rightarrow W \} \rightleftarrows Mat_{m\times n}$.
\end{Comment}

\begin{Comment}
В $A^{(j)}$ стоят координаты $\phi(e_j)$ в базисе $\mathbb{f}$.
\[
(\phi(e_1), \ldots, \phi(e_n)) = (f_1, \ldots, f_n)\cdot A
\]
\end{Comment}

Рассмотрим пример.

Пусть $P_n = F[x]_{\leqslant n}$ --- множество многочленов над полем $F$ степени не выше $n$. Возьмем дифференцирование $\Delta: P_n \rightarrow P_{n-1}$.

Базис $P_n$ --- $1, x, x^2, \ldots, x^n$. Базис $P_{n-1}$ --- $1, x, \ldots, x^{n-1}$. Тогда матрица линейного отображения будет размерности $n \times (n+1)$ и иметь следующий вид.
\[
\begin{pmatrix*}
0 & 1 & 0 & 0 & \dots & 0 \\
0 & 0 & 2 & 0 & \dots & 0 \\
0 & 0 & 0 & 3 & \dots & 0 \\
\hdotsfor{6} \\
0 & 0 & 0 & 0 & \dots & n
\end{pmatrix*}
\] 

\begin{Suggestion}
Если $v = x_1 e_1 + \ldots + x_ne_n$ и $\phi(v) = y_1f_1 + \ldots + y_mf_m$, то 
\[
\begin{pmatrix*}
y_1 \\
\vdots \\
y_m
\end{pmatrix*}
= A \cdot
\begin{pmatrix*}
x_1 \\
\vdots \\
x_n
\end{pmatrix*}
\]
\end{Suggestion}

\begin{proof}
С одной стороны:
\[
\phi(v) = x_1 \phi(e_1) + \ldots + x_n\phi(e_n) = (\phi(e_1), \ldots, \phi(e_n))\begin{pmatrix*}x_1 \\ \vdots \\ x_n\end{pmatrix*} = (f_1, \ldots, f_m)A\begin{pmatrix*}x_1 \\ \vdots \\ x_n\end{pmatrix*}.
\]
Однако с другой стороны:
\[
\phi(v) = (f_1, \ldots, f_m)\begin{pmatrix*}y_1 \\ \vdots \\ y_m\end{pmatrix*}.
\]
Сравнивая обе части, получаем требуемое.
\end{proof}

А теперь проанализируем операции над матрицами линейных отображений.

$V$ и $W$ --- векторные пространства. \textbf{Обозначение:} $\Hom(V, W):=$ множество всех линейных отображений $V \rightarrow W$.

Пусть $\phi, \psi \in \Hom(V, W)$.

\begin{Def}\ 
\begin{enumerate}
\item $\phi + \psi \in \Hom(V, W)$ --- это $(\phi + \psi)(v):= \phi(v) + \psi(v)$.
\item $\alpha \in F, \alpha\phi \in \Hom(V, W)$ --- это $(\alpha\phi)(v) := \alpha(\phi(v))$.
\end{enumerate}
\end{Def}

\begin{Task}\
\begin{enumerate}
\item Проверить, что $\phi + \psi$ и $\alpha\phi$ действительно принадлежат $\Hom(V, W)$.
\item Проверить, что $\Hom(V, W)$ является векторным пространством.
\end{enumerate}
\end{Task}

\begin{Suggestion}
Пусть $\mathbb{e} = (e_1, \ldots, e_n)$ --- базис $V$, $\mathbb{f} = (f_1, \ldots, f_m)$ --- базис $W$, $\phi, \psi \in \Hom(V, W)$. При этом $A_{\phi}$ --- матрица линейного отображения $\phi$, $A_{\psi}$ --- матрица для $\psi$, $A_{\phi+\psi}$ --- для $\phi + \psi$, а $A_{\alpha\phi}$ --- для $\alpha\phi$.

Тогда $A_{\phi+\psi} = A_{\phi} + A_{\psi}$ и $A_{\alpha\phi} = \alpha A_{\phi}$.
\end{Suggestion}
\begin{proof}
Упражнение.
\end{proof}

Теперь возьмем три векторных пространства --- $U, V$ и $W$ размерности $n, m$ и $k$ соответственно, и их базисы $\mathbb{e}, \mathbb{f}$ и $\mathbb{g}$. Также рассмотрим цепочку линейных отображений $U \xrightarrow{\psi} V \xrightarrow{\phi} W$. Пусть $A$ --- матрица $\phi$ в базисах $\mathbb{f}$ и $\mathbb{g}$, $B$ --- матрица $\psi$ в базисах $\mathbb{e}$ и $\mathbb{f}$, $C$ --- матрица $\phi\circ\psi$ в базисах $\mathbb{e}$ и $\mathbb{g}$.

\begin{Suggestion}
$C = AB$.
\end{Suggestion}
\begin{Comment}
Собственно говоря, отсюда и взялось впервые определение умножения матриц.
\end{Comment}
\begin{proof} Запишем по определению:
\begin{align*}
(\phi \circ \psi)(e_r) &= \sum_{p = 1}^{k}c_{pr}g_p, \quad r = 1, \ldots, n \\
\psi(e_r) &= \sum_{q = 1}^{m}b_{qr}f_q, \quad r = 1, \ldots, n \\
\phi(f_q) &= \sum_{p = 1}^{k}a_{pq}g_p, \quad q = 1, \ldots, m
\end{align*}
Тогда:
\begin{gather*}
(\psi\circ\psi)(e_r) = \phi(\psi(e_r)) = \phi\left(\sum_{q = 1}^{m}b_{qr}f_g \right) = \sum_{q = 1}^{m}b_{qr}\phi(f_g) = \sum_{q = 1}^{m}b_{qr}\left(\sum_{p = 1}^{k}a_{pq}g_p \right) = \sum_{p = 1}^{k}\left(\sum_{q = 1}^{m}a_{pq}b_{qr} \right)g_p \\
\Downarrow \\
c_{pr} = \sum_{q = 1}^{m}a_{pq}b_{qr} \\
\Downarrow\\
 C = AB
\end{gather*}
\end{proof}

И снова, пусть $V$ и $W$ --- векторные пространства с линейным отображением $\phi: V \rightarrow W$.

\begin{Def}
\textit{Ядро $\phi$} --- это множество $\Ker\phi := \{v \in V \mid \phi(v) = 0 \}$.
\end{Def} 
\begin{Def}
\textit{Образ $\phi$} --- это множество $\Im \phi := \{w \in W \mid \exists v \in V : \phi(v) = w \}$.
\end{Def}

\begin{Examples}
Все то же $\Delta: P_n \rightarrow P_{n-1}$. Для него $\Ker \Delta = \{f \mid f = const\},\  \Im \Delta = P_{n-1}$.
\end{Examples}
\end{document}
