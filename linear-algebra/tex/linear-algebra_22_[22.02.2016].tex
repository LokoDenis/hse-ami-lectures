\documentclass[a4paper,12pt]{article}

%% Работа с русским языком
\usepackage{cmap}					% поиск в PDF
\usepackage{mathtext} 				% русские буквы в формулах
\usepackage[T2A]{fontenc}			% кодировка
\usepackage[utf8]{inputenc}			% кодировка исходного текста
\usepackage[english,russian]{babel}	% локализация и переносы

%% Отступы между абзацами и в начале абзаца 
\setlength{\parindent}{0pt}
\setlength{\parskip}{\medskipamount}

%% Изменяем размер полей
\usepackage[top=0.5in, bottom=0.75in, left=0.625in, right=0.625in]{geometry}

%% Графика
\usepackage[pdftex]{graphicx}
\graphicspath{{images/}}

%% Различные пакеты для работы с математикой
\usepackage{mathtools}				% Тот же amsmath, только с некоторыми поправками

\usepackage{amssymb}				% Математические символы
\usepackage{amsthm}					% Пакет для написания теорем
\usepackage{amstext}
\usepackage{array}
\usepackage{amsfonts}
\usepackage{icomma}					% "Умная" запятая: $0,2$ --- число, $0, 2$ --- перечисление
\usepackage{bbm}				    % Для красивого (!) \mathbb с  буквами и цифрами
\usepackage{enumitem}               % Для выравнивания itemise (\begin{itemize}[align=left])

% Номера формул
\mathtoolsset{showonlyrefs=true} % Показывать номера только у тех формул, на которые есть \eqref{} в тексте.

% Ссылки
\usepackage[colorlinks=true, urlcolor=blue]{hyperref}

% Шрифты
\usepackage{euscript}	 % Шрифт Евклид
\usepackage{mathrsfs}	 % Красивый матшрифт

% Свои команды\textbf{}
\DeclareMathOperator{\sgn}{\mathop{sgn}}

% Перенос знаков в формулах (по Львовскому)
\newcommand*{\hm}[1]{#1\nobreak\discretionary{}
{\hbox{$\mathsurround=0pt #1$}}{}}

% Графики
\usepackage{tikz}
\usepackage{pgfplots}
%\pgfplotsset{compat=1.12}

% Изменим формат \section и \subsection:
\usepackage{titlesec}
\titleformat{\section}
{\vspace{1cm}\centering\LARGE\bfseries}	% Стиль заголовка
{}										% префикс
{0pt}									% Расстояние между префиксом и заголовком
{} 										% Как отображается префикс
\titleformat{\subsection}				% Аналогично для \subsection
{\Large\bfseries}
{}
{0pt}
{}

% Информация об авторах
\author{Группа лектория ФКН ПМИ 2015-2016 \\
	Анастасия Иовлева \\
	Ксюша Закирова \\
	Руслан Хайдуров}
\title{Лекции по предмету \\
	\textbf{Линейная алгебра и геометрия}}
\date{2016 год}

\newtheorem*{Def}{Определение}
\newtheorem*{Lemma}{Лемма}
\newtheorem*{Suggestion}{Предложение}
\newtheorem*{Examples}{Пример}
\newtheorem*{Comment}{Замечание}
\newtheorem*{Consequence}{Следствие}
\newtheorem*{Theorem}{Теорема}
\newtheorem*{Statement}{Утверждение}
\newtheorem*{Task}{Упражнение}
\newtheorem*{Designation}{Обозначение}
\newtheorem*{Generalization}{Обобщение}
\newtheorem*{Thedream}{Предел мечтаний}
\newtheorem*{Properties}{Свойства}

\renewcommand{\mathbb}{\mathbbm}
\renewcommand{\Re}{\mathrm{Re\:}}
\renewcommand{\Im}{\mathrm{Im\:}}
\newcommand{\Arg}{\mathrm{Arg\:}}
\renewcommand{\arg}{\mathrm{arg\:}}
\newcommand{\Mat}{\mathrm{Mat}}
\newcommand{\id}{\mathrm{id}}
\newcommand{\isom}{\xrightarrow{\sim}} 
\newcommand{\leftisom}{\xleftarrow{\sim}}
\newcommand{\Hom}{\mathrm{Hom}}
\newcommand{\Ker}{\mathrm{Ker}\:}
\newcommand{\rk}{\mathrm{rk}\:}
\newcommand{\diag}{\mathrm{diag}}
\newcommand{\ort}{\mathrm{ort}}
\newcommand{\pr}{\mathrm{pr}}
\newcommand{\vol}{\mathrm{vol\:}}

\renewcommand{\epsilon}{\varepsilon}
\renewcommand{\phi}{\varphi}
\newcommand{\e}{\mathbb{e}}
\renewcommand{\l}{\lambda}
\renewcommand{\C}{\mathbb{C}}
\newcommand{\R}{\mathbb{R}}
\newcommand{\E}{\mathbb{E}}

\newcommand{\vvector}[1]{\begin{pmatrix}{#1}_1 \\\vdots\\{#1}_n\end{pmatrix}}
\renewcommand{\vector}[1]{({#1}_1, \ldots, {#1}_n)}

\begin{document}
\renewcommand{\phi}{\varphi}

\section{Лекция 22 от 22.02.2016}

\subsection*{Деление многочленов с остатком}

Пусть $F$ -- поле, $\mathbb{F}[x]$ --множество всех множеств от переменных $x$ с коэффициентами из $\mathbb{F}$.
\begin{Theorem}
	Пусть $G(x), H(x) \in \mathbb{F}[x]$ -- ненулевые многочлены, тогда существует и единственная пара $Q(x), R(x) \in \mathbb{F}(x)$, такие что:
	\begin{enumerate}
		\item $G(x) = Q(x)\cdot H(x) + R(x);$
		\item $degR(x) < degH(x)$
	\end{enumerate}
\end{Theorem}
\begin{proof}
Аналогично делению рациональных чисел с остатком.
\end{proof}
Важный частный случай: $H(x) = x - a$
Вспомним теорему Безу:
\begin{Theorem}
	Если $G(x), Q(x) \in \mathbb{F}[x]$ -- ненулевые многочлены, $a \in \mathbb{F}$, то $G(x) = Q(x)(x - a) + R, R = G(a).$
\end{Theorem}
\begin{proof}
	$G(x) = Q(x)\cdot H(x) + R(x);$
	$H(x) = x - a \Rightarrow degR < deg(x - a) \Rightarrow degR = 0$
	Подставим $x = a$, получим:
	$G(a) = Q(a)(a-a) + R = 0 + R = R \Rightarrow G(a) = R$
\end{proof}
\begin{Theorem}
	Многочлен степени $n$ в после комплексных чисел имеет $n$ комплексных корней.
\end{Theorem}
\begin{proof}
	По основной теореме алгебры каждый многочлен $G(x) \in \mathbb{C}[x]$ степени больше 1 имеет корень. Тогда $G(x) = (x - a_1)G_1(x),$ где $a_1$ -- корень многочлена $G(x)$. В свою очередь многочлен $G_1(x)$ также имеет корень и $G(x) = (x - a_1)G_1(x) = (x - a_1)(x - a_2)G_2(x) = \ldots = (x - a_1)(x - a_2)\ldots(x - a_n)b_n$, где $b_n$ -- коэффициент при старшем члене.
\end{proof}
Получим, что $b_nx^n + b_{n-1}x^{n-1} + \ldots + b_0 = b_n(x - a_1)^k_1\ldots(x - a_s)^k_s$
\begin{Def}
	Кратностью корня $a_i$ называется число $k_i$, такое что в многочлене $b_n(x - a_1)^k_1\ldots(x - a_s)^k_s$ множитель $(x - a_i)$ имеет степень $k_i$.
\end{Def}
\begin{Def}
	Пусть $V$ -- конечномерное векторное пространство над полем $\mathbb{F}$. $\phi: V \to V$ -- линейный оператор. Тогда характеристический многочлен $\phi$ имеет вид:
	\[\chi_{\phi}(t) = (-1)^n\det(\phi - tE) =
  \begin{pmatrix}
  a_{11} - t & a_{12} &\ldots &a_{1n}\\
  a_{21} & a_{22} - t &\ldots &a_{2n} \\
  \vdots &\vdots &\ddots &\vdots\\
  a_{n1} &a_{n2} &\ldots & a_{nn} - t
  \end{pmatrix}
  = (-1)^n(t^n(-1)^n + \ldots)  = t^n + \ldots \]
\end{Def}
\begin{Task}
	$c_{n-1}$ -- коэффициент при $t^{n-1}, c_0$ -- свободный член:
	\[c_{n-1} = -tr\phi;\]
        \[c_0 = (-1)^n \det\phi.\]
\end{Task}
\begin{Statement}
	$\lambda$ -- собственное значение $\phi \Leftrightarrow \chi_\phi(\lambda) = 0.$ 
\end{Statement}
\begin{proof}
	$\lambda$ -- собственное значение $\Leftrightarrow \exists v \neq 0: \phi(v) = \lambda {v} \Leftrightarrow A\phi v - \lambda {E}v = 0 \Leftrightarrow (A\phi - \lambda {E})v = 0 \Leftrightarrow \Ker(\phi - \lambda {E}) \neq \{0\}
	\Leftrightarrow \det(\phi - \lambda {E}) = 0 \Leftrightarrow \chi_\phi(\lambda) = 0.$
\end{proof}
\begin{Statement}
	Если $\mathbb{F} = \mathbb{C}, \dim V > 0$, то любой линейный оператор собственный вектор.
\end{Statement}
\begin{proof}
	Пусть $\phi: V \to V$ -- линейный оператор. У него существует характеристический многочлен $\chi_\phi(x)$. Тогда по основной теореме алгебры у $\chi_\phi(x)$ есть корень $t_0$ -- собственное значение $\phi$, следовательно существует и собственный вектор $v_0$  с собственным значением $t_0$.
\end{proof}
\begin{Examples}
	Для линейного оператора $\phi = \begin{pmatrix}
    0& -1 \\
    1& 0
    \end{pmatrix}$
    (поворот на $90^\circ$ градусов против часовой стрелки относительно начала координат), характеристический многочлен имеет вид: $\chi_\phi(x) = t^2+1$.
    \\ При $\mathbb{F}  = \mathbb{R} \Rightarrow$ собственных значений нет.
    \\ При $\mathbb{F} = \mathbb{C} \Rightarrow$ собственные значения $\pm i$.
\end{Examples}
\begin{Def}
	Пусть $\lambda$ -- собственное значение $\phi$, тогда $V_\lambda = \{v \ in V \; | \; \phi v = \lambda v\}$ -- собственное подпространство (пространство, состоящее из собственных векторов с собственным значением $\lambda$ и нуля).
\end{Def}
\begin{Def}
	$\dim V_\lambda$ -- геометрическая кратность собственного значения $\lambda$.
\end{Def}
\begin{Def}
	Если $k$ -- кратность корня (определение см. выше, $(x - a_k)^k$), то $k$ -- алгебраическая кратность корня.
\end{Def}
\begin{Statement}
	Геометрическая кратность не больше алгебраической кратности.
\end{Statement}
\begin{proof}
	Зафиксируем базис $u_1, \ldots, u_p$ в пространстве $V_\lambda (p = \dim{V_\lambda})$. Дополним базис $u_1, \ldots, u_p$ до базиса $u_1, \ldots, u_p, u_{p+1}, \ldots, u_n$ пространства $V$. Матрица линейного оператора $\phi$ будет выглядеть следующим образом:
	\\ (тут должна быть блочная матрица)
	\[\chi_\phi(t) = (-1)^n \text{блочная матрица} = (-1)^n(\lambda - t)^p\dim(B - tE)\]
	\\ $\chi_\phi(t)$ имеет корень кратности хотя бы $p$, следовательно геометрическая кратность $= p \leqslant$ алгебраическая кратность. 
\end{proof}
\begin{Examples} Когда алгебраическая кратность больше геометрической. Для линейного оператора $\phi = \begin{pmatrix}
    2& 1 \\
    0& 2
    \end{pmatrix}
    \\ V_2 = <e_1> \Rightarrow \text{геом. кратность} = 1, \chi_\phi(t) = (t-2)^2 \Rightarrow \text{алг. кратность} = 2.$
\end{Examples}
\begin{Def}
	Пусть $\{U_1, \ldots, U_k \subseteq V\}$. Прямая сумма нескольких пространств -- это $U_1 + \ldots + U_k = \{u_1 + \ldots + u_k \; | \; u_i \in U_i \}$
\end{Def}
\begin{Task}
	$U_1+\ldots + U_k$ -- подпространство.
\end{Task}
\begin{Def}
	Сумма называет прямой, если  $U_1 + \ldots + U_k = 0 \Rightarrow U_1 = \ldots = U_k = 0.$
\end{Def}
\begin{Task}
	Если $v \in U_1 \oplus \ldots \oplus U_k$, то существует и единственный набор $u_1 \in U_1, \ldots, u_k \in U_k: v = u_1 + \ldots + u_k.$
\end{Task}
\begin{Theorem}
	Следующие условия эквивалентны:
	\begin{enumerate}
		\item Сумма $U_1 + \ldots + U_k$ -- прямая;
		\item Если $\mathbb{e}_i$ -- базис $U_i (\mathbb{e}_i \cap \mathbb{e}_j)$, то $\mathbb{e} = \mathbb{e}_1 \cup \ldots \cup \mathbb{e}_k$ -- базис $U_1 + \ldots + U_k;$
		\item $\dim(U_1 + \ldots + U_k) = \dim{U_1} + \ldots + \dim{U_k}.$
	\end{enumerate}
\end{Theorem}
\begin{proof}
	$(1) \Rightarrow (2).$ Сумма $U_1 + \ldots + U_k$ прямая. Покажем, что $\mathbb{e}_1 \cup \ldots \cup \mathbb{e}_k$ -- базис $U_1 + \ldots + U_k.$
	\\ Если $v \in U_1 \oplus \ldots \oplus U_k$, то  $v = u_1 + \ldots + u_k = \{ u_i \in U_i\} = c^1_1e^1_1 + \ldots + c^1_{s_1}e^1_{s_1} + \ldots + c^k_1e^k_1 + \ldots c^k_{sk}e^k_{sk}$, но $\mathbb{e}$ -- базис.
	\\ Пусть существует два представления, тогда вычтем из одного второе. По определению прямой суммы каждый вектор равен нулю, следовательно коэффициенты при них равны.
	\\ $(2) \Rightarrow (1).$ Пусть $\mathbb{e} = \mathbb{e}_1 \cup \ldots \cup \mathbb{e}_k$ -- базис $U_1 + \ldots + U_k$. Пусть $0 = u_1 + \ldots + u_k$. Разложим по базисам:
	\\ $0 = c^1_1e^1_1 + \ldots + c^1_{s_1}e^1_{s_1} + \ldots + c^k_1e^k_1 + \ldots c^k_{sk}e^k_{sk}$, следовательно все коэффициенты равны $0$ и $u_1 = 0 = u_k.$
	\\ $(2) \Rightarrow (3).$ Пусть $\mathbb{e} = \mathbb{e}_1 \cup \ldots \cup \mathbb{e}_k$ -- базис $U_1 + \ldots + U_k. \dim(U_1 + \ldots + U_k)  = \dim(\mathbb{e}) = \dim(\mathbb{e_1})+ \ldots + \dim(\mathbb{e_k}) = \dim(U_1) + \ldots + \dim(U_k).$
	\\ $(3) \Rightarrow (2).$ Пусть $\dim(U_1 + \ldots + U_k) = \dim{U_1} + \ldots + \dim{U_k}.$
	\\ $\mathbb{e}$ порождает сумму, следовательно из $\mathbb{e}$ можно выделить базис суммы:
	\\ $\dim(U_1 + \ldots + U_k) \leqslant \dim(\mathbb{e}) \leqslant \dim(\mathbb{e_1})+ \ldots + \dim(\mathbb{e_k}) = \dim{U_1} + \ldots + \dim{U_k}.$
 \end{proof}
\end{document}

