\documentclass[a4paper,12pt]{article}

%% Работа с русским языком
\usepackage{cmap}					% поиск в PDF
\usepackage{mathtext} 				% русские буквы в формулах
\usepackage[T2A]{fontenc}			% кодировка
\usepackage[utf8]{inputenc}			% кодировка исходного текста
\usepackage[english,russian]{babel}	% локализация и переносы

%% Отступы между абзацами и в начале абзаца 
\setlength{\parindent}{0pt}
\setlength{\parskip}{\medskipamount}

%% Изменяем размер полей
\usepackage[top=0.5in, bottom=0.75in, left=0.625in, right=0.625in]{geometry}

%% Графика
\usepackage[pdftex]{graphicx}
\graphicspath{{images/}}

%% Различные пакеты для работы с математикой
\usepackage{mathtools}				% Тот же amsmath, только с некоторыми поправками

\usepackage{amssymb}				% Математические символы
\usepackage{amsthm}					% Пакет для написания теорем
\usepackage{amstext}
\usepackage{array}
\usepackage{amsfonts}
\usepackage{icomma}					% "Умная" запятая: $0,2$ --- число, $0, 2$ --- перечисление
\usepackage{bbm}				    % Для красивого (!) \mathbb с  буквами и цифрами
\usepackage{enumitem}               % Для выравнивания itemise (\begin{itemize}[align=left])

% Номера формул
\mathtoolsset{showonlyrefs=true} % Показывать номера только у тех формул, на которые есть \eqref{} в тексте.

% Ссылки
\usepackage[colorlinks=true, urlcolor=blue]{hyperref}

% Шрифты
\usepackage{euscript}	 % Шрифт Евклид
\usepackage{mathrsfs}	 % Красивый матшрифт

% Свои команды\textbf{}
\DeclareMathOperator{\sgn}{\mathop{sgn}}

% Перенос знаков в формулах (по Львовскому)
\newcommand*{\hm}[1]{#1\nobreak\discretionary{}
{\hbox{$\mathsurround=0pt #1$}}{}}

% Графики
\usepackage{tikz}
\usepackage{pgfplots}
%\pgfplotsset{compat=1.12}

% Изменим формат \section и \subsection:
\usepackage{titlesec}
\titleformat{\section}
{\vspace{1cm}\centering\LARGE\bfseries}	% Стиль заголовка
{}										% префикс
{0pt}									% Расстояние между префиксом и заголовком
{} 										% Как отображается префикс
\titleformat{\subsection}				% Аналогично для \subsection
{\Large\bfseries}
{}
{0pt}
{}

% Информация об авторах
\author{Группа лектория ФКН ПМИ 2015-2016 \\
	Анастасия Иовлева \\
	Ксюша Закирова \\
	Руслан Хайдуров}
\title{Лекции по предмету \\
	\textbf{Линейная алгебра и геометрия}}
\date{2016 год}

\newtheorem*{Def}{Определение}
\newtheorem*{Lemma}{Лемма}
\newtheorem*{Suggestion}{Предложение}
\newtheorem*{Examples}{Пример}
\newtheorem*{Comment}{Замечание}
\newtheorem*{Consequence}{Следствие}
\newtheorem*{Theorem}{Теорема}
\newtheorem*{Statement}{Утверждение}
\newtheorem*{Task}{Упражнение}
\newtheorem*{Designation}{Обозначение}
\newtheorem*{Generalization}{Обобщение}
\newtheorem*{Thedream}{Предел мечтаний}
\newtheorem*{Properties}{Свойства}

\renewcommand{\mathbb}{\mathbbm}
\renewcommand{\Re}{\mathrm{Re\:}}
\renewcommand{\Im}{\mathrm{Im\:}}
\newcommand{\Arg}{\mathrm{Arg\:}}
\renewcommand{\arg}{\mathrm{arg\:}}
\newcommand{\Mat}{\mathrm{Mat}}
\newcommand{\id}{\mathrm{id}}
\newcommand{\isom}{\xrightarrow{\sim}} 
\newcommand{\leftisom}{\xleftarrow{\sim}}
\newcommand{\Hom}{\mathrm{Hom}}
\newcommand{\Ker}{\mathrm{Ker}\:}
\newcommand{\rk}{\mathrm{rk}\:}
\newcommand{\diag}{\mathrm{diag}}
\newcommand{\ort}{\mathrm{ort}}
\newcommand{\pr}{\mathrm{pr}}
\newcommand{\vol}{\mathrm{vol\:}}

\renewcommand{\epsilon}{\varepsilon}
\renewcommand{\phi}{\varphi}
\newcommand{\e}{\mathbb{e}}
\renewcommand{\l}{\lambda}
\renewcommand{\C}{\mathbb{C}}
\newcommand{\R}{\mathbb{R}}
\newcommand{\E}{\mathbb{E}}

\newcommand{\vvector}[1]{\begin{pmatrix}{#1}_1 \\\vdots\\{#1}_n\end{pmatrix}}
\renewcommand{\vector}[1]{({#1}_1, \ldots, {#1}_n)}

\begin{document}
\section{Лекция 17 от 25.01.2016}
\subsection{Овеществление и комплексификация}
Пусть $V$ --- векторное пространство над $\mathbb{C}$.
\begin{Def}
Овеществление пространства $V$ --- это то же пространство $V$, рассматриваемое как пространство над $\mathbb{R}$. Обозначение: $V_\mathbb{R}$.
\end{Def}
Операция умножения на элементы $\mathbb{R}$ в $V$ уже есть, так как $\mathbb{R}$ --- подполе в $\mathbb{C}$.

\begin{Examples}
$\mathbb{C}_\mathbb{R} = \mathbb{R}^2$.
\end{Examples}
\begin{Suggestion}
$V$ --- векторное пространство над $\mathbb{C}$, $\dim V < \infty$. Тогда $\dim V_\mathbb{R} = 2\dim V$.
\end{Suggestion}
\begin{proof}
Пусть $e_1, \ldots, e_n$ --- базис в $V$. Тогда $V = \{z_1e_1 + \ldots + z_ne_n\ |\ z_k \in \mathbb{C}\}$, причём такая запись единственная в силу определения базиса. Пусть $z_k = a_k+ib_k$, причём такая запись тоже единственная. Тогда будем иметь
\begin{gather*}
V = \{ \left(a_1+ib_1\right)e_1 + \ldots + \left(a_n+ib_n\right)e_n\ |\ a_k, b_k \in \mathbb{R}\} =\\
= \{a_1e_1 + \ldots + a_ne_n + b_1ie_1 + \ldots + b_nie_n\ |\ a_k, b_k \in \mathbb{R}\}
\end{gather*}
И причём такая запись тоже единственная. Выходит, что $e_1, e_2, \ldots, e_n, ie_1, ie_2, \ldots, ie_n$ --- базис в $V_\mathbb{R}$, в котором $2n = 2\dim V$ элементов.
\end{proof}
\begin{Def}
Комплексификация пространства $W$ --- это множество $W\times W = W^\mathbb{C} \hm{=} \{\left( u, v\right)\ |\ u,v \in W\}$ с операциями $\left(u_1, v_1\right) + \left(u_2, v_2\right) = \left(u_1+u_2, v_1+v_2\right)$, $\left(a+ib\right)\left(u, v\right) \hm{=} \left(au-bv, av-bu\right)$.
\end{Def}
\begin{Examples}
$\mathbb{R}^\mathbb{C} = \mathbb{R}$.
\end{Examples}

\begin{Statement}
В нём выполняются все 8 аксиом векторного пространства над $\mathbb{C}$.
\end{Statement}
$W$ отождествляется подмножеством $\{\left(u, 0\right)\ |\ u\in W\}$. Действительно
\[
w\in W\Leftrightarrow \left(w,0\right) \in W^\mathbb{C};\ i\left(w,0\right) = \left(0, w\right) \in W^\mathbb{C}
\]
В итоге $\forall \left(u, v\right) \in W^\mathbb{C}$ представим в виде 
\[
\left(u,v\right) = \left(u,0\right) + \left(0,v\right) = \left(u,0\right) + i\left(v,0\right) = u+iv
\]
То есть $W^\mathbb{C} = \{u+iv\ |\ u,v\in W\}$. 
\begin{Suggestion}
$\dim W^\mathbb{C} = \dim W$
\end{Suggestion}
\begin{Comment}
Здесь $W^\mathbb{C}$ --- пространство над $\mathbb{C}$, а $W$ --- над $\mathbb{R}$.
\end{Comment}
\begin{proof}
Пусть $e_1, \ldots, e_n$ --- базис в $W$. Тогда 
\begin{gather*}
W^\mathbb{C} = \{\left(u,v\right)\ |\ u,v \in W\} = \{\left(a_1e_1 + a_2e_2 + \ldots + a_ne_n, b_1e_1 + b_2e_2 + \ldots + b_ne_n\right)\ |\ a_k,b_k \in \mathbb{R}\} = \\
= \{\left(a_1e_1,b_1e_1\right) + \ldots + \left(a_ne_n, b_ne_n\right)\} = \{\left(a_1+ib_1\right)e_1 + \ldots +\left(a_n + ib_n)e_n\right)\} = \\
= \{z_1e_1 + \ldots + z_ne_n\ |\ z_k \in \mathbb{C} \}
\end{gather*}
То есть выходит, что $e_1, \ldots, e_n$ --- базис в $W^\mathbb{C}$.
\end{proof}

\subsection{Сумма подпространств}
Пусть $V$ --- конечномерное векторное пространство, а $U$ и $W$ --- подпространства (в качестве упражнения лектор предлагает доказать, что их пересечение --- тоже подпространство).

\begin{Def}
Сумма подпространств $U$ и $W$ --- это множество.
\[
U+W = \{u + w\ |\ u \in U, w \in W\}
\]
\end{Def}

\begin{Comment}
$\dim \left( U \cap W \right) \leqslant \dim U \leqslant \dim \left(U + W\right)$
\end{Comment}

\begin{Examples}
Двумерные плоскости в пространстве $\mathbb{R}^3$ содержат общую прямую.
\end{Examples}

\begin{Theorem}
$\dim \left(U \cap W\right) = \dim U + \dim W - \dim \left(U+W\right)$
\end{Theorem}

\begin{proof}
Положим $p = \dim \left(U \cap W\right)$, $k = \dim U$, $m = \dim W$. Выберем базис $a \hm{=} \{ a_1, \ldots, a_p\}$ в пересечении. Его можно дополнить до базиса $W$ и до базиса $U$. Значит $\exists b = \{ b_1, \ldots, b_{k-p}\}$ такой, что $a\cup b$ --- базис в $U$ и $\exists c = \{ c_1, \ldots, c_{m-p}\}$ такой, что $a \cup c$ --- базис в $W$.

Докажем, что $a \cup b \cup c$ --- базис в $U+W$.

Во-первых, докажем, что $U+W$ порождается множеством $a \cup b \cup c$.
\begin{gather*}
\left.
\begin{aligned}
    &v \in U+W \Rightarrow \exists u \in U, w \in W\colon \ v = u+w\\
    &u \in U=\langle a \cup b\rangle \subset \langle a \cup b \cup c\rangle\\
    &w \in W=\langle a \cup c\rangle \subset \langle a \cup b \cup c\rangle\\
\end{aligned}
\right|
    \Rightarrow v = u + w \in \langle a \cup b \cup c\rangle 
    \Rightarrow U + W = \langle a \cup b \cup c\rangle
\end{gather*}

Во-вторых, докажем линейную независимость векторов из $a \cup b \cup c$.

Пусть скаляры $\alpha_1, \ldots, \alpha_p$, $\beta_1, \ldots, \beta_{k-p}$, $\gamma_1, \ldots, \gamma_{m-p}$ таковы, что:
\begin{gather*}
\underbrace{\alpha_1a_1+ \ldots +\alpha_pa_p}_x + \underbrace{\beta_1b_1+ \ldots +\beta_{k-p}b_{k-p}}_y + \underbrace{\gamma_1c_1 + \ldots + \gamma_{m-p}c_{m-p}}_z=0\\
x+y+z = 0\\
z = -x -y\\
z \in W\\
-x-y \in U\cap W\\
\Rightarrow \exists \lambda_1,\ldots, \lambda_p \in F \colon z = \lambda_1a_1+\ldots+\lambda_pa_p 
\end{gather*}
Тогда $\lambda_1a_1 + \ldots + \lambda_pa_p - \gamma_1c_1 - \ldots - \gamma_{m-p}c_{m-p} = 0$. Но $a \cup c$ --- базис $W$. Следовательно, $\lambda_1 \hm{=} \ldots = \lambda_p = \gamma_1 = \ldots = \gamma_{m-p} = 0$. Но тогда $0 = x+y = \alpha_1a_1 + \ldots + \alpha_pa_p + \beta_1b_1 + \ldots + \beta_{k-p}b_{k-p}$. Но $a\cup b$ --- базис $U+W \Rightarrow \alpha_1 = \ldots = \alpha_p = \beta_1 = \ldots = \beta_{k-p} = 0$. Итого, все коэффициенты равны нулю и линейная независимость тем самым доказана. То есть $a \cup b \cup c$ --- базис $U+W$. 
\begin{gather*}
    \dim \left(U+W\right) = |a\cup b \cup c| = |a| + |b| + |c| = p + k-p + m-p = k + m -p =\\
    =\dim U + \dim W - \dim\left(U\cap W\right)
\end{gather*}
\end{proof}

\begin{Def}
Если $U \cap W = \{0\}$, то $U + W$ называется прямой суммой.
\end{Def}
\begin{Consequence}
В таком случае $\dim\left(U+W\right) = \dim U + \dim W$.
\end{Consequence}
\begin{Examples}
    $U$ --- плоскость, $W$ --- прямая в $\mathbb{R}^3$.
\end{Examples}

\subsection{Переход к новому базису}
Пусть $V$ --- векторное пространство, $\dim V = n$, $e_1, \ldots, e_n$ --- базис. То есть
\[
\forall v \in V \quad \exists!\: v = x_1e_1 + \ldots + x_ne_n,
\]
где $x_1, \ldots, x_n \in F$ --- координаты вектора $v$ в базисе $\left(e_1, \ldots, e_n\right)$. 
Пусть также есть базис $e_1', \ldots, e_n'$:
\begin{align*}
    e_1' &= c_{11}e_1 + c_{21}e_2 + \ldots + c_{n1}e_n\\
    e_2' &= c_{12}e_2 + c_{22}e_2 + \ldots + c_{n2}e_n\\
    \vdots\\
    e_n' &= c_{1n}e_1 + c_{2n}e_2 + \ldots + c_{nn}e_n
\end{align*}
Обозначим матрицу $C = \left(c_{ij}\right)$. Тогда можно переписать $\left(e_1', \ldots, e_n'\right)$ как $\left(e_1, \ldots, e_n\right)\cdot C$.
\begin{Suggestion}
$e_1', \ldots, e_n'$ образуют базис тогда и только тогда, когда $\det C \neq 0$.
\end{Suggestion}

\begin{proof} \ 
\begin{itemize}
\item[{$[\Rightarrow]$}] $e_1', \ldots, e_n'$ --- базис, а значит $\exists C' \in M_n \colon$
\begin{gather*}
\left(e_1, \ldots, e_n\right) = \left(e_1', \ldots, e_n'\right)C' = \left(e_1, \ldots, e_n\right)C'C\\
E = CC'\\
C' = C^{-1} \Leftrightarrow \exists C^{-1} \Leftrightarrow \det{C} \neq 0
\end{gather*}
\item[{$[\Leftarrow]$}] $\det C \neq 0 \Rightarrow \exists C^{-1}$. Покажем, что $e_1', \ldots, e_n'$ в таком случае линейно независимы. Пусть $x_1e_1' + x_2e_2' + \ldots + x_ne_n' = 0$. Тогда можно записать
\begin{gather*}
\left(e_1', e_2', \ldots, e_n'\right) 
\begin{pmatrix}
    x_1\\
    x_2\\
    \vdots\\
    x_n
\end{pmatrix} = 0\\
\left(e_1, \ldots, e_n\right)C\begin{pmatrix}
    x_1\\
    x_2\\
    \vdots\\
    x_n
\end{pmatrix} = 0
\end{gather*}
Поскольку $\left(e_1, \ldots, e_n\right)$ --- базис, то $C \begin{pmatrix}
    x_1\\
    x_2\\
    \vdots\\
    x_n
\end{pmatrix} = 0$. Умножая слева на обратную матрицу, получаем, что $x_1 = x_2 = \ldots = x_n = 0$
\end{itemize}
\end{proof}
\end{document}
