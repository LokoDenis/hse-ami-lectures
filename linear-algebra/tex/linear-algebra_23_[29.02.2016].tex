\documentclass[a4paper,12pt]{article}

%% Работа с русским языком
\usepackage{cmap}					% поиск в PDF
\usepackage{mathtext} 				% русские буквы в формулах
\usepackage[T2A]{fontenc}			% кодировка
\usepackage[utf8]{inputenc}			% кодировка исходного текста
\usepackage[english,russian]{babel}	% локализация и переносы

%% Отступы между абзацами и в начале абзаца 
\setlength{\parindent}{0pt}
\setlength{\parskip}{\medskipamount}

%% Изменяем размер полей
\usepackage[top=0.5in, bottom=0.75in, left=0.625in, right=0.625in]{geometry}

%% Графика
\usepackage[pdftex]{graphicx}
\graphicspath{{images/}}

%% Различные пакеты для работы с математикой
\usepackage{mathtools}				% Тот же amsmath, только с некоторыми поправками

\usepackage{amssymb}				% Математические символы
\usepackage{amsthm}					% Пакет для написания теорем
\usepackage{amstext}
\usepackage{array}
\usepackage{amsfonts}
\usepackage{icomma}					% "Умная" запятая: $0,2$ --- число, $0, 2$ --- перечисление
\usepackage{bbm}				    % Для красивого (!) \mathbb с  буквами и цифрами
\usepackage{enumitem}               % Для выравнивания itemise (\begin{itemize}[align=left])

% Номера формул
\mathtoolsset{showonlyrefs=true} % Показывать номера только у тех формул, на которые есть \eqref{} в тексте.

% Ссылки
\usepackage[colorlinks=true, urlcolor=blue]{hyperref}

% Шрифты
\usepackage{euscript}	 % Шрифт Евклид
\usepackage{mathrsfs}	 % Красивый матшрифт

% Свои команды\textbf{}
\DeclareMathOperator{\sgn}{\mathop{sgn}}

% Перенос знаков в формулах (по Львовскому)
\newcommand*{\hm}[1]{#1\nobreak\discretionary{}
{\hbox{$\mathsurround=0pt #1$}}{}}

% Графики
\usepackage{tikz}
\usepackage{pgfplots}
%\pgfplotsset{compat=1.12}

% Изменим формат \section и \subsection:
\usepackage{titlesec}
\titleformat{\section}
{\vspace{1cm}\centering\LARGE\bfseries}	% Стиль заголовка
{}										% префикс
{0pt}									% Расстояние между префиксом и заголовком
{} 										% Как отображается префикс
\titleformat{\subsection}				% Аналогично для \subsection
{\Large\bfseries}
{}
{0pt}
{}

% Информация об авторах
\author{Группа лектория ФКН ПМИ 2015-2016 \\
	Анастасия Иовлева \\
	Ксюша Закирова \\
	Руслан Хайдуров}
\title{Лекции по предмету \\
	\textbf{Линейная алгебра и геометрия}}
\date{2016 год}

\newtheorem*{Def}{Определение}
\newtheorem*{Lemma}{Лемма}
\newtheorem*{Suggestion}{Предложение}
\newtheorem*{Examples}{Пример}
\newtheorem*{Comment}{Замечание}
\newtheorem*{Consequence}{Следствие}
\newtheorem*{Theorem}{Теорема}
\newtheorem*{Statement}{Утверждение}
\newtheorem*{Task}{Упражнение}
\newtheorem*{Designation}{Обозначение}
\newtheorem*{Generalization}{Обобщение}
\newtheorem*{Thedream}{Предел мечтаний}
\newtheorem*{Properties}{Свойства}

\renewcommand{\mathbb}{\mathbbm}
\renewcommand{\Re}{\mathrm{Re\:}}
\renewcommand{\Im}{\mathrm{Im\:}}
\newcommand{\Arg}{\mathrm{Arg\:}}
\renewcommand{\arg}{\mathrm{arg\:}}
\newcommand{\Mat}{\mathrm{Mat}}
\newcommand{\id}{\mathrm{id}}
\newcommand{\isom}{\xrightarrow{\sim}} 
\newcommand{\leftisom}{\xleftarrow{\sim}}
\newcommand{\Hom}{\mathrm{Hom}}
\newcommand{\Ker}{\mathrm{Ker}\:}
\newcommand{\rk}{\mathrm{rk}\:}
\newcommand{\diag}{\mathrm{diag}}
\newcommand{\ort}{\mathrm{ort}}
\newcommand{\pr}{\mathrm{pr}}
\newcommand{\vol}{\mathrm{vol\:}}

\renewcommand{\epsilon}{\varepsilon}
\renewcommand{\phi}{\varphi}
\newcommand{\e}{\mathbb{e}}
\renewcommand{\l}{\lambda}
\renewcommand{\C}{\mathbb{C}}
\newcommand{\R}{\mathbb{R}}
\newcommand{\E}{\mathbb{E}}

\newcommand{\vvector}[1]{\begin{pmatrix}{#1}_1 \\\vdots\\{#1}_n\end{pmatrix}}
\renewcommand{\vector}[1]{({#1}_1, \ldots, {#1}_n)}

\begin{document}

\section{Лекция 23 от 29.02.2016}

Вспомним, чем закончилась прошлая лекция.

Пусть $V$ --- векторное пространство, $U_1, \dots, U_k \subseteq V$ --- векторные подпространства.

Сумма $U = U_1 + \ldots + U_k$ является прямой, если из условия $u_1 + \ldots + u_k = 0$ следует, что $u_1 = \ldots = u_k = 0$, где $u_i \in U_i$. Обозначение: $U = U_1 \oplus \ldots \oplus U_k$.

\vspace{0.2cm}
Эквивалентные условия:
\begin{enumerate}
\item $U = U_1 \oplus \ldots \oplus U_k$.
\item Если $\e_i$ --- базис $U_i$, то $\e = \e_1 \cup \ldots \cup \e_k$ --- базис $U$.
\item $\dim U = \dim U_1 + \ldots + \dim U_k$.
\end{enumerate}

Пусть $V$ --- векторное пространство над полем $F$, $\phi \in L(V)$, $\lambda_1, \ldots, \lambda_k$ --- набор собственных значений $\phi$, где $\lambda_i \neq \lambda_j$ при $i \neq j$, и $V_{\lambda_i}(\phi) \subseteq V$ --- соответствующее собственное подпространство.

\begin{Suggestion}
Сумма $V_{\lambda_1}(\phi) + \ldots + V_{\lambda_k}(\phi)$ является прямой.
\end{Suggestion}

\begin{proof}
Докажем индукцией по $k$.

База: $k = 1$. Тут все ясно.

Теперь пусть утверждение доказано для всех значений, меньших $k$. Докажем для $k$.

Пусть $v_i \in V_{\lambda_i}(\phi)$ и пусть $v_1 + \ldots + v_k = 0$. Тогда:
\begin{gather*}
\phi(v_1 + \ldots + v_k) = \phi(0) = 0 \\
\phi(v_1) + \ldots + \phi(v_k) = 0 \\
\lambda_1v_1 + \ldots + \lambda_kv_k = 0
\end{gather*}
Теперь вычтем из нижней строчки $v_1 + \ldots + v_k = 0$, домноженное на $\lambda_k$. Получим:
\begin{gather*}
(\lambda_1 - \lambda_k)v_1 + \ldots + (\lambda_k -\lambda_k)v_k = 0 \\
(\lambda_1 - \lambda_k)v_1 + \ldots + (\lambda_{k-1} -\lambda_k)v_{k-1} + 0v_k= 0
\end{gather*}
Но из предположения индукции, а также потому что $\lambda_i \neq \lambda_j$ при $i \neq j$, следует, что $v_1 = \ldots \hm= v_{k-1} = 0$. Но тогда и $v_k = 0$. 

Следовательно, сумма прямая, что нам и требовалось.
\end{proof}

\begin{Consequence}
Если характеристический многочлен имеет ровно $n$ попарно различных корней, где $n = \dim V$, то $\phi$ диагонализируем.
\end{Consequence}

\begin{proof}
Пусть $\l_1, \ldots, \l_n$ --- корни $\chi_\phi(t)$, $\l_i \neq \l_j$. Тогда для всех $i$ выполняется, что $V_{\l_i}(\phi) \neq \{0\}$ и, следовательно, $\dim V_{\l_i}(\phi) = 1$. Но так как сумма $V_{\l_1}(\phi) + \ldots + V_{\l_k}(\phi)$ --- прямая, то $\dim (V_{\l_1}(\phi) + \ldots + V_{\l_k}(\phi)) = \dim V_{\l_1}(\phi) + \ldots + \dim V_{\l_k}(\phi) = n$. Иными словами, $V = V_{\l_1}(\phi) \oplus \ldots \oplus V_{\l_k}(\phi)$.

Выберем произвольные $v_i \in V_{\l_i}\setminus\{0\}$. Тогда $(v_1, \ldots, v_n)$ будет базисом в $V$. И так как все $v_i$~--- собственные значения для $\phi$, то $\phi$ диагонализируем.
\end{proof}

\begin{Theorem}[Критерий диагонализируемости - 2]
Линейный оператор $\phi$ диагонализируем тогда и только тогда, когда 
\begin{enumerate}
\item $\chi_\phi(t)$ разлагается на линейные множители;
\item Если $\chi_\phi(t) = (t - \l_1)^{k_1}\dots(t - \l_s)^{k_s}$, где $\l_i \neq \l_j$ при $i \neq j$, то $\dim V_{\l_i}(\phi) = k_i \ \forall i$ (то есть для любого собственного значения $V$ равны геометрическая и алгебраическая кратности).
\end{enumerate}
\end{Theorem}

\begin{proof}\
\begin{itemize}
\item[$\Rightarrow$] Так как $\phi$ --- диагонализируем, то существует базис $\e = (e_1, \ldots, e_n)$ такой, что:
\begin{gather*}
A(\phi, \e) = 
\begin{pmatrix*}
\mu_1 & & 0 \\
& \ddots & \\
0 & & \mu_n
\end{pmatrix*} = \diag(\mu_1, \ldots, \mu_n).
\end{gather*}
Тогда:
$$
\chi_\phi(t) = (-1)^n 
\begin{vmatrix}
\mu_1 & & 0 \\
& \ddots & \\
0 & & \mu_n
\end{vmatrix} = (-1)^n(\mu_1 - t)\ldots(\mu_n - t) = (t - \mu_1)\ldots(t-\mu_n).
$$
Итого, первое условие выполняется.

Теперь перепишем характеристический многочлен в виде $\chi_\phi(t) = (t - \l_1)^{k_1}\dots(t - \l_s)^{k_s}$, где $\l_i \neq \l_j$ при $i \neq j$ и $\{ \l_1, \ldots, \l_s \} = \{\mu_1, \ldots, \mu_n \}$. Тогда $V_{\l_i} \supseteq \langle e_j \mid \mu_j = \l_i \rangle$, следовательно, $\dim V_{\l_i}(\phi) \geqslant k_i$. Но мы знаем, что $\dim V_{\l_i} \leqslant k_i$! Значит, $\dim V_{l_i} = k_i$.

\item[$\Leftarrow$] Так как $V_{\l_1}(\phi) + \ldots + V_{\l_s}(\phi)$ --- прямая, то $\dim (V_{\l_1}(\phi) + \ldots + V_{\l_s}(\phi)) = k_1 + \ldots + k_s = n$.

Пусть $\e_i$ --- базис в $V_{\l_i}$. Тогда $\e_1 \cup \ldots \cup \e_s$ --- базис в $V$. То есть мы нашли базис из собственных векторов, следовательно, $\phi$ диагонализируем.
\end{itemize}
\end{proof}

Пусть $V$ --- векторное пространство над полем $\C$, $\phi \in L(V)$. Тогда в $V$ есть собственный вектор (или одномерное $\phi$--инвариантное пространство).

\vspace{0.2cm}
Теперь пусть $V$ --- векторное пространство над полем $\R$, $\phi \in L(V)$.

\begin{Theorem}
Существует одномерное или двумерное $\phi$--инвариантное векторное подпространство.
\end{Theorem}

\begin{proof}
Пусть $\e = (e_1, \ldots, e_n)$ --- базис в $V$. Комплексифицируем $V$.
\begin{gather*}
V^{\C} = \{ u + iv \mid u, v \in V \} \\
V^{\C} \supset V = \{u + i0 \mid u \in V  \}
\end{gather*}
Рассмотрим линейный оператор $\phi_\C \in L(V^\C)$, заданный как $\phi_\C(e_i) = \phi(e_i),\ \forall i$. Значит, $e_1, \ldots, e_n$~--- базис в $V^\C$. Следовательно, $\chi_{\phi_\C}(t) = \chi_\phi(t)$, так как $A(\phi_\C, \e) = A(\phi, \e)$.

\underline{Случай 1}: $\chi_\phi(t)$ имеет один действительный корень. Отсюда следует, что в $V$ есть собственный вектор, что равносильно существованию одномерного $\phi$--инвариантного подпространства (тогда $V^\C$ нам не нужен).

\underline{Случай 2}: $\chi_\phi$ не имеет действительных корней. Пусть $\l + i\mu$ --- некоторый корень $\chi_\phi(t) $, который, напомним, равен $\chi_{\phi_\C}(t)$. Тогда существует собственный вектор $u+iv \in V^\C$ такой, что:
\begin{gather*}
\phi_\C(u+iv) = (\l + i\mu)(u+iv)\\
\phi_\C(u + iv) = \phi_\C(u) + i\phi_\C(v) = \phi(u) + i\phi(v) \\
(\l + u\mu)(u + iv) = \l \mu - \mu v + i(\mu u + \l v)
\end{gather*}
Сравнивая два последних равенства, получаем:
\begin{gather*}
\phi(u) =  \l u - \mu v \\
\phi(v) = \mu u + \l v
\end{gather*}
Следовательно, $\langle u, v \rangle$ --- $\phi$--инвариантное подпространство, двумерное если $u$ и $v$ линейно независимы и одномерное в противном случае.
\end{proof}

\begin{Examples}
Поворот на $\alpha$ в $\R^2$: $
\begin{pmatrix}
\cos \alpha & -\sin \alpha \\
\sin \alpha & \cos \alpha  
\end{pmatrix}
$. Тогда $u = e_1$, $v = e_2$, $\l + i\mu = \cos \alpha + i\sin \alpha$.
\end{Examples}

Пусть $V$ --- векторное пространство над $F$, $\dim V = n$.

\vspace{0.2cm}
\underline{Операции над $L(V)$}:
\begin{enumerate}
\item Сложение: $(\phi + \psi)(v) = \phi(v) + \psi(v)$.
\item Умножение на скаляр: $(\alpha\phi)(v) = \alpha\phi(v)$.
\item Умножение: $(\phi\psi)(v) = \phi(\psi(v))$.
\end{enumerate}
В частности, для любого $P(x) \in \mathbb{F}[x]$, $P(x) = a_nx^n + \ldots + a_1x + a_0$ и для любого $\phi \in L(V)$ определен линейный оператор $P(\phi)\in L(V)$: $P(\phi) = a_n \phi^n + \ldots + a_1\phi + a_0\id$.

\begin{Def}
Вектор $v \in V$ называется корневым вектором линейного оператора $\phi$, отвечающим значению $\l \in F$, если существует $m \geqslant 0$ такое, что $(\phi - \l\id)^m(v) = 0$.

Наименьшее такое $m$ называют высотой корневого вектора $v$.
\end{Def}

\begin{Comment}\
\begin{enumerate}
\item Вектор $v = 0$ для любого $\phi$ имеет высоту 0.
\item Высоту 1 имеют все собственные векторы.
\end{enumerate}
\end{Comment}

\begin{Examples}
$V = \mathbb{F}[x]_{\leqslant n}$, $\Delta: f \rightarrow f'$. Здесь $\l = 0$ --- единственное собственное значение. Все векторы --- корневые, отвечающие $\l = 0$. 
\end{Examples}

\begin{Def}
Множество $V^\l(\phi) = \{ v \in V \mid \exists m \geqslant 0  : (\phi - \l\id)^m(v) = 0 \}$ называется корневым пространством для $\l \in F$.
\end{Def}

\begin{Task}
$V^\l(\phi)$ --- подпространство в $V$.
\end{Task}

\begin{Comment}
$V_\l(\phi) \subseteq V^\l(\phi) \ \forall \l \in F$.
\end{Comment}

\end{document}