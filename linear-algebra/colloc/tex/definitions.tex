\documentclass[a4paper,12pt]{article}

%% Работа с русским языком
\usepackage{cmap}					% поиск в PDF
\usepackage{mathtext} 				% русские буквы в формулах
\usepackage[T2A]{fontenc}			% кодировка
\usepackage[utf8]{inputenc}			% кодировка исходного текста
\usepackage[english,russian]{babel}	% локализация и переносы

%% Отступы между абзацами и в начале абзаца 
\setlength{\parindent}{0pt}
\setlength{\parskip}{\medskipamount}

%% Изменяем размер полей
\usepackage[top=0.5in, bottom=0.75in, left=0.625in, right=0.625in]{geometry}

%% Графика
\usepackage[pdftex]{graphicx}
\graphicspath{{images/}}

%% Различные пакеты для работы с математикой
\usepackage{mathtools}				% Тот же amsmath, только с некоторыми поправками

\usepackage{amssymb}				% Математические символы
\usepackage{amsthm}					% Пакет для написания теорем
\usepackage{amstext}
\usepackage{array}
\usepackage{amsfonts}
\usepackage{icomma}					% "Умная" запятая: $0,2$ --- число, $0, 2$ --- перечисление
\usepackage{bbm}				    % Для красивого (!) \mathbb с  буквами и цифрами
\usepackage{enumitem}               % Для выравнивания itemise (\begin{itemize}[align=left])

% Номера формул
\mathtoolsset{showonlyrefs=true} % Показывать номера только у тех формул, на которые есть \eqref{} в тексте.

% Ссылки
\usepackage[colorlinks=true, urlcolor=blue]{hyperref}

% Шрифты
\usepackage{euscript}	 % Шрифт Евклид
\usepackage{mathrsfs}	 % Красивый матшрифт

% Свои команды\textbf{}
\DeclareMathOperator{\sgn}{\mathop{sgn}}

% Перенос знаков в формулах (по Львовскому)
\newcommand*{\hm}[1]{#1\nobreak\discretionary{}
{\hbox{$\mathsurround=0pt #1$}}{}}

% Графики
\usepackage{tikz}
\usepackage{pgfplots}
%\pgfplotsset{compat=1.12}

% Изменим формат \section и \subsection:
\usepackage{titlesec}
\titleformat{\section}
{\vspace{1cm}\centering\LARGE\bfseries}	% Стиль заголовка
{}										% префикс
{0pt}									% Расстояние между префиксом и заголовком
{} 										% Как отображается префикс
\titleformat{\subsection}				% Аналогично для \subsection
{\Large\bfseries}
{}
{0pt}
{}

% Информация об авторах
\author{Группа лектория ФКН ПМИ 2015-2016 \\
	Анастасия Иовлева \\
	Ксюша Закирова \\
	Руслан Хайдуров}
\title{Лекции по предмету \\
	\textbf{Линейная алгебра и геометрия}}
\date{2016 год}

\newtheorem*{Def}{Определение}
\newtheorem*{Lemma}{Лемма}
\newtheorem*{Suggestion}{Предложение}
\newtheorem*{Examples}{Пример}
\newtheorem*{Comment}{Замечание}
\newtheorem*{Consequence}{Следствие}
\newtheorem*{Theorem}{Теорема}
\newtheorem*{Statement}{Утверждение}
\newtheorem*{Task}{Упражнение}
\newtheorem*{Designation}{Обозначение}
\newtheorem*{Generalization}{Обобщение}
\newtheorem*{Thedream}{Предел мечтаний}
\newtheorem*{Properties}{Свойства}

\renewcommand{\mathbb}{\mathbbm}
\renewcommand{\Re}{\mathrm{Re\:}}
\renewcommand{\Im}{\mathrm{Im\:}}
\newcommand{\Arg}{\mathrm{Arg\:}}
\renewcommand{\arg}{\mathrm{arg\:}}
\newcommand{\Mat}{\mathrm{Mat}}
\newcommand{\id}{\mathrm{id}}
\newcommand{\isom}{\xrightarrow{\sim}} 
\newcommand{\leftisom}{\xleftarrow{\sim}}
\newcommand{\Hom}{\mathrm{Hom}}
\newcommand{\Ker}{\mathrm{Ker}\:}
\newcommand{\rk}{\mathrm{rk}\:}
\newcommand{\diag}{\mathrm{diag}}
\newcommand{\ort}{\mathrm{ort}}
\newcommand{\pr}{\mathrm{pr}}
\newcommand{\vol}{\mathrm{vol\:}}

\renewcommand{\epsilon}{\varepsilon}
\renewcommand{\phi}{\varphi}
\newcommand{\e}{\mathbb{e}}
\renewcommand{\l}{\lambda}
\renewcommand{\C}{\mathbb{C}}
\newcommand{\R}{\mathbb{R}}
\newcommand{\E}{\mathbb{E}}

\newcommand{\vvector}[1]{\begin{pmatrix}{#1}_1 \\\vdots\\{#1}_n\end{pmatrix}}
\renewcommand{\vector}[1]{({#1}_1, \ldots, {#1}_n)}

\title{Линейная Алгебра и Геометрия}
\author{Лекторий ПМИ ФКН}
\date{3-4 июня 2016}

\begin{document}
\maketitle

\section{Определения}

\begin{enumerate}
\item \textbf{Алгебраическая форма комплексного числа. Сложение, умножение и деление комплексных чисел в алгебраической форме.}

Запись $z = a + bi$, где $a, b \in \R$, называется алгебраической формой комплексного числа $z \in \mathbb{C}$.

$a = \Re z$ — действительная часть числа $z$.

$b = \Im z$ — мнимая часть числа $z$.

Сложение: $$ (a + bi) + (c + di) = (a + c) + (b + d)i.$$

Умножение: $$(a + bi)(c + di) = ac + adi + bci + bdi^2 = (ac - bd) + (ad + bc)i.$$

Деление: $$\frac{a + bi}{c + di} = \frac{(a + bi)(c - di)}{c^2 + d^2} = \frac{ac + bd}{c^2 + d^2} + \frac{bc - ad}{c^2 + d^2}i, \quad (c + di) \neq 0.$$

В делении используется формула обратного элемента:
$$
\frac{1}{a + bi} = \frac{\overline{a + bi}}{|a + bi|^2} = \frac{a - bi}{a^2 + b^2}.
$$

\item \textbf{Комплексное сопряжение и его свойства: сопряжение суммы и произведения комплексных чисел.}

Отображение $\mathbb{C} \rightarrow \mathbb{C} : a + bi \mapsto a - bi$ называется (комплексным) сопряжением. Само число $\overline{z} = a - bi$ называется (комплексно) сопряженным к числу $z = a + bi$. 

Для любых двух комплексных чисел $z, w \in \mathbb{C}$ выполняется, что
\begin{enumerate}
\item $\overline{z + w} = \overline{z} + \overline{w}$;
\item $\overline{zw} = \overline{z} \cdot \overline{w}$.
\end{enumerate}

\item \textbf{Геометрическая модель комплексных чисел, интерпретация в ней сложения и сопряжения.}

Заметим, что поле комплексных чисел $\mathbb{C} = \{(a, b) \mid a, b \in \mathbb{R}\}$ равно $\mathbb{R}^2$. Следовательно, комплексные числа можно представить как точки на действительной плоскости $\mathbb{R}^2$, или сопоставить их векторам.

В таком представлении сложение комплексных чисел интерпретируется как сложение векторов, а сопряжение — как отражение относительно оси $Ox (\Re z)$.

\item \textbf{Модуль комплексного числа и его свойства: неотрицательность, неравенство треугольника, модуль произведения двух комплексных чисел.}

Модулем комплексного числа $z = a + bi$ называется длина соответствующего вектора. Обозначение: $|z|; |a+bi| = \sqrt{a^2 + b^2}$.

Свойства модуля:
\begin{enumerate}
\item $|z| \geqslant 0$, причем $|z| = 0$ тогда и только тогда, когда $z = 0$;
\item $|z + w| \leqslant |z| + |w|$ — неравенство треугольника;
\item $z\cdot\overline{z} = |z|^2$;
\item $|zw| = |z| \cdot |w|$;
\end{enumerate}
\item \textbf{Аргумент комплексного числа.}

Аргументом комплексного числа $z \neq 0$  называется всякий угол $\varphi$ такой что 
\[
\cos \varphi = \frac{a}{|z|} = \frac{a}{\sqrt{a^2 + b^2}}; \quad \sin \varphi = \frac{b}{|z|} = \frac{b}{\sqrt{a^2 + b^2}}.
\]
\item \textbf{Тригонометрическая форма комплексного числа.}

Используя аргумент, можно представить комплексное число следующим образом:
\[
\left.
\begin{aligned}
&a = |z|\cos \varphi \\
&b = |z|\sin \varphi
\end{aligned} 
\right| \Rightarrow z = a + bi = |z| \cos \varphi + i |z| \sin \varphi = |z|(\cos\varphi + i\sin\varphi)
\] 

Запись $z = |z|(\cos\varphi + i\sin\varphi)$ называется тригонометрической формой \\ комплексного числа $z$.

\item \textbf{Формула Муавра.}

Пусть $z = |z|\left(\cos\varphi + i \sin \varphi\right)$. Тогда:
\[z^n = |z|^n\left(\cos\left(n\varphi\right)+i\sin\left(n\varphi\right)\right) \quad \forall n \in \mathbb{Z}.
\]

\item \textbf{Извлечение корней из комплексных чисел.}

Пусть $n\in\mathbb N$ и $n\geqslant2$.

Корнем $n$-й степени из числа $z$ называется всякое $w\in\mathbb C$: $w^n=z$. То есть
\[
\sqrt[n]{z} = \{w\in\mathbb C\ |\ w^n = z\}.
\]

Представим $z$ и $w$ в тригонометрическом виде:
\begin{gather*}
z = |z|\left(\cos\varphi+i\sin\varphi\right), \quad w = |w|\left(\cos\psi+i\sin\psi\right)
\end{gather*}

Если $z=0$, то $w=0$. В противном случае, $z$ имеет ровно $n$ корней $n$-й степени:

\[ \sqrt[n]{z} = \Biggl\{\sqrt[n]{|z|}\left(\cos\cfrac{\varphi+2\pi k}{n}+i\sin\cfrac{\varphi+2\pi k }{n}\right)\ \biggl|\ k=0,\ldots,n-1\Biggr\}.
\]

\item \textbf{Решение квадратных уравнений с комплексными коэффициентами.}

Пусть дано квадратное уравнение $az^2+bz+c=0$, где $a,\ b,\ c\in\mathbb{C}$ и 	$ a \neq 0$. Тогда оно решается аналогично квадратным уравнениям над полем $\R$, с тем лишь отличием, что из дискриминанта всегда можно извлечь корень.
\begin{gather*}
\{d_1,d_2\} = \sqrt[2]{b^2-4ac} \\
z_1 = \cfrac{-b+d_1}{2a},\ z_2 = \cfrac{-b+d_2}{2a}
\end{gather*}

\item \textbf{Основная теорема алгебры комплексных чисел.}

Всякий многочлен $P\left(z\right) = a_nz^n + a_{n-1}z^{n-1} + \ldots \hm{+} a_1z + a_0$ степени $n$, где $n \geqslant 1$, $a_n \neq 0$, и $a_0,\ldots,a_n \in \mathbb{C}$ имеет корень.

\item \textbf{Овеществление комплексного векторного пространства и его размерность.}

$V$ --- векторное пространство над $\mathbb{C}$. Овеществление пространства $V$ --- это то же пространство $V$, рассматриваемое как пространство над $\mathbb{R}$. Обозначение: $V_\mathbb{R}$.

Пусть $\dim V < \infty$. Тогда $\dim V_\mathbb{R} = 2\dim V$.

\item \textbf{Комплексификация вещественного векторного пространства и его размерность.}

Пусть $W$ --- пространство над $\mathbb{R}$. Комплексификация пространства $W$ --- это множество $W\times W = W^\mathbb{C} \hm{=} \{\left( u, v\right)\ |\ u,v \in W\}$ с операциями $\left(u_1, v_1\right) + \left(u_2, v_2\right) = \left(u_1+u_2, v_1+v_2\right)$, $\left(a +  bi\right)\left(u, v\right) \hm{=} \left(au-bv, av+bu\right)$, где $(a + bi) \in \C$.

Рутинная проверка показывает, что $W^\C$ является векторным пространством над полем~$\C$, причем $\dim W^\mathbb{C} = \dim W$.

\item \textbf{Сумма двух подпространств векторного пространства.}

Пусть $V$ --- конечномерное векторное пространство, а $U$ и $W$ --- его подпространства.

Сумма подпространств $U$ и $W$ --- это множество
\[
U+W = \{u + w\ |\ u \in U, w \in W\},
\]
которое является подпространством векторного пространства $V$.

\item \textbf{Теорема о связи размерности суммы двух подпространств с размерностью их пересечения.}

Пусть $V$ --- конечномерное векторное пространство, а $U$ и $W$ --- его подпространства.

$\dim \left(U \cap W\right) = \dim U + \dim W - \dim \left(U+W\right)$

\item \textbf{Прямая сумма двух подпространств векторного пространства.}
Пусть $V$ --- конечномерное векторное пространство, а $U$ и $W$ --- подпространства.

Если $U \cap W = \{0\}$, то $U + W$ называется прямой суммой.

\item \textbf{Матрица перехода от одного базиса векторного пространства к другому.}

Пусть $V$ --- векторное пространство, $\dim V = n$, $\e = (e_1, \ldots, e_n)$ и $\e' =(e_1', \ldots, e_n')$ --- базисы в $V$.

Матрицей перехода от базисе $\e$ к базису $\e'$ называется матрица, по столбцам которой стоят координаты базиса $\e'$ в базисе $\e$.
\begin{gather*}
e'_j = \sum_{i = 1}^{n} c_{ij}e_i, \quad c_{ij} \in F \\
(e'_1, \ldots, e'_n) = (e_1, \ldots, e_n) \cdot C, \quad C = (c_{ij}) \text{--- матрица перехода}
\end{gather*}

\item \textbf{Формула преобразования координат вектора при замене базиса векторного пространства.}

Пусть $V$ --- векторное пространство. Формула преобразования координат вектора $v \in V$ при переходе от базиса $\e$ к $\e'$:
\begin{gather*}
\begin{pmatrix*}
x_1 \\
\vdots \\
x_n
\end{pmatrix*}
= C 
\begin{pmatrix*}
x'_1 \\
\vdots \\
x'_n
\end{pmatrix*}
\qquad \text{или} \qquad
x_i = \sum_{j = 1}^{n}c_{ij}x'_j,
\end{gather*}
где $(x_1, \ldots, x_n)$ --- координаты вектора $v$ в базисе $\e $, $(x_1', \ldots, x_n')$ --- координаты вектора $v$  в базисе $\e '$ и $C$ --- матрица перехода от базиса $\e $ к базису $\e '$.

\item \textbf{Линейное отображение.}

Пусть $V$ и $W$ --- два векторных пространства над полем $F$.

Отображение $f : V \rightarrow W$ называется линейным, если:
\begin{enumerate}
\item $f(u_1 + u_2) = f(u_1) + f(u_2), \quad \forall u_1, u_2 \in V$;
\item $f(\alpha u) = \alpha f(u), \quad \forall u \in V,\ \forall \alpha \in F$.
\end{enumerate}

\item \textbf{Изоморфизм векторных пространств. Изоморфные векторные пространства.}

Пусть $V$ и $W$ --- два векторных пространства над полем $F$.

Отображение $\phi: V \rightarrow W$ называется изоморфизмом, если $\phi$ линейно и биективно. Обозначение: $\phi : V \isom  W$.

Два векторных пространства $V$ и $W$ называются изоморфными, если существует изоморфизм $\phi: V \isom W$ (и тогда существует изоморфизм $V \leftisom W$). Обозначение: $V \simeq W$ или $V \cong W$.

\item \textbf{Критерий изоморфности двух конечномерных векторных пространств.}

Два конечномерных векторных пространства $V$ и $W$ изоморфны тогда и только тогда, когда $\dim V \hm= \dim W$.

\item \textbf{Матрица линейного отображения.}

Пусть $V$ и $W$ --- векторные пространства, $\mathbb{e} = (e_1, \ldots, e_n)$ --- базис $V$, $\mathbb{f} = (f_1, \ldots, f_m)$ --- базис $W$, $\phi: V \rightarrow W$ --- линейное отображение.

Матрицей линейного отображения $\phi$ в базисах $\e$ и $\mathbb{f}$ (или по отношению к базисам $\mathbb{e}$ и~$\mathbb{f}$) называется такая матрица, у которой в $j$-ом столбце выписаны координаты вектора $\phi(e_j)$ в базисе $\mathbb{f}$.
\[
\phi(e_j) = a_{1j}f_1 + \ldots + a_{mj}f_m = \sum_{i = 1}^{m}a_{ij}f_i, \quad A = (a_{ij}) \in \Mat_{m\times n} \text{ --- матрица $\phi$}
\]

\item \textbf{Сумма двух линейных отображений и её матрица.}

Пусть $\phi, \psi \in \Hom(V, W)$.

Отображение $\phi + \psi \in \Hom(V, W)$ --- это $(\phi + \psi)(v):= \phi(v) + \psi(v)$ -- сумма отображений.

Пусть $\mathbb{e} = (e_1, \ldots, e_n)$ --- базис $V$, $\mathbb{f} = (f_1, \ldots, f_m)$ --- базис $W$, $\phi, \psi \in \Hom(V, W)$. При этом $A_{\phi}$ --- матрица линейного отображения $\phi$, $A_{\psi}$ --- матрица для $\psi$, $A_{\phi+\psi}$ --- для $\phi + \psi$.

Тогда $A_{\phi+\psi} = A_{\phi} + A_{\psi}$.

\item \textbf{Произведение линейного отображения на скаляр и его матрица.}

Пусть $\phi, \psi \in \Hom(V, W)$.

Отображение $\alpha \in F, \alpha\phi \in \Hom(V, W)$ --- это $(\alpha\phi)(v) := \alpha(\phi(v))$ -- произведение линейного отображения на скаляр.

Пусть $\mathbb{e} = (e_1, \ldots, e_n)$ --- базис $V$, $\mathbb{f} = (f_1, \ldots, f_m)$ --- базис $W$, $\phi, \psi \in \Hom(V, W)$. При этом $A_{\phi}$ --- матрица линейного отображения $\phi$, $A_{\psi}$ --- матрица для $\psi$,  $A_{\alpha\phi}$ --- для $\alpha\phi$.

Тогда $A_{\alpha\phi} = \alpha A_{\phi}$.

\item \textbf{Композиция линейных отображений и её матрица.}

Возьмем три векторных пространства --- $U, V$ и $W$ размерности $n, m$ и $k$ соответственно, и их базисы $\mathbb{e}, \mathbb{f}$ и $\mathbb{g}$. Также рассмотрим цепочку линейных отображений $U \xrightarrow{\psi} V \xrightarrow{\phi} W$. 

Отображение $\phi\circ\psi \in \Hom(U, W)$ -- это $(\phi\circ\psi)(v) := \phi(\psi(v))$ -- композиция линейных отображений.

Пусть $A$ --- матрица $\phi$ в базисах $\mathbb{f}$ и $\mathbb{g}$, $B$ --- матрица $\psi$ в базисах $\mathbb{e}$ и $\mathbb{f}$, $C$ --- матрица $\phi\circ\psi$ в базисах $\mathbb{e}$ и $\mathbb{g}$.

Тогда $C = AB$.

\item \textbf{Ядро и образ линейного отображения.}

Пусть $V$ и $W$ --- векторные пространства, $\phi: V \rightarrow W$ --- линейное отображение.

\textit{Ядро $\phi$} --- это множество $\Ker\phi := \{v \in V \mid \phi(v) = 0 \}$.

\textit{Образ $\phi$} --- это множество $\Im \phi := \{w \in W \mid \exists v \in V : \phi(v) = w \}$.

\item \textbf{Критерий инъективности линейного отображения в терминах его ядра.}

Пусть $\phi\colon V \rightarrow W$ --- линейное отображение.

Отображение $\phi$ инъективно тогда и только тогда, когда $\Ker \phi = \{0\}$.

\item \textbf{Связь между рангом матрицы линейного отображения и размерностью его образа.}

Пусть $V, W$ --- векторные пространства, $\mathbb{e} = (e_1, \ldots, e_n)$ --- базис $V$, $\mathbb{f} = (f_1, \ldots, f_m)$ --- базис $W$, $A$ --- матрица $\phi$ по отношению к $\mathbb{e},\ \mathbb{f}$.

Тогда $\dim \Im \phi = \rk A$.

\item \textbf{Критерий изоморфности линейного отображения в терминах его матрицы.}

Пусть $V$ и $W$ --- векторные пространства, $\phi: V \rightarrow W$ --- линейное отображение.

Отображение $\phi$ является изоморфизмом тогда и только тогда, когда его матрица является квадратной и невырожденной.

\item \textbf{Ранг произведения двух матриц.}

Пусть $A \in \Mat_{k\times m},\ B\in \Mat_{m\times n}$. Тогда $\rk AB \leqslant \min\{\rk A, \rk B\}$.

\item \textbf{Теорема о связи размерностей ядра и образа линейного отображения.}

Пусть $\phi\colon V \rightarrow W$ --- линейное отображение.

Тогда $\dim \Im \phi = \dim V - \dim \Ker \phi$.

\item \textbf{Линейный оператор.}

Пусть $V$ --- конечномерное векторное пространство.

Линейным оператором (или линейным преобразованием) называется всякое линейное отображение $\phi \colon V \rightarrow V$, то есть из $V$ в себя.

\item \textbf{Матрица линейного оператора.}

Пусть $V$ --- векторное пространство, $\mathbb{e} = (e_1, \ldots, e_n)$ --- его базис и $\phi$ --- его линейный оператор.

Матрицей линейного оператора $\phi$ называется такая матрица, в $j$-ом столбце которой стоят координаты вектора $\phi(e_j)$ в базисе $\e$.
$$
\left(\phi(e_1), \ldots, \phi(e_n)\right) = \left(e_1, \ldots, e_n\right)A, \quad A \text{ --- матрица $\phi$}
$$

\item \textbf{Формула изменения матрицы линейного оператора при переходе к другому базису.}

Пусть $\phi$ --- линейный оператор векторного пространства $V$, $A$ --- матрица $\phi$ в базисе $\mathbb{e} = (e_1, \ldots, e_n)$. Пусть $\mathbb{e}' = (e_1', \ldots, e_n')$ --- другой базис, причём $(e_1', \ldots, e_n') = (e_1, \ldots, e_n)C$, где $C$ --- матрица перехода, и $A'$ --- матрица $\phi$ в базисе $\mathbb{e}'$.

Тогда $A' = C^{-1}AC$.

\item \textbf{Подобные матрицы.}

Две матрицы $A', A \in M_n(F)$ называются подобными, если существует такая матрица $C \in M_n(F), \det C \neq 0$, что $A' = C^{-1}AC$.

\item \textbf{Подпространство, инвариантное относительно линейного оператора.}

Пусть $\phi\colon V \rightarrow V$ --- линейный оператор.

Подпространство $U \subseteq V$ называется инвариантным относительно $\phi$ (или $\phi$-инвариантным), если $\phi(U)\subseteq U$. То есть $\forall u\in U \colon \phi(u)\in U$. 

\item \textbf{Матрица линейного оператора в базисе, дополняющем базис инвариантного подпространства.}

Пусть $\phi\colon V \rightarrow V$ --- линейный оператор.

Пусть $U\subset V$ --- $\phi$-инвариантное подпространство. Также пусть $(e_1, \ldots, e_k)$ --- базис в $U$. Дополним его до базиса $V\colon$ $\mathbb{e} = (e_1, \ldots, e_n)$. Тогда
\begin{gather}
    \underbrace{A(\phi,\;\mathbb{e})}_{\text{Матрица с углом нулей}} = \begin{pmatrix}
    B& C \\
    0& D
    \end{pmatrix}, \quad\text{где $B\in M_k$}
\end{gather}

\item \textbf{Собственный вектор линейного оператора.}

Пусть $\phi\colon V \rightarrow V$ --- линейный оператор.

Ненулевой вектор $v\in V$ называется \textit{собственным} для $V$, если $\phi(v) = \lambda v$ для некоторго $\lambda \in F$.

\item \textbf{Собственное значение линейного оператора.}

Элемент $\lambda \in F$ называется собственным значением линейного оператора $\phi$ векторно пространства $V$, если существует такой ненулевой вектор $v \in V$, что $\phi(v) = \l v$. 

\item \textbf{Собственное подпространство линейного оператора.}

Пусть $\phi\colon V \rightarrow V$ --- линейный оператор.

Множество $V_{\lambda}(\phi) = \{v\in V\ |\ \phi(v) = \lambda v\}$ называется собственным подпространством линейного оператора, отвечающим собственному значению $\lambda$.

\item \textbf{Диагонализуемый линейный оператор.}

Линейный оператор $\phi\colon V \rightarrow V$ называется диагонализуемым, если существует базис $\mathbb{e}$ в $V$ такой, что $A(\phi, \mathbb{e})$ диагональна. 

\item \textbf{Критерий диагонализуемости линейного оператора в терминах собственных векторов.}

Линейный оператор $\phi\colon V \rightarrow V$ диагонализуем тогда и только тогда, когда в $V$ существует базис из собственных векторов.

\item \textbf{Характеристический многочлен линейного оператора.}

Пусть $\phi\colon V \rightarrow V$ --- линейный оператор.

Многочлен $\chi_{\phi}(t) = (-1)^n\det(\phi - t \id)$ называется характеристическим для линейного оператора $\phi$.

\item \textbf{Связь собственных значений линейного оператора с его характеристическим многочленом.}

$\lambda$ --- собственное значение линейного оператора $\phi$ тогда и только тогда, когда $\chi_\phi(\lambda) = 0$. 

\item \textbf{Алгебраическая кратность собственного значения линейного оператора.}

Алгебраической кратностью собственного значения $\l$ линейного оператора $\phi \colon V \to V$ называется число $k$, которое равно кратности $\lambda$ как корня характеристического многочлена~$\phi$.

\item \textbf{Геометрическая кратность собственного значения линейного оператора.}

Пусть $\phi\colon V \rightarrow V$ --- линейный оператор, $\lambda$ --- его собственное значение и $V_\lambda(\phi)$ --- соответствующее собственное подпространство.

Геометрической кратностью собственного значения $\lambda$ называется число $\dim V_\lambda(\phi)$.

\item \textbf{Связь между алгебраической и геометрической кратностями собственного значения линейного оператора.}

Геометрическая кратность не больше алгебраической кратности.

\item \textbf{Сумма нескольких подпространств векторных пространств.}

Пусть $U_1, \ldots, U_k$ --- подпространства векторного пространства $V$. Суммой нескольких пространств называется $U_1 + \ldots + U_k = \{u_1 + \ldots + u_k \; | \; u_i \in U_i \}$.

\item \textbf{Прямая сумма нескольких подпространств векторных пространств.}

Пусть $U_1, \ldots, U_k$ --- подпространства векторного пространства $V$.

Сумма нескольких подпространств $U_1 + \ldots + U_k$ называется прямой, если из условия $\:u_1 + \ldots + u_k = 0$, где $u_i \in U_i$, следует, что $u_1 = \ldots = u_k = 0$.

\item \textbf{Эквивалентные условия, определяющие прямую сумму нескольких подпространств векторного пространства.}

Пусть $U_1, \ldots, U_k$ --- подпространства векторного пространства $V$.

Следующие условия эквивалентны:
	\begin{enumerate}
		\item Сумма $U_1 + \ldots + U_k$ --- прямая;
		\item Если $\mathbb{e}_i$ --- базис $U_i$, то $\mathbb{e} = \mathbb{e}_1 \cup \ldots \cup \mathbb{e}_k$ --- базис $U_1 + \ldots + U_k;$
		\item $\dim(U_1 + \ldots + U_k) = \dim{U_1} + \ldots + \dim{U_k}.$
	\end{enumerate}
	
\item \textbf{Сумма собственных подпространств линейного оператора, отвечающих попарно различным собственным значениям.}

Пусть $V$ --- векторное пространство над полем $F$, $\phi$ его линейный оператор, $\lambda_1, \ldots, \lambda_k$ --- набор собственных значений $\phi$, где $\lambda_i \neq \lambda_j$ при $i \neq j$, и $V_{\lambda_i}(\phi) \subseteq V$ --- соответствующее собственное подпространство.

Тогда сумма $V_{\lambda_1}(\phi) + \ldots + V_{\lambda_k}(\phi)$ является прямой.

\item \textbf{Критерий диагонализуемости линейного оператора в терминах его характеристического многочлена и кратностей собственных значений.}

Пусть $\phi\colon V \rightarrow V$ --- линейный оператор.

Линейный оператор $\phi$ диагонализируем тогда и только тогда, когда 
\begin{enumerate}
\item $\chi_\phi(t)$ разлагается на линейные множители;
\item Если $\chi_\phi(t) = (t - \l_1)^{k_1}\dots(t - \l_s)^{k_s}$, где $\l_i \neq \l_j$ при $i \neq j$, то $\dim V_{\l_i}(\phi) = k_i \ \forall i$ (то есть для любого собственного значения $\phi$ равны геометрическая и алгебраическая кратности).
\end{enumerate}

\item \textbf{Корневой вектор линейного оператора.}

Пусть $\phi\colon V \rightarrow V$ --- линейный оператор.

Вектор $v \in V$ называется корневым вектором линейного оператора $\phi$, отвечающим значению $\l \in F$, если существует $m \geqslant 0$ такое, что $(\phi - \l\id)^m(v) = 0$.

Наименьшее такое $m$ называют высотой корневого вектора $v$.

\item \textbf{Корневое подпространство линейного оператора.}

Пусть $\phi\colon V \rightarrow V$ --- линейный оператор.

Множество $V^\l(\phi) = \{ v \in V \mid \exists m \geqslant 0  : (\phi - \l\id)^m(v) = 0 \}$ называется корневым пространством для $\l \in F$.

\item \textbf{Характеристический многочлен ограничения линейного оператора на корневое подпространство.}

Пусть $\phi\colon V \rightarrow V$ --- линейный оператор.

Будем обозначать как $\phi \; \vline_{W}$ ограничение линейного оператора на пространство $W$.

Характеристический многочлен линейного отображения $\phi \; \vline_{V^{\lambda}(\phi)}$ равен $(t - \lambda)^{k}$, где $k \hm= \dim V^\l (\phi)$.

\item \textbf{Размерность корневого подпространства линейного оператора.}

Пусть $\phi\colon V \rightarrow V$ --- линейный оператор.

Если $\lambda$ --- собственное значение $\phi$, то $\dim{V^{\lambda}(\phi)}$ равна алгебраической кратности $\lambda$.

\item \textbf{Сумма корневых подпространств линейного оператора, отвечающих попарно различным собственным значениям.}

Пусть $\phi\colon V \rightarrow V$ --- линейный оператор.

Если $\lambda_1, \ldots, \lambda_k$, где $\lambda_i \neq \lambda_j$ при $i \neq j$ --- собственные значения $\phi$, то сумма $V^{\lambda_1}(\phi) + \ldots \hm+ V^{\lambda_k}(\phi)$ --- прямая.

\item \textbf{Признак разложимости пространства в прямую сумму корневых подпространств линейного оператора.}

Пусть $\phi\colon V \rightarrow V$ --- линейный оператор.

Если характеристический многочлен $\chi_\phi(t)$ разлагается на линейные множители, причём $\chi_\phi(t) = (t - \lambda_1)^{k_1}\ldots(t - \lambda_s)^{k_s}$, то $V = \bigoplus_{i = 1}^s  V^{\lambda_i}(\phi)$.

\item \textbf{Жорданова клетка.}

Пусть $\lambda \in F$. \textbf{Жорданова клетка} порядка $n$, отвечающая значению $\lambda$ (с собственным значением $\l $) --- это матрица следующего вида:
\begin{gather*}
J_\l^n = 
\begin{pmatrix}
\l & 1 & 0 & \ldots & 0 & 0 \\
0 & \l & 1 & \ldots & 0 & 0 \\
0 & 0 & \l & \ddots & 0 & 0 \\
\vdots & \vdots & \vdots & \ddots & \ddots & \vdots \\
0 & 0 & 0 & \ldots & \l & 1 \\
0 & 0 & 0 & \ldots & 0 & \l
\end{pmatrix} \in M_n(F).
\end{gather*}

\item \textbf{Теорема о Жордановой нормальной форме линейного оператора.}

Пусть $V$ --- векторное пространство, $\phi$ --- линейный оператор.

Пусть $\chi_\phi(t)$ разлагается на линейные множители. Тогда существует базис $\e$ в $V$ такой, что 
\begin{gather*}
A(\phi, \e) = 
\begin{pmatrix*}
J_{\mu_1}^{n_1} & 0 & \ldots & 0 \\
0 & J_{\mu_2}^{n_2} & \ldots & 0 \\
\vdots & \vdots & \ddots & \vdots \\
0 & 0 & \ldots & J_{\mu_p}^{n_p}
\end{pmatrix*} \quad (*)
\end{gather*}
Кроме того, матрица $(*)$ определена однозначно с точностью до перестановки жордановых клеток.

Матрица $(*)$ называется жордановой нормальной формой линейного оператора.

\item \textbf{Линейная функция.}

Линейной функцией (формой, функционалом) на векторном пространстве $V$ называется всякое линейное отображение $\sigma \colon V \rightarrow F$.

\item \textbf{Двойственный (сопряжённый) базис пространства линейных функций.} 

Пусть $\e = (e_1, \ldots, e_n)$ --- базис $V$. Рассмотрим линейные формы $\epsilon_1, \ldots, \epsilon_n$ такие, что $\epsilon_i(e_j) =~\delta_{ij}$, где $\delta_{ij} =
\begin{cases}
1, & i = j \\
0, & i \neq j
\end{cases}
$ --- символ Кронекера. \\То есть $\epsilon_i = (\delta_{i1}, \ldots, \delta_{ii}, \ldots, \delta_{in}) = (0, \ldots, 1, \ldots, 0)$.

Тогда $(\epsilon_1, \ldots, \epsilon_n)$ --- базис в $V^*$, называющийся двойственным (сопряжённым) к базису $\e$.
\item \textbf{Билинейная функция.}

Билинейной функцией (формой) на векторном пространстве $V$ называется всякое билинейное отображение $\beta \colon V \times V \rightarrow F$. То есть это отображение, линейное по каждому аргументу:
\begin{enumerate}
\item $\beta(x_1 + x_2, y) = \beta(x_1, y) + \beta(x_2, y)$; 
\item $\beta(\l x, y) = \l\beta(x, y)$;
\item $\beta(x, y_1 + y_2) = \beta(x, y_1) + \beta(x, y_2)$;
\item $\beta(x, \l y) = \l\beta(x, y)$.
\end{enumerate}

\item \textbf{Матрица билинейной функции.}

Пусть $V$ --- векторное пространство, $\dim V < \infty$, $\beta \colon V \times V \rightarrow F$ --- билинейная функция.

Матрицей билинейной функции в базисе $\e = (e_1, \ldots, e_n)$ называется матрица $B = (b_{ij})$, где $b_{ij} \hm= \beta(e_i, e_j)$.

\item \textbf{Формула для вычисления значений билинейной функции в координатах.}

Пусть $(e_1, \ldots, e_n)$ --- базис $V$, $\beta \colon V \times V \to F$ --- билинейная функция, $B$ --- ее матрица в базисе $\e $. Тогда для некоторых векторов $x = x_1e_1 + \ldots + x_ne_n \in V$ и $y = y_1e_1 + \ldots + y_ne_n \in V$ верно, что:
\begin{gather*}
\beta(x, y) = (x_1, \ldots, x_n)B \vvector{y}
\end{gather*}

\item \textbf{Формула изменения матрицы билинейной функции при переходе к другому базису.} 

Пусть $\e = (e_1, \ldots, e_n)$ и $\e' = (e'_1, \ldots, e'_n)$ --- два базиса $V$, $\beta$ --- билинейная функция на $V$. Пусть также $\e' = \e C$, где $C$ --- матрица перехода, также $B(\beta, \e) = B$ и $B(\beta, \e') = B'$.

Тогда $B' = C^TBC$.

\item \textbf{Ранг билинейной функции.}

Пусть $B(\beta, \e)$ -- матрица билинейной функции $\beta$ в базисе $\e$.

Число $\rk B$ называется рангом билинейной функции $\beta$. Обозначение: $\rk \beta$.

\item \textbf{Симметричная билинейная функция.}

Билинейная функция $\beta$ называется симметричной, если $\beta(x, y) =\beta(y, x)$ для любых $x, y \in V$.

\item \textbf{Квадратичная форма.}

Пусть $\beta \colon V\times V \rightarrow F$ --- билинейная функция. Тогда функция $Q_\beta \colon V \rightarrow F$, заданная формулой $Q_\beta(x) = \beta(x, x)$, называется квадратичной формой (функцией), ассоциированной с билинейной функцией $\beta$.

\item \textbf{Соответствие между симметричными билинейными функциями и квадратичными формами.}

Пусть $\beta \colon V\times V \rightarrow F$ --- симметричная билинейная функция, где $F$ --- поле, в котором $0 \neq 2$ (то есть можно делить на два).

Отображение $\beta(x, y) \mapsto Q_\beta(x) = \beta(x, x)$ является биекцией между симметричными билинейными функциями на $V$ и квадратичными функциями на $V$.

\item \textbf{Матрица квадратичной формы.}

Пусть $V$ --- векторное пространство, $\dim V < \infty$.

Матрицей квадратичной формы $Q \colon V \to F$   в базисе $\e$ называется матрица  соответствующей ей симметричной билинейной функции $\beta \colon V \times V \rightarrow F$ в том же базисе. 
\end{enumerate}

\end{document}