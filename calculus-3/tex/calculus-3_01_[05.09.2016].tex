\documentclass[a4paper, 12pt]{article}
\usepackage{header}

\begin{document}
\pagestyle{fancy}
\section{Лекция 01 от 05.09.2016}
\begin{Def}
	Пусть \(\{a_n\}^{\infty}_{n=1}\) --- последовательность действительных чисел. \emph{Числовым рядом} называется выражение вида \(\sum\limits_{n=1}^{\infty}a_n\), записываемая также как \(a_1 + a_2 + \ldots + a_n + \ldots \)
	
\end{Def}
\begin{Def}
	\(N\)-й частичной суммой называется сумма первых \(N\) членов. \(S_n = a_1 + \ldots + a_N\).
\end{Def}

\begin{Def}
	Последовательность \(\{S_n\}^{\infty}_{n=1}\)
	называется последовательностью частичных сумм.
\end{Def}
\begin{Def}
	Говорят, что ряд сходится, если сходится его последовательность его частичных сумм.
\end{Def}
\begin{Def}
	Суммой ряда называется этот предел, если он существует.
\end{Def}
\begin{Def}
	А если предела нет, то говорят, что ряд расходится.
\end{Def}
Вспоминая, что $a_n = S_{n} - S_{n-1}$, можно заключить, что особой разницы между самим рядом и последовательностью его частичных сумм нет.

\begin{Examples}[Предел Коши для последовательностей]
	$\{S_n\}_{n=1}^\infty$ сходится тогда и только тогда, когда она удовлетворяет условию Коши, т.е.
	$$
	\forall \varepsilon>0\; \exists N\in \N\colon \forall m,k > N \Rightarrow |S_m - S_k|<\varepsilon
	$$
\end{Examples}
Нахаляву получили теорему.
\begin{Theorem}[Критерий Коши сходимости ряда]
	Для сходимости ряда $\sum\limits_{n=1}^{\infty} a_n$ необходимо и достаточно, чтобы 
	$$
	\forall \varepsilon>0\; \exists N\in \N\colon \forall k>N,\; \forall p\in \N |a_{k+1} + a_{k+2} \ldots + a_{k+p}| < \varepsilon
	$$
	
\end{Theorem}
Отсюда сразу же очевидно следует утверждение.
\begin{Statement}[Необходимое условие сходимости ряда]
	Если ряд сходится, то $\lim\limits_{n\to \infty} a_n = 0$.
\end{Statement}
\begin{proof}
	Ряд сходится, значит 
	$$
	\forall \varepsilon>0\; \exists N \in \N\colon \forall k>N, p = 1 \Rightarrow |a_{k+1}| < \varepsilon$$ --- определение предела, равного нулю.
	\par Другой способ доказательства:
	$$
	a_n = S_n - S_{n-1}$$
	А принимая во внимание, что $S_n$, как и $S_{n-1}$ стремятся к одному пределу при стремлении $n$ к бесконечности, получим, что $\lim\limits_{n \to \infty} a_n = 0$.
\end{proof}
Сформулируем и докажем несколько тривиальных свойств.
\begin{Theorem}
	Пусть $\sum\limits_{n = 1}^{\infty}a_n = a$, $\sum\limits_{n = 1}^{\infty}b_n = b$. Тогда $\sum\limits_{n = 1}^{\infty}\left(a_n + b_n\right) = a + b$.
			\end{Theorem}
			\begin{proof}
				Это напрямую следует из свойсва пределов $S^{a+b}_n = S_n^a + S_n^b$
			\end{proof}
Аналогично, вспоминая свойства предела последовательности, можно доказать, что, если $\sum\limits_{n = 1}^{\infty} a_n = a$, то $\sum\limits_{n = 1}^{\infty} \alpha a_n = \alpha a$ для любого действительного $\alpha$.

Введём важное определение.

\begin{Def}
	Пусть дан ряд $\sum\limits_{n=1}^{\infty}a_n$. Обозначим некоторые из его сумм
	$$
	\underbrace{a_1 + \ldots + a_{n_1}}_{b_1} + \underbrace{a_{n_1+1} + \ldots + a_{n_2}}_{b_2} + \ldots + a_{n_3} + a_{n_3 + 1} + \ldots
	 $$
	 где $\{b_j\}_{j=1}^{\infty}$ --- возрастающая последовательность натуральных чисел. В таком случае говорят, что ряд $\sum\limits_{j =1}^{\infty} b_j$ получен из исходного \emph{расстановкой скобок}.
\end{Def}
\begin{Statement}
	Если ряд сходится, то после любой расстановки скобок он сходится, скажем неформально, туда же.
\end{Statement}

\begin{proof}
	Достаточно заметить, что частичные суммы ряда, полученного расстановкой скобок, образуют подпоследовательость в последовательности частичных сумм исходного ряда. Вспоминая свойство предела последовательности, что любая подпоследовательность сходящейся последовательности сходится туда же, куда и сама последовательность.
\end{proof}
\emph{Обратное неверно!!!} Пример такого ряда 
$$
1 - 1 + 1 - \ldots = \sum\limits_{n = 0}^{\infty} \left(-1\right)^n
$$
который при расстановке скобок $(1 - 1) + (1 - 1) + \ldots = 0$ даёт сходящийся ряд, в то время как исходный ряд расходится (не выполняется необходимое условие о стремлении членов ряда 0). 
\begin{Statement}
	Если $a_n \to 0$ и длины скобок ограничены (т.е. существует такое $c \in \R$, что $n_{k+1} - n_{k} < c$ при всех $k$), то из сходимости исходного ряда следует сходимость исходного ряда
\end{Statement}
\begin{Statement}
	Изменение, удаление, добавление конечного числа членов ряда не влияют на сходимость.
\end{Statement}
\begin{Def}
	Если сходится ряд $\sum\limits_{n = 1}^{\infty}|a_n|$, то говорят, что ряд $\sum\limits_{n = 1}^{\infty}a_n$ сходится абсолютно.
\end{Def}
\begin{Def}
	Если ряд сходится, но не сходится абсолютно, то говорят, что ряд сходится условно.
\end{Def}
\begin{Statement}
	Если ряд $\sum\limits_{n=1}^{\infty}a_n$ сходится абслютно, то он сходится.
\end{Statement}
\begin{proof}
	Сразу следует из критерия Коши. Возьмём произвольное $\varepsilon>0$. Так как ряд из модулей сходится, то $$\exists N\in \N\colon \forall k>N,\; \forall p\in \N \Rightarrow \sum\limits_{k+1}^{k+p}|a_k| < \varepsilon$$
	Тогда $$\left| \sum\limits_{n=k+1}^{k+p}a_n\right| \leqslant \sum\limits_{n=k+1}^{k+p}|a_n| < \varepsilon$$
\end{proof}
Введём ещё парочку определений
\begin{Def}
	Для ряда $\sum \limits_{n=1}^{\infty}a_n$ $N$-й хвост это сумма $r_N = \sum \limits_{n=N}^{\infty}a_n$.
\end{Def}
Для сходящегося ряда очевидно, что $r_n \in \R$.
\end{document}