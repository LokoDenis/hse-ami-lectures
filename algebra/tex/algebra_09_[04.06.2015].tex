\documentclass[a4paper,10pt]{amsart}

\usepackage[T2A]{fontenc}
\usepackage[utf8x]{inputenc}
\usepackage{amssymb}
\usepackage[russian]{babel}
\usepackage{geometry}
\usepackage{hyperref}
\usepackage{enumitem}

\geometry{a4paper,top=2cm,bottom=2cm,left=2cm,right=2cm}

\setlength{\parindent}{0pt}
\setlength{\parskip}{\medskipamount}

\newcommand{\Ker}{\mathop{\mathrm{Ker}}}
\renewcommand{\Im}{\mathop{\mathrm{Im}}}
\DeclareMathOperator{\Tor}{\mathrm{Tor}}
\newcommand{\xar}{\mathop{\mathrm{char}}}

%\def\Ker{{\rm Ker}}%
%\def\Im{{\rm Im}}%
\def\Mat{{\rm Mat}}%
\def\cont{{\rm cont}}%
%\def\Tor{{\rm Tor}}%
\def\Char{{\rm Char}}%
\def\signum{{\rm sig}}%
\def\Sym{{\rm Sym}}%
\def\St{{\rm St}}%
\def\Aut{{\rm Aut}}%
\def\Chi{{\mathbb X}}%
\def\Tau{{\rm T}}%
\def\Rho{{\rm R}}%
\def\rk{{\rm rk}}%
\def\ggT{{\rm ggT}}%
\def\kgV{{\rm kgV}}%
\def\Div{{\rm Div}}%
\def\div{{\rm div}}%
\def\quot{/\!\!/}%
\def\mal{\! \cdot \!}%
\def\Of{{\mathcal{O}}}
%
\def\subgrpneq{\le}%
\def\subgrp{\le}%
\def\ideal#1{\le_{#1}}%
\def\submod#1{\le_{#1}}%
%
\def\Bild{{\rm Bild}}%
\def\Kern{{\rm Kern}}%
\def\bangle#1{{\langle #1 \rangle}}%
\def\rq#1{\widehat{#1}}%
\def\t#1{\widetilde{#1}}%
\def\b#1{\overline{#1}}%
%
\def\abs#1{{\vert #1 \vert}}%
\def\norm#1#2{{\Vert #1 \Vert}_{#2}}%
\def\PS#1#2{{\sum_{\nu=0}^{\infty} #1_{\nu} #2^{\nu}}}%
%
\def\C{{\rm C}}%
\def\O{{\rm O}}%
\def\HH{{\mathbb H}}%
\def\LL{{\mathbb L}}%
\def\FF{{\mathbb F}}%
\def\CC{{\mathbb C}}%
\def\KK{{\mathbb K}}%
\def\TT{{\mathbb T}}%
\def\ZZ{{\mathbb Z}}%
\def\RR{{\mathbb R}}%
\def\SS{{\mathbb S}}%
\def\NN{{\mathbb N}}%
\def\QQ{{\mathbb Q}}%
\def\PP{{\mathbb P}}%
\def\AA{{\mathbb A}}%
%
\def\eins{{\mathbf 1}}%
%
\def\AG{{\rm AG}}%
\def\Aut{{\rm Aut}}%
\def\Hol{{\rm Hol}}%
\def\GL{{\rm GL}}%
\def\SL{{\rm SL}}%
\def\SO{{\rm SO}}%
\def\Sp{{\rm Sp}}%
\def\gl{\mathfrak{gl}}%
\def\rg{{\rm rg}}%
\def\sl{\mathfrak{sl}}%
\def\HDiv{{\rm HDiv}}%
\def\CDiv{{\rm CDiv}}%
\def\Res{{\rm Res}}%
\def\Pst{{\rm Pst}}%
\def\Nst{{\rm Nst}}%
\def\rad{{\rm rad}}%
\def\GL{{\rm GL}}%
\def\Tr{{\rm Tr}}%
\def\Pic{{\rm Pic}}%
\def\Hom{{\rm Hom}}%
\def\hom{{\rm hom}}%
\def\Mor{{\rm Mor}}%
\def\codim{{\rm codim}}%
\def\Supp{{\rm Supp}}%
\def\Spec{{\rm Spec}}%
\def\Proj{{\rm Proj}}%
\def\Maps{{\rm Maps}}%
\def\cone{{\rm cone}}%
\def\ord{{\rm ord}}%
\def\pr{{\rm pr}}%
\def\id{{\rm id}}%
\def\mult{{\rm mult}}%
\def\inv{{\rm inv}}%
\def\neut{{\rm neut}}%
%
\def\AAA{\mathcal{A}}
\def\BBB{\mathcal{B}}
\def\CCC{\mathcal{C}}
\def\EEE{\mathcal{E}}
\def\FFF{\mathcal{F}}

\def\CF{{\rm CF}}
\def\GCD{{\rm GCD}}
\def\Mat{{\rm Mat}}
\def\End{{\rm End}}
\def\cont{{\rm cont}}
\def\Kegel{{\rm Kegel}}
\def\Char{{\rm Char}}
\def\Der{{\rm Der}}
\def\signum{{\rm sg}}
\def\grad{{\rm grad}}
\def\Spur{{\rm Spur}}
\def\Sym{{\rm Sym}}
\def\Alt{{\rm Alt}}
\def\Abb{{\rm Abb}}
\def\Chi{{\mathbb X}}
\def\Tau{{\rm T}}
\def\Rho{{\rm R}}
\def\ad{{\rm ad}}
\def\Frob{{\rm Frob}}
\def\Rang{{\rm Rang}}
\def\SpRang{{\rm SpRang}}
\def\ZRang{{\rm ZRang}}
\def\ggT{{\rm ggT}}
\def\kgV{{\rm kgV}}
\def\Div{{\rm Div}}
\def\div{{\rm div}}
\def\quot{/\!\!/}
\def\mal{\! \cdot \!}
\def\add{{\rm add}}
\def\mult{{\rm mult}}
\def\smult{{\rm smult}}

\def\subgrpneq{\le}
\def\subgrp{\le}
\def\ideal#1{\unlhd_{#1}}
\def\submod#1{\le_{#1}}

\def\Bild{{\rm Bild}}
\def\Kern{{\rm Kern}}
\def\Kon{{\rm Kon}}
\def\bangle#1{{\langle #1 \rangle}}
\def\rq#1{\widehat{#1}}
\def\t#1{\widetilde{#1}}
\def\b#1{\overline{#1}}

\def\abs#1{{\vert #1 \vert}}
\def\norm#1#2{{\Vert #1 \Vert}_{#2}}
\def\PS#1#2{{\sum_{\nu=0}^{\infty} #1_{\nu} #2^{\nu}}}


\def\eins{{\mathbf 1}}

\def\ElM{{\rm ElM}}
\def\ZOp{{\rm ZOp}}
\def\SpOp{{\rm SpOp}}
\def\Gal{{\rm Gal}}
\def\Def{{\rm Def}}
\def\Fix{{\rm Fix}}
\def\ord{{\rm ord}}
\def\Aut{{\rm Aut}}
\def\Hol{{\rm Hol}}
\def\GL{{\rm GL}}
\def\SL{{\rm SL}}
\def\SO{{\rm SO}}
\def\Sp{{\rm Sp}}
\def\Spann{{\rm Spann}}
\def\Lin{{\rm Lin}}
\def\gl{\mathfrak{gl}}
\def\rg{{\rm rg}}
\def\sl{\mathfrak{sl}}
\def\so{\mathfrak{so}}
\def\sp{\mathfrak{sp}}
\def\gg{\mathfrak{g}}
\def\HDiv{{\rm HDiv}}
\def\CDiv{{\rm CDiv}}
\def\Res{{\rm Res}}
\def\Pst{{\rm Pst}}
\def\Nst{{\rm Nst}}
\def\WDiv{{\rm WDiv}}
\def\GL{{\rm GL}}
\def\Tr{{\rm Tr}}
\def\Pic{{\rm Pic}}
\def\Hom{{\rm Hom}}
\def\hom{{\rm hom}}
\def\Mor{{\rm Mor}}
\def\codim{{\rm codim}}
\def\Supp{{\rm Supp}}
\def\Spec{{\rm Spec}}
\def\Proj{{\rm Proj}}
\def\Maps{{\rm Maps}}
\def\cone{{\rm cone}}
\def\ord{{\rm ord}}
\def\pr{{\rm pr}}
\def\id{{\rm id}}
\def\mult{{\rm mult}}
\def\inv{{\rm inv}}
\def\neut{{\rm neut}}
\def\trdeg{{\rm trdeg}}
\def\sing{{\rm sing}}
\def\reg{{\rm reg}}


%%%%%%%%%%%%%%%%%%%%%%%%%%%

\newtheorem{theorem}{Теорема}
\newtheorem{proposition}{Предложение}
\newtheorem{lemma}{Лемма}
\newtheorem{corollary}{Следствие}
\theoremstyle{definition}
\newtheorem{definition}{Определение}
\newtheorem{problem}{Задача}
%
\theoremstyle{remark}
\newtheorem{exercise}{Упражнение}
\newtheorem{remark}{Замечание}
\newtheorem{example}{Пример}

\renewcommand{\theenumi}{\textup{(\alph{enumi})}}
\renewcommand{\labelenumi}{\theenumi}
\newcounter{property}
\renewcommand{\theproperty}{\textup{(\arabic{property})}}
\newcommand{\property}{\refstepcounter{property}\item}
\newcounter{prooperty}
\renewcommand{\theprooperty}{\textup{(\arabic{prooperty})}}
\newcommand{\prooperty}{\refstepcounter{prooperty}\item}

\makeatletter
\def\keywords#1{{\def\@thefnmark{\relax}\@footnotetext{#1}}}
\let\subjclass\keywords
\makeatother
%
\begin{document}
%
\sloppy
%\thispagestyle{empty}
%
\centerline{\large \bf Лекции курса \guillemotleft
Алгебра\guillemotright{}, лекторы И.\,В.~Аржанцев и Р.\,С.~Авдеев}

\smallskip

\centerline{\large ФКН НИУ ВШЭ, 1-й курс ОП ПМИ, 4-й модуль,
2014/2015 учебный год}


\bigskip

\section*{Лекция~9}

\medskip

{\it Примеры полей. Характеристика поля. Расширения полей,
алгебраические и трансцендентные элементы. Минимальные многочлен.
Конечное расширение и его степень. Присоединение корня многочлена.
Поле разложения многочлена: существование и единственность.}

\medskip

Мы знаем не так много примеров полей. Это бесконечные поля $\QQ$,
$\RR$, $\CC$ и конечные поля $\ZZ_p$, где $p$~--- простое число.
Конструкция поля отношений позволяет строить новые поля из уже
имеющихся. А именно, если $K$~--- произвольное поле, то можно
рассмотреть поле отношений $K(x)$ кольца многочленов $K[x]$ (это
поле называется \textit{полем рациональных дробей} над~$K$).
Элементами поля $K(x)$ являются дроби $f(x)/g(x)$, где $f(x), g(x)
\in K[x]$ и $g(x) \ne 0$.

Несколько других примеров полей:
$$
\QQ(\sqrt{2}) = \{a + b \sqrt{2} \mid a, b \in \QQ\}, \quad
\QQ(\sqrt[3]{2}) = \{a + b \sqrt[3]{2} + c\sqrt[3]{4} \mid a, b,
c\in\QQ\}, \quad \QQ(\sqrt{-1}) = \{a + b\sqrt{-1} \mid a, b \in \QQ
\}.
$$

\begin{definition}
Пусть $K$~--- произвольное поле. {\it Характеристикой} поля $K$
называется такое наименьшее натуральное число $p$, что
$\underbrace{1+\ldots+1}_p = 0$. Если такого натурального $p$ не
существует, говорят, что характеристика поля равна нулю.
Обозначение: $\xar K$.
\end{definition}

Например, $\xar \QQ = \xar \RR = \xar \CC = 0$ и $\xar \ZZ_p = \xar
\ZZ_p(x) = p$.

Из определения следует, что всякое поле характеристики нуль
бесконечно. Примером бесконечного поля характеристики $p > 0$
является поле $\ZZ_p(x)$.

\begin{proposition}
Характеристика произвольного поля $K$ либо равна нулю, либо является
простым числом.
\end{proposition}

\begin{proof}
Положим $p = \xar K$ и предположим, что $p > 0$. Так как $0 \ne 1$
в~$K$, то $p \geqslant 2$. Если число $p$ не является простым, то $p
= mk$ для некоторых $m,k \in \NN$, $1 < m,k < p$. Тогда в $K$ верно
равенство
$$
0 = \underbrace{1 + \ldots + 1}_{mk} = (\underbrace{1 + \ldots +
1}_m)(\underbrace{1 + \ldots + 1}_k).
$$
В~силу минимальности числа~$p$ в последнем выражении обе скобки
отличны от нуля, но такое невозможно, так как в поле нет делителей
нуля.
\end{proof}

\begin{exercise}
Пересечение любого семейства подполей фиксированного поля~$K$
является подполем в~$K$. В частности, для всякого подмножества $S
\subseteq K$ существует наименьшее по включению подполе в~$K$,
содержащее~$S$. Это подполе совпадает с пересечением всех подполей
в~$K$, содержащих~$S$.
\end{exercise}

Из приведённого выше упражнения следует, что в каждом поле
существует наименьшее по включению подполе, оно называется {\it
простым подполем}.

\begin{proposition}
Пусть $K$~--- поле и $K_0$~--- его простое подполе. Тогда:

\textup{(1)} если $\xar K = p > 0$, то $K_0 \cong \ZZ_p$;

\textup{(2)} если $\xar K = 0$, то $K_0 \cong \QQ$.
\end{proposition}

\begin{proof}
Пусть $\langle 1 \rangle \subseteq K$~--- циклическая подгруппа по
сложению, порождённая единицей. Заметим, что $\langle 1 \rangle$~---
подкольцо в~$K$. Поскольку всякое подполе поля $K$ содержит единицу,
оно содержит и множество~$\langle 1 \rangle$. Следовательно,
$\langle 1 \rangle \subseteq K_0$.

Если $\xar K = p > 0$, то мы имеем изоморфизм колец $\langle 1
\rangle \simeq \ZZ_p$. Но, как мы уже знаем из лекции~6,
кольцо~$\ZZ_p$ является полем, поэтому $K_0 = \langle 1 \rangle
\simeq \ZZ_p$.

Если же $\xar K = 0$, то мы имеем изоморфизм колец $\langle 1
\rangle \cong \ZZ$. Тогда $K_0$ содержит все дроби вида $a/b$, где
$a,b \in \langle 1 \rangle$ и $b \ne 0$. Ясно, что все такие дроби
образуют поле, изоморфное полю~$\QQ$.
\end{proof}

\begin{definition}
Если $K$~--- подполе поля $F$, то говорят, что $F$~--- {\it
расширение} поля~$K$.
\end{definition}

Например, всякое поле есть расширение своего простого подполя.

\begin{definition}
{\it Степенью} расширения полей $K \subseteq F$ называется
размерность поля $F$ как векторного пространства над полем~$K$.
Обозначение $[F : K]$.
\end{definition}

Например, $[\CC : \RR] = 2$ и $[\RR : \QQ] = \infty$.

\begin{definition}
Расширение полей $K\subseteq F$ называется {\it конечным}, если $[F
: K] < \infty$.
\end{definition}

\begin{proposition}
Пусть $K\subseteq F$ и $F\subseteq L$~--- конечные расширения полей.
Тогда расширение $F\subseteq L$ также конечно и $[L:K]=[L:F][F:K]$.
\end{proposition}

\begin{proof}
Пусть $e_1,\ldots,e_n$~--- базис $F$ над $K$ и $f_1,\ldots,f_m$~---
базис $L$ над $F$. Достаточно доказать, что множество
\begin{equation} \label{eqn_basis}
\lbrace e_i f_j \mid i = 1,\ldots,n;\, j = 1,\ldots, m \rbrace
\end{equation}
является базисом $L$ над~$K$. Для этого сначала покажем, что
произвольный элемент $a\in L$ представим в виде линейной комбинации
элементов (\ref{eqn_basis}) с коэффициентами из~$K$. Поскольку $f_1,
\ldots, f_m$~--- базис $L$ над~$F$, имеем $a = \sum \limits_{j=1}^m
\alpha_j f_j$ для некоторых $\alpha_j \in F$. Далее, поскольку $e_1,
\ldots, e_n$~--- базис $F$ над~$K$, для каждого $j = 1, \ldots, m$
имеем $\alpha_j = \sum \limits_{i = 1}^n \beta_{ij} e_i$ для
некоторых $\beta_{ij}\in K$. Отсюда получаем, что $a = \sum
\limits_{i=1}^n \sum \limits_{j=1}^n \beta_{ij} (e_if_j)$.

Теперь проверим линейную независимость элементов~(\ref{eqn_basis}).
Пусть $\sum \limits_{i=1}^n \sum \limits_{j=1}^n \gamma_{ij}
(e_jf_i) = 0$, где $\gamma_{ij} \in K$. Переписав это равенство в
виде $\sum \limits_{j=1}^m (\sum \limits_{i=1}^n \gamma_{ij}e_i)f_j
= 0$ и воспользовавшись тем, что элементы $f_1, \ldots, f_m$ линейно
независимы над~$F$, мы получим $\sum \limits_{i=1}^n \gamma_{ij}e_i
= 0$ для каждого $j = 1, \ldots, m$. Теперь из линейной
независимости элементов $e_1, \ldots, e_n$ над~$K$ вытекает, что
$\gamma_{ij} = 0$ при всех $i,j$. Таким образом,
элементы~(\ref{eqn_basis}) линейно независимы.
\end{proof}

Пусть $K\subseteq F$~--- расширение полей.

\begin{definition}
Элемент $\alpha \in F$ называется {\it алгебраическим} над подполем
$K$, если существует ненулевой многочлен $f(x)\in K[x]$, для
которого $f(\alpha) = 0$. В~противном случае $\alpha$ называется
{\it трансцендентным} элементом над~$K$.
\end{definition}

\begin{definition}
{\it Минимальным многочленом} алгебраического элемента $\alpha \in
F$ над подполем $K$ называется ненулевой многочлен $h_\alpha(x)$
наименьшей степени, для которого $h_\alpha(\alpha) = 0$.
\end{definition}

\begin{lemma} \label{lemma_min_pol}
Пусть $\alpha \in F$~--- алгебраический элемент над~$K$ и
$h_\alpha(x)$~--- его минимальный многочлен. Тогда:

\textup{(а)} $h_\alpha(x)$ определён однозначно с точностью до
пропорциональности;

\textup{(б)} $h_\alpha(x)$ является неприводимым многочленом над
полем~$K$;

\textup{(в)} для произвольного многочлена $f(x)\in K[x]$ равенство
$f(\alpha)=0$ имеет место тогда и только тогда, когда $h_\alpha(x)$
делит~$f(x)$.
\end{lemma}

\begin{proof}
(а) Пусть $h'_\alpha(x)$~--- ещё один минимальный многочлен элемента
$\alpha$ над~$K$. Тогда $\deg h_\alpha(x) = \deg h'_\alpha(x)$.
Умножив многочлены $h_\alpha(x)$ и $h'_\alpha(x)$ на подходящие
константы, добьёмся того, чтобы их старшие коэффициенты стали равны
единице. После этого положим $g(x) = h_\alpha(x) - h'_\alpha(x)$.
Тогда $g(\alpha) = 0$ и $\deg g(x) < \deg h_\alpha(x)$. Учитывая
определение минимального многочлена, мы получаем $g(x) = 0$.

(б) Пусть $h_\alpha(x) = h_1(x) h_2(x)$ для некоторых $h_1(x),
h_2(x) \in K[x]$, причём $0 < \deg h_i(x) < \deg h_\alpha(x)$ при $i
= 1,2$. Так как $h_\alpha(\alpha) = 0$, то либо $h_1(\alpha)=0$,
либо $h_2(\alpha)=0$, что противоречит минимальности~$h_\alpha(x)$.

(в) Очевидно, что если $h_\alpha(x)$ делит $f(x)$, то $f(\alpha) =
0$. Докажем обратное утверждение. Разделим $f(x)$ на $h_\alpha(x)$ с
остатком: $f(x) = q(x)h_\alpha(x) + r(x)$, где $q(x), r(x) \in K[x]$
и $\deg r(x) < \deg h_\alpha(x)$. Тогда условие $f(\alpha)=0$ влечёт
$r(\alpha) = 0$. Из минимальности многочлена $h_\alpha(x)$ получаем
$r(x)=0$.
\end{proof}

Для каждого элемента $\alpha \in F$ обозначим через $K(\alpha)$
наименьшее подполе в~$F$, содержащее $K$ и~$\alpha$.

\begin{proposition}
Пусть $\alpha \in F$~--- алгебраический элемент над~$K$ и $n$~---
степень его минимального многочлена над~$K$. Тогда
$$
K(\alpha) = \{\beta_0 + \beta_1 \alpha + \ldots + \beta_{n-1}
\alpha^{n-1} \mid \beta_0, \ldots, \beta_{n-1} \in K\}.
$$
Кроме того, элементы $1, \alpha, \alpha^2, \ldots, \alpha^{n-1}$
линейно независимы над~$K$. В~частности, $[K(\alpha) : K] = n$.
\end{proposition}

\begin{proof}
Легко видеть, что
$$
K(\alpha) = \lbrace \frac{f(\alpha)}{g(\alpha)} \mid f(x), g(x) \in
K[x], f(\alpha) \ne 0 \rbrace.
$$
Действительно, такие элементы лежат в любом подполе поля~$F$,
содержащем~$K$ и~$\alpha$, и сами образуют поле. Теперь возьмём
произвольный элемент $\frac{f(\alpha)}{g(\alpha)} \in K(\alpha)$ и
покажем, что он представим в виде, указанном в условии. Пусть
$h_\alpha(x) \in K[x]$~--- минимальный многочлен элемента~$\alpha$
над~$K$. Поскольку $g(\alpha) \ne 0$, в силу
леммы~\ref{lemma_min_pol}(в) многочлен $h_\alpha(x)$ не
делит~$g(x)$. Но $h_\alpha(x)$ неприводим по
лемме~\ref{lemma_min_pol}(б), поэтому $(g(x), h_\alpha(x)) = 1$.
Значит, существуют такие многочлены $u(x), v(x) \in K[x]$, что $u(x)
g(x) + v(x) h_\alpha(x) = 1$. Подставляя в последнее равенство $x =
\alpha$, мы получаем $u(\alpha) g(\alpha) = 1$. Отсюда
$\frac{f(\alpha)}{g(\alpha)} = f(\alpha) u(\alpha)$, и мы избавились
от знаменателя. Теперь уменьшим степень числителя. Пусть $r(x)$~---
остаток от деления $f(x)u(x)$ на~$h_\alpha(x)$. Тогда $f(\alpha)
u(\alpha) = r(\alpha)$ и, значит, $\frac{f(\alpha)}{g(\alpha)} =
r(\alpha)$, что показывает представимость элемента
$\frac{f\alpha)}{g(\alpha)}$ в требуемом виде.

Остаётся показать, что элементы $1, \alpha, \ldots, \alpha^{n-1}$
поля $F$ линейно независимы над~$K$. Если $$\gamma_0 + \gamma_1
\alpha + \ldots + \gamma_{n-1} \alpha^{n-1} = 0$$ для некоторых
$\gamma_0, \gamma_1, \ldots, \gamma_{n-1} \in K$, то для многочлена
$w(x) = \gamma_0 + \gamma_1x + \ldots + \gamma_{n-1}x^{n-1} \in
K[x]$ получаем $w(\alpha) = 0$. Тогда из
леммы~\ref{lemma_min_pol}(в) и условия $\deg w(x) < \deg
h_\alpha(x)$ вытекает, что $w(x) = 0$, то есть $\gamma_0 = \gamma_1
= \ldots = \gamma_{n-1} = 0$.
\end{proof}

\begin{theorem}
Пусть $K$~--- произвольное поле и $f(x)\in K[x]$~--- многочлен
положительной степени. Тогда существует конечное расширение
$K\subseteq F$, в котором многочлен $f(x)$ имеет корень.
\end{theorem}

\begin{proof}
Достаточно построить конечное расширение, в~котором имеет корень
один из неприводимых делителей $p(x)$ многочлена~$f(x)$.

Покажем сначала, что факторкольцо $K[x]/(p(x))$ является полем. В
самом деле, если многочлен $g(x) \in\nobreak K[x]$ не делится
на~$p(x)$, то $(g(x), p(x)) = 1$, и тогда существуют многочлены
$u(x), v(x) \in K[x]$, для которых $u(x) g(x) + v(x) p(x) = 1$. Взяв
образ последнего равенства в факторкольце $K[x] / (p(x))$, мы
получим
$$(u(x) + (p(x))) (g(x) + (p(x))) = 1 + (p(x)),$$ т.\,е. элемент
$u(x)+(p(x))$ является обратным к $g(x)+(p(x))$. Значит, $K[x] /
(p(x))$~--- поле, и мы возьмём его в качестве~$F$.

Заметим теперь, что расширение $K \subseteq F$ является конечным.
Действительно, для всякого многочлена $g(x) \in K[x]$ в поле $F =
K[x]/(p(x))$ имеем $g(x) + (p(x)) = r(x) + (p(x))$, где $r(x)$~---
остаток от деления $g(x)$ на $p(x)$. Отсюда следует, что $F$
порождается как векторное пространство над~$K$ элементами
$$
1 + (p(x)), x + (p(x)), \ldots, x^{n-1} + (p(x)),
$$
где $n = \deg p(x)$. (Так же легко показать, что эти элементы
образуют базис в $F$ над~$K$.)

Остаётся показать, что в поле $F$ многочлен $p(x)$ имеет корень. Это
похоже на обман, но корнем будет... $x + (p(x))$. Действительно,
пусть $p(x) = a_n x^n + a_{n-1} x^{n-1} + a_1x + a_0$, где $a_0,
a_1, \ldots, a_n \in K$. Тогда
\begin{multline*}
p(x + (p(x))) = a_n(x + (p(x)))^n + a_{n-1} (x + (p(x)))^{n-1} +
\ldots + a_1 (x + (p(x)) + a_0 =\\
= (a_nx^n + a_{n-1}x^{n-1} + \ldots a_1 x + a_0) + (p(x)) = p(x) +
(p(x)) = (p(x)),
\end{multline*}
а $(p(x))$ есть не что иное, как нуль в~$F$.
\end{proof}

Говорят, что поле $K[x] / (p(x))$ получено из поля $K$ {\it
присоединением корня} неприводимого многочлена~$p(x)$. Нетрудно
проверить, что если $\alpha$~--- некоторый корень многочлена $p(x)$
в $K[x]/(p(x))$, то поле $K[x]/(p(x))$ совпадает с подполем
$K(\alpha)$.

\begin{definition}
Пусть $K$~--- некоторое поле и $f(x)\in K[x]$~--- многочлен
положительной степени. {\it Полем разложения} многочлена $f(x)$
называется такое расширение $F$ поля~$K$, что

(1) многочлен $f(x)$ разлагается над $F$ на линейные множители;

(2) корни многочлена $f(x)$ не лежат ни в каком собственном подполе
поля~$F$, содержащем~$K$.
\end{definition}

\begin{example}
Рассмотрим многочлен $f(x) = x^4+x^3+x^2+x+1$ над $\QQ$. Так как
$(x-1)f(x) = x^5-1$, корнями многочлена $f(x)$ являются все корни
степени $5$ из единицы, отличные от единицы. Если присоединить к
$\QQ$ один из корней $\epsilon$ многочлена~$f$, то его остальные
корни можно получить, возводя число $\epsilon$ в натуральные
степени. Таким образом, присоединение одного корня сразу приводит к
полю разложения многочлена.
\end{example}


\begin{example}
Многочлен $f(x)=x^3-2$ неприводим над полем $\QQ$. Присоединение к
полю $\QQ$ корня этого многочлена приводит к полю $\QQ[x]/(x^3-2)
\cong \QQ(\sqrt[3]{2})$. Данное поле не является полем разложения
многочлена~$f(x)$, поскольку в нём $f(x)$ имеет только один корень и
не имеет двух других корней. Поскольку корнями данного многочлена
являются числа
$$
\sqrt[3]{2}, \quad \sqrt[3]{2}(-\frac{1}{2} + \cfrac{\sqrt{-3}}{2}),
\quad \sqrt[3]{2}(-\frac{1}{2} - \frac{\sqrt{-3}}{2}),
$$
полем разложения многочлена $f(x)$ является поле
$$
F = \{\alpha_0 + \alpha_1 \sqrt[3]{2} + \alpha_2 \sqrt[3]{4} +
\alpha_3 \sqrt{-3} + \alpha_4 \sqrt[3]{2} \sqrt{-3} + \alpha_5
\sqrt[3]{4} \sqrt{-3} \mid \alpha_i \in \QQ\},
$$
которое имеет над $\QQ$ степень~$6$.
\end{example}

Пусть $F$ и $F'$~--- два расширения поля~$K$. Говорят, что
изоморфизм $F \xrightarrow{\sim} F'$ является \textit{тождественным
на~$K$}, если при этом изоморфизме каждый элемент поля $K$ переходит
в себя.

\begin{theorem}
Поле разложения любого многочлена $f(x) \in K[x]$ существует и
единственно с точностью до изоморфизма, тождественного на~$K$.
\end{theorem}

Доказательство этой теоремы можно найти, например, в книге
Э.\,Б.~Винберга \guillemotleft Курс алгебры\guillemotright{}. Мы не
включаем это доказательство в программу нашего курса.

\bigskip

\begin{thebibliography}{99}
\bibitem{Vi}
Э.\,Б.~Винберг. Курс алгебры. М.: Факториал Пресс, 2002 (глава~1,
\S\S\,3--6 и глава~9, \S\,5)
\bibitem{Ko1}
А.\,И.~Кострикин. Введение в алгебру. Основы алгебры. М.: Наука.
Физматлит, 1994 (глава~4, \S\,3)
\bibitem{Ko3}
А.\,И.~Кострикин. Введение в алгебру. Основные структуры алгебры.
М.: Наука. Физматлит, 2000 (глава~5, \S\,1)
\bibitem{SZ}
Сборник задач по алгебре под редакцией А.\,И.~Кострикина. Новое
издание. М.: МЦНМО, 2009 (глава~14, \S\S\,66-67)
\end{thebibliography}



\end{document}
