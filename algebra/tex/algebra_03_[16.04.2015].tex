\documentclass[a4paper,10pt]{amsart}

\usepackage[T2A]{fontenc}
\usepackage[utf8x]{inputenc}
\usepackage{amssymb}
\usepackage[russian]{babel}
\usepackage{geometry}
\usepackage{hyperref}

\geometry{a4paper,top=2cm,bottom=2cm,left=2cm,right=2cm}

\setlength{\parindent}{0pt}
\setlength{\parskip}{\medskipamount}

\newcommand{\Ker}{\mathop{\mathrm{Ker}}}
\renewcommand{\Im}{\mathop{\mathrm{Im}}}

%\def\Ker{{\rm Ker}}%
%\def\Im{{\rm Im}}%
\def\Mat{{\rm Mat}}%
\def\cont{{\rm cont}}%
\def\Tor{{\rm Tor}}%
\def\Char{{\rm Char}}%
\def\signum{{\rm sig}}%
\def\Sym{{\rm Sym}}%
\def\St{{\rm St}}%
\def\Aut{{\rm Aut}}%
\def\Chi{{\mathbb X}}%
\def\Tau{{\rm T}}%
\def\Rho{{\rm R}}%
\def\rk{{\rm rk}}%
\def\ggT{{\rm ggT}}%
\def\kgV{{\rm kgV}}%
\def\Div{{\rm Div}}%
\def\div{{\rm div}}%
\def\quot{/\!\!/}%
\def\mal{\! \cdot \!}%
\def\Of{{\mathcal{O}}}
%
\def\subgrpneq{\le}%
\def\subgrp{\le}%
\def\ideal#1{\le_{#1}}%
\def\submod#1{\le_{#1}}%
%
\def\Bild{{\rm Bild}}%
\def\Kern{{\rm Kern}}%
\def\bangle#1{{\langle #1 \rangle}}%
\def\rq#1{\widehat{#1}}%
\def\t#1{\widetilde{#1}}%
\def\b#1{\overline{#1}}%
%
\def\abs#1{{\vert #1 \vert}}%
\def\norm#1#2{{\Vert #1 \Vert}_{#2}}%
\def\PS#1#2{{\sum_{\nu=0}^{\infty} #1_{\nu} #2^{\nu}}}%
%
\def\C{{\rm C}}%
\def\O{{\rm O}}%
\def\HH{{\mathbb H}}%
\def\LL{{\mathbb L}}%
\def\FF{{\mathbb F}}%
\def\CC{{\mathbb C}}%
\def\KK{{\mathbb K}}%
\def\TT{{\mathbb T}}%
\def\ZZ{{\mathbb Z}}%
\def\RR{{\mathbb R}}%
\def\SS{{\mathbb S}}%
\def\NN{{\mathbb N}}%
\def\QQ{{\mathbb Q}}%
\def\PP{{\mathbb P}}%
\def\AA{{\mathbb A}}%
%
\def\eins{{\mathbf 1}}%
%
\def\AG{{\rm AG}}%
\def\Aut{{\rm Aut}}%
\def\Hol{{\rm Hol}}%
\def\GL{{\rm GL}}%
\def\SL{{\rm SL}}%
\def\SO{{\rm SO}}%
\def\Sp{{\rm Sp}}%
\def\gl{\mathfrak{gl}}%
\def\rg{{\rm rg}}%
\def\sl{\mathfrak{sl}}%
\def\HDiv{{\rm HDiv}}%
\def\CDiv{{\rm CDiv}}%
\def\Res{{\rm Res}}%
\def\Pst{{\rm Pst}}%
\def\Nst{{\rm Nst}}%
\def\rad{{\rm rad}}%
\def\GL{{\rm GL}}%
\def\Tr{{\rm Tr}}%
\def\Pic{{\rm Pic}}%
\def\Hom{{\rm Hom}}%
\def\hom{{\rm hom}}%
\def\Mor{{\rm Mor}}%
\def\codim{{\rm codim}}%
\def\Supp{{\rm Supp}}%
\def\Spec{{\rm Spec}}%
\def\Proj{{\rm Proj}}%
\def\Maps{{\rm Maps}}%
\def\cone{{\rm cone}}%
\def\ord{{\rm ord}}%
\def\pr{{\rm pr}}%
\def\id{{\rm id}}%
\def\mult{{\rm mult}}%
\def\inv{{\rm inv}}%
\def\neut{{\rm neut}}%
%
\def\AAA{\mathcal{A}}
\def\BBB{\mathcal{B}}
\def\CCC{\mathcal{C}}
\def\EEE{\mathcal{E}}
\def\FFF{\mathcal{F}}

\def\CF{{\rm CF}}
\def\GCD{{\rm GCD}}
\def\Mat{{\rm Mat}}
\def\End{{\rm End}}
\def\cont{{\rm cont}}
\def\Kegel{{\rm Kegel}}
\def\Char{{\rm Char}}
\def\Der{{\rm Der}}
\def\signum{{\rm sg}}
\def\grad{{\rm grad}}
\def\Spur{{\rm Spur}}
\def\Sym{{\rm Sym}}
\def\Alt{{\rm Alt}}
\def\Abb{{\rm Abb}}
\def\Chi{{\mathbb X}}
\def\Tau{{\rm T}}
\def\Rho{{\rm R}}
\def\ad{{\rm ad}}
\def\Frob{{\rm Frob}}
\def\Rang{{\rm Rang}}
\def\SpRang{{\rm SpRang}}
\def\ZRang{{\rm ZRang}}
\def\ggT{{\rm ggT}}
\def\kgV{{\rm kgV}}
\def\Div{{\rm Div}}
\def\div{{\rm div}}
\def\quot{/\!\!/}
\def\mal{\! \cdot \!}
\def\add{{\rm add}}
\def\mult{{\rm mult}}
\def\smult{{\rm smult}}

\def\subgrpneq{\le}
\def\subgrp{\le}
\def\ideal#1{\unlhd_{#1}}
\def\submod#1{\le_{#1}}

\def\Bild{{\rm Bild}}
\def\Kern{{\rm Kern}}
\def\Kon{{\rm Kon}}
\def\bangle#1{{\langle #1 \rangle}}
\def\rq#1{\widehat{#1}}
\def\t#1{\widetilde{#1}}
\def\b#1{\overline{#1}}

\def\abs#1{{\vert #1 \vert}}
\def\norm#1#2{{\Vert #1 \Vert}_{#2}}
\def\PS#1#2{{\sum_{\nu=0}^{\infty} #1_{\nu} #2^{\nu}}}


\def\eins{{\mathbf 1}}

\def\ElM{{\rm ElM}}
\def\ZOp{{\rm ZOp}}
\def\SpOp{{\rm SpOp}}
\def\Gal{{\rm Gal}}
\def\Def{{\rm Def}}
\def\Fix{{\rm Fix}}
\def\ord{{\rm ord}}
\def\Aut{{\rm Aut}}
\def\Hol{{\rm Hol}}
\def\GL{{\rm GL}}
\def\SL{{\rm SL}}
\def\SO{{\rm SO}}
\def\Sp{{\rm Sp}}
\def\Spann{{\rm Spann}}
\def\Lin{{\rm Lin}}
\def\gl{\mathfrak{gl}}
\def\rg{{\rm rg}}
\def\sl{\mathfrak{sl}}
\def\so{\mathfrak{so}}
\def\sp{\mathfrak{sp}}
\def\gg{\mathfrak{g}}
\def\HDiv{{\rm HDiv}}
\def\CDiv{{\rm CDiv}}
\def\Res{{\rm Res}}
\def\Pst{{\rm Pst}}
\def\Nst{{\rm Nst}}
\def\WDiv{{\rm WDiv}}
\def\GL{{\rm GL}}
\def\Tr{{\rm Tr}}
\def\Pic{{\rm Pic}}
\def\Hom{{\rm Hom}}
\def\hom{{\rm hom}}
\def\Mor{{\rm Mor}}
\def\codim{{\rm codim}}
\def\Supp{{\rm Supp}}
\def\Spec{{\rm Spec}}
\def\Proj{{\rm Proj}}
\def\Maps{{\rm Maps}}
\def\cone{{\rm cone}}
\def\ord{{\rm ord}}
\def\pr{{\rm pr}}
\def\id{{\rm id}}
\def\mult{{\rm mult}}
\def\inv{{\rm inv}}
\def\neut{{\rm neut}}
\def\trdeg{{\rm trdeg}}
\def\sing{{\rm sing}}
\def\reg{{\rm reg}}


%%%%%%%%%%%%%%%%%%%%%%%%%%%

\newtheorem{theorem}{Теорема}
\newtheorem{proposition}{Предложение}
\newtheorem{lemma}{Лемма}
\newtheorem{corollary}{Следствие}
\theoremstyle{definition}
\newtheorem{definition}{Определение}
\newtheorem{problem}{Задача}
%
\theoremstyle{remark}
\newtheorem{exc}{Упражнение}
\newtheorem{remark}{Замечание}
\newtheorem{example}{Пример}

\renewcommand{\theenumi}{\textup{(\alph{enumi})}}
\renewcommand{\labelenumi}{\theenumi}
\newcounter{property}
\renewcommand{\theproperty}{\textup{(\arabic{property})}}
\newcommand{\property}{\refstepcounter{property}\item}
\newcounter{prooperty}
\renewcommand{\theprooperty}{\textup{(\arabic{prooperty})}}
\newcommand{\prooperty}{\refstepcounter{prooperty}\item}

\makeatletter
\def\keywords#1{{\def\@thefnmark{\relax}\@footnotetext{#1}}}
\let\subjclass\keywords
\makeatother
%
\begin{document}
%
\sloppy
%\thispagestyle{empty}
%
\centerline{\large \bf Лекции курса \guillemotleft
Алгебра\guillemotright{}, лекторы И.\,В.~Аржанцев и Р.\,С.~Авдеев}

\smallskip

\centerline{\large ФКН НИУ ВШЭ, 1-й курс ОП ПМИ, 4-й модуль,
2014/2015 учебный год}


\bigskip

\section*{Лекция 3}

\medskip

{\it Конечно порождённые и свободные абелевы группы. Подгруппы
свободных абелевых групп. Теорема о согласованных базисах. Алгоритм
приведения целочисленной матрицы к диагональному виду.}

\medskip

Всюду в этой и следующей лекции $(A,+)$~--- абелева группа с
аддитивной формой записи операции. Для произвольного элемента $a\in
A$ и целого числа $s$ положим
$$
sa =
\begin{cases}
\underbrace{a + \ldots + a}_s, & \text{ если } s > 0; \\
0, & \text{ если } s = 0; \\
\underbrace{(-a) + \ldots + (-a)}_{|s|}, & \text{ если } s < 0.
\end{cases}
$$

\begin{definition}
Абелева группа $A$ называется {\it конечно порождённой}, если
найдутся такие элементы $a_1,\ldots,a_n\in A$, что всякий элемент
$a\in A$ представим в виде $a=s_1a_1 + \ldots + s_na_n$ для
некоторых целых чисел $s_1, \ldots, s_n$. При этом элементы $a_1,
\ldots, a_n$ называются {\it порождающими} или {\it образующими}
группы~$A$.
\end{definition}

\begin{remark}
Всякая конечно порождённая группа конечна или счётна.
\end{remark}

\begin{remark}
Всякая конечная группа является конечно порождённой.
\end{remark}

\begin{definition}
Конечно порождённая абелева группа $A$ называется {\it свободной},
если в ней существует {\it базис}, т.\,е. такой набор элементов
$a_1,\ldots, a_n$, что каждый элемент $a\in A$ единственным образом
представим в виде $a=s_1a_1 + \ldots + s_na_n$, где $s_1, \ldots,
s_n \in \ZZ$. При этом число $n$ называется {\it рангом} свободной
абелевой группы $A$ и обозначается $\rk\,A$.
\end{definition}

\begin{example}
Абелева группа $\ZZ^n:=\{(c_1,\ldots,c_n) \mid c_i\in\ZZ\}$ является
свободной с базисом
$$
\begin{aligned}
e_1 &= (1,0,\ldots,0), \\
e_2 &= (0,1,\ldots,0),\\
 &\ldots \\
e_n &= (0,0,\ldots,1).
\end{aligned}
$$
Этот базис называется {\it стандартным}. В группе $\ZZ^n$ можно
найти и много других базисов. Ниже мы все их опишем.
\end{example}

\begin{proposition}
Ранг свободной абелевой группы определён корректно, т.\,е. любые два
её базиса содержат одинаковое число элементов.
\end{proposition}

\begin{proof}
Пусть $a_1, \ldots, a_n$ и $b_1, \ldots, b_m$~--- два базиса
группы~$A$. Предположим, что $n < m$. Элементы $b_1, \ldots, b_m$
однозначно разлагаются по базису $a_1, \ldots, a_n$, поэтому мы
можем записать
$$
\begin{aligned}
b_1 &= s_{11}a_1 + s_{12}a_2 + \ldots + s_{1n}a_n, \\
b_2 &= s_{21}a_1 + s_{22}a_2 + \ldots + s_{2n}a_n,\\
 &\ldots \\
b_m &= s_{m1}a_1 + s_{m2}a_2 + \ldots + s_{mn}a_n,
\end{aligned}
$$
где все коэффициенты $s_{ij}$~--- целые числа. Рассмотрим
прямоугольную матрицу $S = (s_{ij})$ размера $m \times n$. Так как
$n < m$, то ранг этой матрицы не превосходит~$n$, а~значит, строки
этой матрицы линейно зависимы над~$\QQ$. Домножая коэффициенты этой
зависимости на наименьшее общее кратное их знаменателей, мы найдём
такие целые $s_1, \ldots, s_m$, из которых не все равны нулю, что
$s_1 b_1 + \ldots + s_m b_m=0$. Поскольку $0 = 0b_1 + \ldots +
0b_m$, это противоречит однозначной выразимости элемента $0$ через
базис $b_1, \ldots, b_m$.
\end{proof}

\begin{proposition}
Всякая свободная абелева группа ранга $n$ изоморфна группе $\ZZ^n$.
\end{proposition}

\begin{proof}
Пусть $A$~--- свободная абелева группа, и пусть $a_1,\ldots,a_n$~---
её базис. Рассмотрим отображение
$$\varphi \colon \ZZ^n \to A, \quad (s_1, \ldots, s_n)
\mapsto s_1a_1 + \ldots + s_na_n.
$$
Легко видеть, что $\varphi$~--- гомоморфизм. Так как всякий элемент
$a \in A$ представим в виде $s_1a_1 + \ldots + s_na_n$, где $s_1,
\ldots, s_n \in \ZZ$, то $\varphi$ сюръективен. Из единственности
такого представления следует инъективность~$\varphi$. Значит,
$\varphi$~--- изоморфизм.
\end{proof}

Пусть $e_1', \ldots, e_n'$~--- некоторый набор элементов из $\ZZ^n$.
Выразив эти элементы через стандартный базис $e_1, \ldots, e_n$, мы
можем записать
$$
(e_1', \ldots, e_n') = (e_1, \ldots, e_n)C,
$$
где $C$~--- целочисленная квадратная матрица порядка~$n$.

\begin{proposition}
Элементы $e_1', \ldots, e_n'$ составляют базис группы $\ZZ^n$ тогда
и только тогда, когда $\det C = \pm 1$.
\end{proposition}

\begin{proof}
Предположим сначала, что $e'_1, \ldots, e'_n$~--- базис. Тогда
элементы $e_1, \ldots, e_n$ через него выражаются, поэтому $(e_1,
\ldots, e_n) = (e'_1, \ldots, e'_n) D$ для некоторой целочисленной
квадратной матрицы $D$ порядка~$n$. Но тогда $(e_1, \ldots, e_n) =
(e_1, \ldots, e_n)CD$, откуда $CD = E_n$, где $E_n$~--- единичная
матрица порядка~$n$. Значит, $(\det C)(\det D) = 1$. Учитывая, что
$\det C$ и $\det D$~--- целые числа, мы получаем $\det C = \pm 1$.

Обратно, пусть $\det C = \pm 1$. Тогда матрица $C^{-1}$ является
целочисленной, а соотношение $(e_1, \ldots, e_n) = (e'_1, \ldots,
e'_n)C^{-1}$ показывает, что элементы $e_1, \ldots, e_n$ выражаются
через $e'_1, \ldots, e'_n$. Но $e_1, \ldots, e_n$~--- базис, поэтому
элементы $e'_1, \ldots, e'_n$ порождают группу~$\ZZ^n$. Осталось
доказать, что всякий элемент из $\ZZ^n$ однозначно через них
выражается. Предположим, что $s'_1e'_1 + \ldots + s'_ne'_n =
s''_1e'_1 + \ldots + s''_n e'_n$ для некоторых целых чисел $s'_1,
\ldots, s'_n, s''_1, \ldots, s''_n$. Мы можем это переписать в
следующем виде:
$$
(e'_1, \ldots, e'_n)
\begin{pmatrix} s'_1 \\ \vdots \\ s'_n \end{pmatrix} =
(e'_1, \ldots, e'_n)
\begin{pmatrix} s''_1 \\ \vdots \\ s''_n \end{pmatrix}.
$$
Учитывая, что $(e'_1, \ldots, e'_n) = (e_1, \ldots, e_n)C$ и что
$e_1, \ldots, e_n$~--- это базис, получаем
$$
C \begin{pmatrix} s'_1 \\ \vdots \\ s'_n \end{pmatrix} = C
\begin{pmatrix} s''_1 \\ \vdots \\ s''_n \end{pmatrix}.
$$
Домножая это равенство слева на~$C^{-1}$, окончательно получаем
$$
\begin{pmatrix} s'_1 \\ \vdots \\ s'_n \end{pmatrix} =
\begin{pmatrix} s''_1 \\ \vdots \\ s''_n \end{pmatrix}.
$$
\end{proof}

\begin{theorem}
Всякая подгруппа $N$ свободной абелевой группы $L$ ранга $n$
является свободной абелевой группой ранга $\leqslant n$.
\end{theorem}

\begin{proof}
Воспользуемся индукцией по $n$. При $n=0$ доказывать нечего. Пусть
$n>0$ и $e_1,\ldots,e_n$~--- базис группы $L$. Рассмотрим в $L$
подгруппу
$$
L_1 = \langle e_1,\ldots,e_{n-1}\rangle : = \ZZ e_1 + \ldots + \ZZ
e_{n-1}.
$$
Это свободная абелева группа ранга $n-1$. По предположению индукции
подгруппа $N_1:=N\cap L_1 \subseteq L_1$ является свободной абелевой
группой ранга $m \leqslant n-1$. Зафиксируем в $N_1$ базис $f_1,
\ldots, f_m$.

Рассмотрим отображение
$$
\varphi \colon N \to \ZZ, \quad s_1e_1 + \ldots + s_ne_n \mapsto
s_n.
$$
Легко видеть, что $\varphi$~--- гомоморфизм и что $\Ker \varphi =
N_1$. Далее, $\Im \varphi$~--- подгруппа в~$\ZZ$, по предложению~1
из лекции~1 она имеет вид $k \ZZ$ для некоторого целого $k \geqslant
0$. Если $k=0$, то $N \subseteq L_1$, откуда $N = N_1$ и всё
доказано. Если $k>0$, то пусть $f_{m+1}$~--- какой-нибудь элемент из
$N$, для которого $\varphi(f_{m+1}) = k$. Докажем, что $f_1, \ldots,
f_m, f_{m+1}$~--- базис в~$N$. Пусть $f \in N$~--- произвольный
элемент, и пусть $\varphi(f) = sk$, где $s \in \ZZ$. Тогда
$\varphi(f - sf_{m+1}) = 0$, откуда $f - sf_{m+1} \in N_1$ и,
следовательно, $f - sf_{m+1} = s_1 f_1 + \ldots + s_m f_m$ для
некоторых $s_1, \ldots, s_m \in \ZZ$. Значит, $f = s_1 f_1 + \ldots
+ s_m f_m + s f_{m+1}$ и элементы $f_1, \ldots, f_m, f_{m+1}$
порождают группу~$N$. Осталось доказать, что они образуют базис
в~$N$. Предположим, что
$$
s_1 f_1 + \ldots + s_m f_m + s_{m+1} f_{m+1} = s'_1 f_1 + \ldots +
s'_m f_m + s'_{m+1} f_{m+1}
$$
для некоторых целых чисел $s_1, \ldots, s_m, s_{m+1}, s'_1, \ldots,
s'_m, s'_{m+1}$. Рассмотрев образ обеих частей этого равенства при
гомоморфизме~$\varphi$, получаем $s_{m+1} k = s'_{m+1} k$, откуда
$s_{m+1} = s'_{m+1}$ и
$$
s_1 f_1 + \ldots + s_m f_m = s'_1 f_1 + \ldots + s'_m f_m.
$$
Но $f_1, \ldots, f_m$~--- базис в~$N_1$, поэтому $s_1 = s'_1$,
\ldots, $s_m = s'_m$.
\end{proof}

Дадим более точное описание подгрупп свободных абелевых групп.

\smallskip

{\bf Теорема о согласованных базисах.}\ Для всякой подгруппы $N$
свободной абелевой группы $L$ ранга $n$ найдётся такой базис $e_1,
\ldots, e_n$ группы $L$ и такие натуральные числа $u_1, \ldots,
u_m$, $m \leqslant n$, что $u_1 e_1, \ldots, u_m e_m$~--- базис
группы $N$ и $u_i | u_{i+1}$ при $i = 1, \ldots, m-1$.

\smallskip


Доказательство этой теоремы потребует некоторой подготовки.

\begin{definition}
{\it Целочисленными элементарными преобразованиями строк} матрицы
называются преобразования следующих трёх типов:

1) прибавление к одной строке другой, умноженной на целое число;

2) перестановка двух строк;

3) умножение одной строки на $-1$.

Аналогично определяются {\it целочисленные элементарные
преобразования столбцов} матрицы.
\end{definition}

Прямоугольную матрицу $C=(c_{ij})$ размера $n\times m$ назовём {\it
диагональной} и обозначим $\text{diag}(u_1,\ldots,u_p)$, если
$c_{ij}=0$ при $i\ne j$ и $c_{ii}=u_i$ при $i=1,\ldots,p$, где
$p=\text{min}(n,m)$.

\begin{proposition} \label{palg}
Всякую прямоугольную целочисленную матрицу $C=(c_{ij})$ с помощью
элементарных преобразований строк и столбцов можно привести к виду
$\text{diag}(u_1,\ldots,u_p)$, где $u_1,\ldots,u_p \geqslant 0$ и
$u_i|u_{i+1}$ при $i=1,\ldots,p-1$.
\end{proposition}

\begin{proof}
Если $C=0$, то доказывать нечего. Если $C\ne 0$, но $c_{11}=0$, то
переставим строки и столбцы и получим $c_{11}\ne 0$. Умножив, если
нужно, первую строку на $-1$, добьёмся условия $c_{11}>0$. Теперь
будем стремиться уменьшить~$c_{11}$.

Если какой-то элемент $c_{i1}$ не делится на $c_{11}$, то разделим с
остатком: $c_{i1}=qc_{11}+r$. Вычитая из $i$-й строки $1$-ю строку,
умноженную на~$q$, и затем переставляя $1$-ю и $i$-ю строки,
уменьшаем~$c_{11}$. Повторяя эту процедуру, в~итоге добиваемся, что
все элементы $1$-й строки и $1$-го столбца делятся на $c_{11}$.

Если какой-то $c_{ij}$ не делится на $c_{11}$, то поступаем
следующим образом. Вычтя из $i$-й строки $1$-ю строку с подходящим
коэффициентом, добьёмся $c_{i1}=0$. После этого прибавим к $1$-й
строке $i$-ю строку. При этом $c_{11}$ не изменится, а $c_{1j}$
перестанет делиться на $c_{11}$, и мы вновь сможем уменьшить
$c_{11}$.

В~итоге добьёмся того, что все элементы делятся на~$c_{11}$. После
этого обнулим все элементы $1$-й строки и $1$-го столбца, начиная со
вторых, и продолжим процесс с меньшей матрицей.
\end{proof}

Теперь мы готовы доказать теорему о согласованных базисах.

\begin{proof}[Доказательство теоремы о согласованных базисах]
Мы знаем, что $N$ является свободной абелевой группой ранга $m
\leqslant n$. Пусть $e_1, \ldots, e_n$~--- базис в $L$ и $f_1,
\ldots, f_m$~--- базис в~$N$. Тогда $(f_1, \ldots, f_m) = (e_1,
\ldots, e_n)C$, где $C$~--- целочисленная матрица размера $n \times
m$ и ранга~$m$. Покажем, что целочисленные элементарные
преобразования строк (столбцов) матрицы $C$~--- это в точности
элементарные преобразования над базисом в~$L$ (в~$N$). Для этого
рассмотрим сначала случай строк. Заметим, что каждое из
целочисленных элементарных преобразований строк реализуется при
помощи умножения матрицы $C$ слева на квадратную матрицу~$P$
порядка~$n$, определяемую следующим образом:

(1) в случае прибавления к $i$-й строке $j$-й, умноженной на целое
число~$z$, в матрице~$P$ на диагонали стоят единицы, на $(ij)$-м
месте~--- число~$z$, а на остальных местах~--- нули;

(2) в случае перестановки $i$-й и $j$-й строк имеем $p_{ij} = p_{ji}
= 1$, $p_{kk} = 1$ при $k \ne i,j$, а на остальных местах стоят
нули;

(3) в случае умножения $i$-й строки на $-1$ имеем $p_{ii} = -1$,
$p_{jj} = 1$ при $j \ne i$, а на остальных местах стоят нули.

Теперь заметим, что равенство $(f_1, \ldots, f_m) = (e_1, \ldots,
e_n)C$ эквивалентно равенству $(f_1, \ldots, f_m) = (e_1, \ldots,
e_n)P^{-1} PC$. Таким образом, базис $(f_1, \ldots, f_m)$ выражается
через новый базис $(e'_1, \ldots, e'_n) := (e_1, \ldots, e_n)P^{-1}$
при помощи матрицы~$PC$.

В случае столбцов всё аналогично: каждое из целочисленых
элементарных преобразований столбцов реализуется при помощи
умножения матрицы $C$ справа на некоторую квадратную матрицу $Q$
порядка~$m$ (определяемую почти так же, как~$P$). В~этом случае
имеем $(f_1, \ldots, f_m)Q = (e_1, \ldots, e_n)CQ$, так что новый
базис $(f'_1, \ldots, f'_m) := (f_1, \ldots, f_m)Q$ выражается через
$(e_1, \ldots, e_n)$ при помощи матрицы $CQ$.

Воспользовавшись предложением~\ref{palg}, мы можем привести матрицу
$C$ при помощи целочисленных элементарных преобразований строк и
столбцов к диагональному виду~$C'' = \text{diag}(u_1, \ldots, u_m)$,
где $u_i | u_{i+1}$ для всех $i = 1, \ldots, m-1$. С~учётом
сказанного выше это означает, что для некоторого базиса $e''_1,
\ldots, e''_n$ в~$L$ и некоторого базиса $f''_1, \ldots, f''_m$
в~$N$ справедливо соотношение $(f''_1, \ldots, f''_m) = (e''_1,
\ldots, e''_n) C''$. Иными словами, $f''_i = u_i e''_i$ для всех $i
= 1, \ldots, m$, а~это и требовалось.
\end{proof}

\begin{corollary}
В~условиях теоремы о согласованных базисах имеет место изоморфизм
$$
L / N \cong \ZZ_{u_1} \times \ldots \times \ZZ_{u_m} \times
\underbrace{\ZZ \times \ldots \times \ZZ}_{n - m}.
$$
\end{corollary}

\begin{proof}
Рассмотрим изоморфизм $L \cong \ZZ^n = \underbrace{\ZZ \times \ldots
\times \ZZ}_n$, сопоставляющий произвольному элементу $s_1 e_1 +
\ldots + s_n e_n \in L$ набор $(s_1, \ldots, s_n) \in \ZZ^n$. При
этом изоморфизме подгруппа $N \subseteq L$ отождествляется с
подгруппой
$$
u_1 \ZZ \times \ldots \times u_m \ZZ \times \underbrace{\lbrace 0
\rbrace \times \ldots \times \lbrace 0 \rbrace}_{n-m} \subseteq
\ZZ^n.
$$
Теперь требуемый результат получается применением теоремы о
факторизации по сомножителям.
\end{proof}

\begin{remark}
Числа $u_1, \ldots, u_p$, фигурирующие в теореме о согласованных
базисах, называются {\it инвариантными множителями} подгруппы $N
\subseteq L$. Можно показать, что они определены по подгруппе
однозначно.
\end{remark}



\bigskip

\begin{thebibliography}{99}
\bibitem{Vi}
Э.\,Б.~Винберг. Курс алгебры. М.: Факториал Пресс, 2002 (глава~9,
\S\,1)
\bibitem{Ko3}
А.\,И.~Кострикин. Введение в алгебру. Основные структуры алгебры.
М.: Наука. Физматлит, 2000 (глава~2, \S\,3)
\bibitem{SZ}
Сборник задач по алгебре под редакцией А.\,И.~Кострикина. Новое
издание. М.: МЦНМО, 2009 (глава~13, \S\,60)
\end{thebibliography}
\end{document}
