\documentclass[a4paper,10pt]{amsart}

\usepackage[T2A]{fontenc}
\usepackage[utf8x]{inputenc}
\usepackage{amssymb}
\usepackage[russian]{babel}
\usepackage{geometry}

\geometry{a4paper,top=2cm,bottom=2cm,left=2cm,right=2cm}

\setlength{\parindent}{0pt}
\setlength{\parskip}{\medskipamount}

\def\Ker{{\rm Ker}}%
\def\Im{{\rm Im}}%
\def\Mat{{\rm Mat}}%
\def\cont{{\rm cont}}%
\def\Tor{{\rm Tor}}%
\def\Char{{\rm Char}}%
\def\signum{{\rm sig}}%
\def\Sym{{\rm Sym}}%
\def\St{{\rm St}}%
\def\Aut{{\rm Aut}}%
\def\Chi{{\mathbb X}}%
\def\Tau{{\rm T}}%
\def\Rho{{\rm R}}%
\def\rk{{\rm rk}}%
\def\ggT{{\rm ggT}}%
\def\kgV{{\rm kgV}}%
\def\Div{{\rm Div}}%
\def\div{{\rm div}}%
\def\quot{/\!\!/}%
\def\mal{\! \cdot \!}%
\def\Of{{\mathcal{O}}}
%
\def\subgrpneq{\le}%
\def\subgrp{\le}%
\def\ideal#1{\le_{#1}}%
\def\submod#1{\le_{#1}}%
%
\def\Bild{{\rm Bild}}%
\def\Kern{{\rm Kern}}%
\def\bangle#1{{\langle #1 \rangle}}%
\def\rq#1{\widehat{#1}}%
\def\t#1{\widetilde{#1}}%
\def\b#1{\overline{#1}}%
%
\def\abs#1{{\vert #1 \vert}}%
\def\norm#1#2{{\Vert #1 \Vert}_{#2}}%
\def\PS#1#2{{\sum_{\nu=0}^{\infty} #1_{\nu} #2^{\nu}}}%
%
\def\C{{\rm C}}%
\def\O{{\rm O}}%
\def\HH{{\mathbb H}}%
\def\LL{{\mathbb L}}%
\def\FF{{\mathbb F}}%
\def\CC{{\mathbb C}}%
\def\KK{{\mathbb K}}%
\def\TT{{\mathbb T}}%
\def\ZZ{{\mathbb Z}}%
\def\RR{{\mathbb R}}%
\def\SS{{\mathbb S}}%
\def\NN{{\mathbb N}}%
\def\QQ{{\mathbb Q}}%
\def\PP{{\mathbb P}}%
\def\AA{{\mathbb A}}%
%
\def\eins{{\mathbf 1}}%
%
\def\AG{{\rm AG}}%
\def\Aut{{\rm Aut}}%
\def\Hol{{\rm Hol}}%
\def\GL{{\rm GL}}%
\def\SL{{\rm SL}}%
\def\SO{{\rm SO}}%
\def\Sp{{\rm Sp}}%
\def\gl{\mathfrak{gl}}%
\def\rg{{\rm rg}}%
\def\sl{\mathfrak{sl}}%
\def\HDiv{{\rm HDiv}}%
\def\CDiv{{\rm CDiv}}%
\def\Res{{\rm Res}}%
\def\Pst{{\rm Pst}}%
\def\Nst{{\rm Nst}}%
\def\rad{{\rm rad}}%
\def\GL{{\rm GL}}%
\def\Tr{{\rm Tr}}%
\def\Pic{{\rm Pic}}%
\def\Hom{{\rm Hom}}%
\def\hom{{\rm hom}}%
\def\Mor{{\rm Mor}}%
\def\codim{{\rm codim}}%
\def\Supp{{\rm Supp}}%
\def\Spec{{\rm Spec}}%
\def\Proj{{\rm Proj}}%
\def\Maps{{\rm Maps}}%
\def\cone{{\rm cone}}%
\def\ord{{\rm ord}}%
\def\pr{{\rm pr}}%
\def\id{{\rm id}}%
\def\mult{{\rm mult}}%
\def\inv{{\rm inv}}%
\def\neut{{\rm neut}}%
%
\def\AAA{\mathcal{A}}
\def\BBB{\mathcal{B}}
\def\CCC{\mathcal{C}}
\def\EEE{\mathcal{E}}
\def\FFF{\mathcal{F}}

\def\CF{{\rm CF}}
\def\GCD{{\rm GCD}}
\def\Mat{{\rm Mat}}
\def\End{{\rm End}}
\def\cont{{\rm cont}}
\def\Kegel{{\rm Kegel}}
\def\Char{{\rm Char}}
\def\Der{{\rm Der}}
\def\signum{{\rm sg}}
\def\grad{{\rm grad}}
\def\Spur{{\rm Spur}}
\def\Sym{{\rm Sym}}
\def\Alt{{\rm Alt}}
\def\Abb{{\rm Abb}}
\def\Chi{{\mathbb X}}
\def\Tau{{\rm T}}
\def\Rho{{\rm R}}
\def\ad{{\rm ad}}
\def\Frob{{\rm Frob}}
\def\Rang{{\rm Rang}}
\def\SpRang{{\rm SpRang}}
\def\ZRang{{\rm ZRang}}
\def\ggT{{\rm ggT}}
\def\kgV{{\rm kgV}}
\def\Div{{\rm Div}}
\def\div{{\rm div}}
\def\quot{/\!\!/}
\def\mal{\! \cdot \!}
\def\add{{\rm add}}
\def\mult{{\rm mult}}
\def\smult{{\rm smult}}

\def\subgrpneq{\le}
\def\subgrp{\le}
\def\ideal#1{\unlhd_{#1}}
\def\submod#1{\le_{#1}}

\def\Bild{{\rm Bild}}
\def\Kern{{\rm Kern}}
\def\Kon{{\rm Kon}}
\def\bangle#1{{\langle #1 \rangle}}
\def\rq#1{\widehat{#1}}
\def\t#1{\widetilde{#1}}
\def\b#1{\overline{#1}}

\def\abs#1{{\vert #1 \vert}}
\def\norm#1#2{{\Vert #1 \Vert}_{#2}}
\def\PS#1#2{{\sum_{\nu=0}^{\infty} #1_{\nu} #2^{\nu}}}


\def\eins{{\mathbf 1}}

\def\ElM{{\rm ElM}}
\def\ZOp{{\rm ZOp}}
\def\SpOp{{\rm SpOp}}
\def\Gal{{\rm Gal}}
\def\Def{{\rm Def}}
\def\Fix{{\rm Fix}}
\def\ord{{\rm ord}}
\def\Aut{{\rm Aut}}
\def\Hol{{\rm Hol}}
\def\GL{{\rm GL}}
\def\SL{{\rm SL}}
\def\SO{{\rm SO}}
\def\Sp{{\rm Sp}}
\def\Spann{{\rm Spann}}
\def\Lin{{\rm Lin}}
\def\gl{\mathfrak{gl}}
\def\rg{{\rm rg}}
\def\sl{\mathfrak{sl}}
\def\so{\mathfrak{so}}
\def\sp{\mathfrak{sp}}
\def\gg{\mathfrak{g}}
\def\HDiv{{\rm HDiv}}
\def\CDiv{{\rm CDiv}}
\def\Res{{\rm Res}}
\def\Pst{{\rm Pst}}
\def\Nst{{\rm Nst}}
\def\WDiv{{\rm WDiv}}
\def\GL{{\rm GL}}
\def\Tr{{\rm Tr}}
\def\Pic{{\rm Pic}}
\def\Hom{{\rm Hom}}
\def\hom{{\rm hom}}
\def\Mor{{\rm Mor}}
\def\codim{{\rm codim}}
\def\Supp{{\rm Supp}}
\def\Spec{{\rm Spec}}
\def\Proj{{\rm Proj}}
\def\Maps{{\rm Maps}}
\def\cone{{\rm cone}}
\def\ord{{\rm ord}}
\def\pr{{\rm pr}}
\def\id{{\rm id}}
\def\mult{{\rm mult}}
\def\inv{{\rm inv}}
\def\neut{{\rm neut}}
\def\trdeg{{\rm trdeg}}
\def\sing{{\rm sing}}
\def\reg{{\rm reg}}


%%%%%%%%%%%%%%%%%%%%%%%%%%%

\newtheorem{theorem}{Теорема}
\newtheorem{proposition}{Предложение}
\newtheorem{lemma}{Лемма}
\newtheorem{corollary}{Следствие}
\theoremstyle{definition}
\newtheorem{definition}{Определение}
\newtheorem{problem}{Задача}
%
\theoremstyle{remark}
\newtheorem{exc}{Упражнение}
\newtheorem{remark}{Замечание}
\newtheorem{example}{Пример}

\renewcommand{\theenumi}{\textup{(\alph{enumi})}}
\renewcommand{\labelenumi}{\theenumi}
\newcounter{property}
\renewcommand{\theproperty}{\textup{(\arabic{property})}}
\newcommand{\property}{\refstepcounter{property}\item}
\newcounter{prooperty}
\renewcommand{\theprooperty}{\textup{(\arabic{prooperty})}}
\newcommand{\prooperty}{\refstepcounter{prooperty}\item}

\makeatletter
\def\keywords#1{{\def\@thefnmark{\relax}\@footnotetext{#1}}}
\let\subjclass\keywords
\makeatother
%
\begin{document}
%
\sloppy
%\thispagestyle{empty}
%
\centerline{\large \bf Лекции курса \guillemotleft
Алгебра\guillemotright{}, лекторы И.\,В.~Аржанцев и Р.\,С.~Авдеев}

\smallskip

\centerline{\large ФКН НИУ ВШЭ, 1-й курс ОП ПМИ, 4-й модуль,
2014/2015 учебный год}


\bigskip

\section*{Лекция 2}

\medskip

{\it Нормальные подгруппы. Факторгруппы и теорема о гомоморфизме.
Центр группы. Прямое произведение групп. Факторизация по
сомножителям. Разложение конечной циклической группы.}

\medskip

\begin{definition}
Подгруппа $H$ группы $G$ называется {\it нормальной}, если $gH=Hg$
для любого $g\in G$.
\end{definition}

\begin{proposition}
Для подгруппы $H \subseteq G$ следующие условия эквивалентны:

\vspace{-2mm}
\begin{enumerate}
\item[(1)]
$H$ нормальна;

\item[(2)]
$gHg^{-1} \subseteq H$ для любого $g \in G$;

\item[(3)]
$gHg^{-1}=H$ для любого $g\in G$.
\end{enumerate}
\end{proposition}

\vspace{-6mm}

\begin{proof}
(1)$\Rightarrow$(2) Пусть $h \in H$ и $g \in G$. Поскольку $gH =
Hg$, имеем $gh = h'g$ для некоторого $h' \in H$. Тогда $ghg^{-1} =
h'gg^{-1} = h' \in H$.\\
(2)$\Rightarrow$(3) Так как $gHg^{-1} \subseteq H$, остаётся
проверить обратное включение. Для $h \in H$ имеем $h = gg^{-1} h g
g^{-1} = g(g^{-1}hg)g^{-1} \subseteq gHg^{-1}$, поскольку $g^{-1}hg
\in H$ в силу пункта~(2), где вместо $g$ взято $g^{-1}$.

(3)$\Rightarrow$(1) Для произвольного $g \in G$ в силу (3) имеем $gH
= gHg^{-1} g \subseteq Hg$, так что $gH \subseteq Hg$. Аналогично
проверяется обратное включение.
\end{proof}
\vspace{-1mm}

Условие (2) в этом предложении кажется излишним, но именно его
удобно проверять при доказательстве нормальности подгруппы~$H$.

Обозначим через $G/H$ множество (левых) смежных классов группы $G$
по нормальной подгруппе~$H$. На $G/H$ можно определить бинарную
операцию следующим образом:
$$
(g_1H)(g_2H):=g_1g_2H.
$$

Зачем здесь нужна нормальность подгруппы $H$? Для проверки
корректности: заменим $g_1$ и $g_2$ другими представителями $g_1h_1$
и $g_2h_2$ тех же смежных классов. Нужно проверить, что
$g_1g_2H=g_1h_1g_2h_2H$. Это следует из того, что
$g_1h_1g_2h_2=g_1g_2(g_2^{-1}h_1g_2)h_2$ и $g_2^{-1}h_1g_2$ лежит в
$H$.

Ясно, что указанная операция на множестве $G/H$ ассоциативна,
обладает нейтральным элементом $eH$ и для каждого элемента $gH$ есть
обратный элемент $g^{-1}H$.

\begin{definition}
Множество $G/H$ с указанной операцией называется {\it факторгруппой}
группы $G$ по нормальной подгруппе $H$.
\end{definition}

\begin{example}
Если $G=(\ZZ,+)$ и $H=n\ZZ$, то $G/H$~--- это в точности группа
вычетов $(\ZZ_n,+)$.
\end{example}

Как представлять себе факторгруппу? В этом помогает теорема о
гомоморфизме. Но прежде чем её сформулировать, обсудим ещё несколько
понятий.

\begin{definition}
Пусть $G$ и $F$~--- группы. Отображение $\varphi\colon G\to F$
называется {\it гомоморфизмом}, если
$\varphi(ab)=\varphi(a)\varphi(b)$ для любых $a,b\in G$.
\end{definition}

\begin{remark}
Подчеркнём, что в этом определении произведение $ab$ берётся в
группе~$G$, в то время как произведение $\varphi(a) \varphi(b)$~---
в группе~$F$.
\end{remark}

\begin{lemma}
Пусть $\varphi \colon G \to F$~--- гомоморфизм групп, и пусть $e_G$
и $e_F$~--- нейтральные элементы групп $G$ и $F$ соответственно.
Тогда:

\vspace{-2mm}
\begin{enumerate}
\item[(а)]
$\varphi(e_G) = e_F$;

\item[(б)]
$\varphi(a^{-1})=\varphi(a)^{-1}$ для любого $a\in G$.
\end{enumerate}
\end{lemma}

\vspace{-5mm}

\begin{proof}
(а) Имеем $\varphi(e_G)=\varphi(e_Ge_G)=\varphi(e_G)\varphi(e_G)$.
Теперь умножая крайние части этого равенства на $\varphi(e_G)^{-1}$
(например, слева), получим $e_F = \varphi(e_G)$.

(б) Имеем $\varphi(a^{-1}) \varphi(a) = \varphi(a^{-1}a) =
\varphi(e_G) = e_F$, откуда $\varphi(a^{-1}) = \varphi(a)^{-1}$.
\end{proof}

\begin{definition}
Гомоморфизм групп $\varphi\colon G\to F$ называется {\it
изоморфизмом}, если отображение $\varphi$ биективно.
\end{definition}

\begin{exc}
Пусть $\varphi\colon G\to F$~--- изоморфизм групп. Проверьте, что
обратное отображение $\varphi^{-1}\colon F \to G$ также является
изоморфизмом.
\end{exc}

\begin{definition}
Группы $G$ и $F$ называют {\it изоморфными}, если между ними
существует изоморфизм.

Обозначение: $G\cong F$ (или $G \simeq F$).
\end{definition}

В алгебре группы рассматривают с точностью до изоморфизма:
изоморфные группы считаются \guillemotleft
одинаковыми\guillemotright{}.

\begin{theorem}
\begin{enumerate}
\item[(а)]
Всякая бесконечная циклическая группа $G$ изоморфна группе
$(\ZZ,+)$.
\item[(б)]
Всякая циклическая группа порядка $n$ изоморфна группе $(\ZZ_n,+)$.
\end{enumerate}
\end{theorem}

\begin{proof}
Пусть $G = \langle g \rangle$. Тогда в первом случае изоморфизм
устанавливает отображение $\langle g \rangle \to \ZZ$, $g^k \mapsto
k$, а~во втором~--- отображение $\langle g \rangle \to \ZZ_n$, $g^k
\mapsto k\, (\text{mod}\, n)$.
\end{proof}

\begin{example}
Отображение $\varphi\colon\RR\to\RR_{>0}$, $a\mapsto e^a$,
устанавливает изоморфизм между группами $(\RR,+)$ и
$(\RR_{>0},\times)$.
\end{example}

\begin{definition}
С каждым гомоморфизмом групп $\varphi\colon G\to F$ связаны его {\it
ядро}
$$
\Ker(\varphi)=\{g\in G \mid \varphi(g)=e_F\}
$$
и {\it образ}
$$
\Im(\varphi)=\{a\in F \mid \exists g\in G : \varphi(g)=a\}.
$$
\end{definition}

Ясно, что $\Ker(\varphi)\subseteq G$ и $\Im(\varphi)\subseteq F$~---
подгруппы.

\begin{lemma}
Гомоморфизм групп $\varphi \colon G \to F$ инъективен тогда и только
тогда, когда $\Ker(\varphi) = \{e_G\}$.
\end{lemma}
\vspace{-3mm}
\begin{proof}
Ясно, что если $\varphi$ инъективен, то $\Ker(\varphi) = \lbrace e_G
\rbrace$. Обратно, пусть $g_1, g_2 \in G$ и $\varphi(g_1) =
\varphi(g_2)$. Тогда $g_1^{-1} g_2 \in \Ker(\varphi)$, поскольку
$\varphi(g_1^{-1} g_2) = \varphi(g_1^{-1}) \varphi(g_2) =
\varphi(g_1)^{-1} \varphi (g_2) = e_F$. Отсюда $g_1^{-1}g_2 = e_G$ и
$g_1 = g_2$.
\end{proof}

\begin{corollary}
Гомоморфизм групп $\varphi\colon G\to F$ является изоморфизмом тогда
и только тогда, когда $\Ker(\varphi)=\{e_G\}$ и $\Im(\varphi)=F$.
\end{corollary}

\begin{proposition}
Пусть $\varphi \colon G \to F$~--- гомоморфизм групп. Тогда
подгруппа $\Ker(\varphi)$ нормальна в~$G$.
\end{proposition}
\vspace{-3mm}
\begin{proof}
Достаточно проверить, что $g^{-1}hg \in \Ker(\varphi)$ для любых
$g\in G$ и $h \in \Ker(\varphi)$. Это следует из цепочки равенств
$$
\varphi(g^{-1}hg) = \varphi(g^{-1}) \varphi(h) \varphi(g) =
\varphi(g^{-1}) e_F \varphi(g) = \varphi(g^{-1}) \varphi(g) =
\varphi(g)^{-1} \varphi (g) = e_F.
$$
\end{proof}

{\bf Теорема о гомоморфизме}. Пусть $\varphi\colon G\to F$~---
гомоморфизм групп. Тогда группа $\Im(\varphi)$ изоморфна
факторгруппе $G/\Ker(\varphi)$.

\begin{proof}
Рассмотрим отображение $\psi \colon G / \Ker(\varphi) \to F$,
заданное формулой $\psi(g\Ker(\varphi)) = \varphi(g)$. Проверка
корректности: равенство $\varphi(gh_1)=\varphi(gh_2)$ для любых
$h_1,h_2\in\Ker(\varphi)$ следует из цепочки
$$
\varphi(gh_1)=\varphi(g)\varphi(h_1)=\varphi(g)=\varphi(g)\varphi(h_2)=\varphi(gh_2).
$$
Отображение $\psi$ сюръективно по построению и инъективно в силу
того, что $\varphi(g) = e_F$ тогда и только тогда, когда $g \in \Ker
(\varphi)$ (т.\,е. $g\Ker(\varphi) = \Ker(\varphi)$). Остаётся
проверить, что $\psi$~--- гомоморфизм:
$$
\psi((g\Ker(\varphi))(g'\Ker(\varphi))) = \psi(gg'\Ker(\varphi)) =
\varphi(gg') = \varphi(g)\varphi(g') =
\psi(g\Ker(\varphi))\psi(g'\Ker(\varphi)).
$$
\end{proof}

Тем самым, чтобы удобно реализовать факторгруппу $G/H$, можно найти
такой гомоморфизм $\varphi\colon G\to F$ в некоторую группу~$F$, что
$H = \Ker(\varphi)$, и тогда $G/H \cong \Im(\varphi)$.

\begin{example}
Пусть $G=(\RR,+)$ и $H=(\ZZ,+)$. Рассмотрим группу
$F=(\CC\setminus\{0\},\times)$ и гомоморфизм
$$
\varphi\colon G\to F, \quad a\mapsto e^{2\pi\imath a} = \cos (2\pi
a) + i \sin (2\pi a).
$$
Тогда $\Ker(\varphi)=H$ и факторгруппа $G/H$ изоморфна окружности
$S^1$, рассматриваемой как подгруппа в~$F$, состоящая из комплексных
чисел с модулем~$1$.
\end{example}

\begin{definition}
{\it Центр} группы $G$~--- это подмножество
$$
Z(G)=\{a\in G \mid ab=ba \ \text{для всех} \ b\in G\}.
$$
\end{definition}

Ясно, что группа $G$ абелева тогда и только тогда, когда $Z(G)=G$.

\begin{proposition}
Центр $Z(G)$ является нормальной подгруппой группы $G$.
\end{proposition}

\begin{proof}
Сначала докажем, что $Z(G)$~--- подгруппа в~$G$. Для этого надо
показать, что $ab^{-1}\in Z(G)$ для любых $a, b \in Z(G)$. В~самом
деле, для произвольного элемента $g \in G$ имеем
$$
ab^{-1}g = ab^{-1}(g^{-1})^{-1} = a (g^{-1}b)^{-1} = a(bg^{-1})^{-1}
= a (g^{-1})^{-1}b^{-1} = agb^{-1} = gab^{-1}.
$$
Далее, если $a \in Z(G)$ и $g\in G$, то
$$
g^{-1}agb=g^{-1}gab=ab=ba=bag^{-1}g=bg^{-1}ag
$$
для всех $b\in G$. Значит, $g^{-1}ag\in Z(G)$ и подгруппа $Z(G)$
нормальна.
\end{proof}

Определим ещё одну важную конструкцию, позволяющую строить новые
группы из имеющихся.

\begin{definition}
{\it Прямым произведением} групп $G_1, \ldots, G_m$ называется
множество
$$
G_1\times\ldots\times G_m=\{(g_1,\ldots,g_m) \mid g_1\in G_1,\ldots,
g_m\in G_m\}
$$
с операцией
$(g_1,\ldots,g_m)(g_1',\ldots,g_m')=(g_1g_1',\ldots,g_mg_m')$.
\end{definition}

Ясно, что эта операция ассоциативна, обладает нейтральным элементом
$(e_{G_1},\ldots,e_{G_m})$ и для каждого элемента $(g_1,\ldots,g_m)$
есть обратный элемент $(g_1^{-1},\ldots,g_m^{-1})$.

\begin{remark}
Группа $G_1\times\ldots\times G_m$ коммутативна в точности тогда,
когда коммутативна каждая из групп $G_1,\ldots, G_m$.
\end{remark}

\begin{remark}
Если все группы $G_1, \ldots, G_m$ конечны, то $|G_1 \times \ldots
\times G_m| = |G_1| \cdot \ldots \cdot |G_m|$.
\end{remark}

Следующий результат связывает конструкции факторгруппы и прямого
произведения.

{\bf Теорема о факторизации по сомножителям}. \ Пусть $H_1, \ldots,
H_m$~--- нормальные подгруппы в группах $G_1, \ldots, G_m$
соответственно. Тогда $H_1 \times \ldots \times H_m$~--- нормальная
подгруппа в $G_1 \times \ldots \times G_m$ и имеет место изоморфизм
групп
$$
(G_1 \times \ldots \times G_m) / (H_1 \times \ldots \times H_m)
\cong G_1 / H_1 \times \ldots \times G_m / H_m.
$$

\begin{proof}
Прямая проверка показывает, что $H_1\times\ldots\times H_m$~---
нормальная подгруппа в $G_1\times\ldots\times G_m$. Требуемый
изоморфизм устанавливается отображением
$$
(g_1,\ldots,g_m)(H_1\times\ldots\times H_m)\mapsto
(g_1H_1,\ldots,g_mH_m).
$$
\end{proof}

\begin{theorem}
Пусть $n=ml$~--- разложение натурального числа $n$ на два взаимно
простых множителя. Тогда имеет место изоморфизм групп
$$
\ZZ_n\cong \ZZ_m\times\ZZ_l.
$$
\end{theorem}

\begin{proof}
Рассмотрим отображение
$$
\varphi\colon \ZZ_n\to \ZZ_m\times\ZZ_l, \quad k \ (\text{mod}\ n)
\mapsto (k\ (\text{mod}\ m), k\ (\text{mod}\  l)).
$$
Поскольку $m$ и $l$ делят~$n$, отображение $\varphi$ определено
корректно. Ясно, что $\varphi$~--- гомоморфизм. Далее, если $k$
переходит в нейтральный элемент $(0,0)$, то $k$ делится и на $m$, и
на $l$, а~значит, делится на $n$ в силу взаимной простоты $m$ и~$l$.
Отсюда следует, что гомоморфизм $\varphi$ инъективен. Поскольку
множества $\ZZ_n$ и $\ZZ_m\times\ZZ_l$ содержат одинаковое число
элементов, отображение $\varphi$ биективно.
\end{proof}

\begin{corollary} \label{corpr}
Пусть $n \geqslant 2$~--- натуральное число и $n=p_1^{k_1}\ldots
p_s^{k_s}$~--- его разложение в произведение простых множителей
\textup(где $p_i \ne p_j$ при $i \ne j$\textup). Тогда имеет место
изоморфизм групп
$$
\ZZ_n\cong\ZZ_{p_1^{k_1}}\times\ldots\times\ZZ_{p_s^{k_s}}.
$$
\end{corollary}


\bigskip

\begin{thebibliography}{99}
\bibitem{Vi}
Э.\,Б.~Винберг. Курс алгебры. М.: Факториал Пресс, 2002 (глава~4,
$\S$~6 и глава~10, $\S$~1)
\bibitem{Ko1}
А.\,И.~Кострикин. Введение в алгебру. Основы алгебры. М.: Наука.
Физматлит, 1994 (глава~4, $\S$~2)
\bibitem{Ko3}
А.\,И.~Кострикин. Введение в алгебру. Основные структуры алгебры.
М.: Наука. Физматлит, 2000 (глава~1, $\S$~4)
\bibitem{SZ}
Сборник задач по алгебре под редакцией А.\,И.~Кострикина. Новое
издание. М.: МЦНМО, 2009 (глава~13, $\S$~58, 60)
\end{thebibliography}


\end{document}
