\documentclass[a4paper,10pt]{amsart}

\usepackage[T2A]{fontenc}
\usepackage[utf8x]{inputenc}
\usepackage{amssymb}
\usepackage[russian]{babel}
\usepackage{geometry}
\usepackage{hyperref}

\geometry{a4paper,top=2cm,bottom=2cm,left=2cm,right=2cm}

\setlength{\parindent}{0pt}
\setlength{\parskip}{\medskipamount}

\newcommand{\Ker}{\mathop{\mathrm{Ker}}}
\renewcommand{\Im}{\mathop{\mathrm{Im}}}
\DeclareMathOperator{\Tor}{\mathrm{Tor}}
%\newcommand{\Tor}{\mathop{\mathrm{Tor}}}

%\def\Ker{{\rm Ker}}%
%\def\Im{{\rm Im}}%
\def\Mat{{\rm Mat}}%
\def\cont{{\rm cont}}%
%\def\Tor{{\rm Tor}}%
\def\Char{{\rm Char}}%
\def\signum{{\rm sig}}%
\def\Sym{{\rm Sym}}%
\def\St{{\rm St}}%
\def\Aut{{\rm Aut}}%
\def\Chi{{\mathbb X}}%
\def\Tau{{\rm T}}%
\def\Rho{{\rm R}}%
\def\rk{{\rm rk}}%
\def\ggT{{\rm ggT}}%
\def\kgV{{\rm kgV}}%
\def\Div{{\rm Div}}%
\def\div{{\rm div}}%
\def\quot{/\!\!/}%
\def\mal{\! \cdot \!}%
\def\Of{{\mathcal{O}}}
%
\def\subgrpneq{\le}%
\def\subgrp{\le}%
\def\ideal#1{\le_{#1}}%
\def\submod#1{\le_{#1}}%
%
\def\Bild{{\rm Bild}}%
\def\Kern{{\rm Kern}}%
\def\bangle#1{{\langle #1 \rangle}}%
\def\rq#1{\widehat{#1}}%
\def\t#1{\widetilde{#1}}%
\def\b#1{\overline{#1}}%
%
\def\abs#1{{\vert #1 \vert}}%
\def\norm#1#2{{\Vert #1 \Vert}_{#2}}%
\def\PS#1#2{{\sum_{\nu=0}^{\infty} #1_{\nu} #2^{\nu}}}%
%
\def\C{{\rm C}}%
\def\O{{\rm O}}%
\def\HH{{\mathbb H}}%
\def\LL{{\mathbb L}}%
\def\FF{{\mathbb F}}%
\def\CC{{\mathbb C}}%
\def\KK{{\mathbb K}}%
\def\TT{{\mathbb T}}%
\def\ZZ{{\mathbb Z}}%
\def\RR{{\mathbb R}}%
\def\SS{{\mathbb S}}%
\def\NN{{\mathbb N}}%
\def\QQ{{\mathbb Q}}%
\def\PP{{\mathbb P}}%
\def\AA{{\mathbb A}}%
%
\def\eins{{\mathbf 1}}%
%
\def\AG{{\rm AG}}%
\def\Aut{{\rm Aut}}%
\def\Hol{{\rm Hol}}%
\def\GL{{\rm GL}}%
\def\SL{{\rm SL}}%
\def\SO{{\rm SO}}%
\def\Sp{{\rm Sp}}%
\def\gl{\mathfrak{gl}}%
\def\rg{{\rm rg}}%
\def\sl{\mathfrak{sl}}%
\def\HDiv{{\rm HDiv}}%
\def\CDiv{{\rm CDiv}}%
\def\Res{{\rm Res}}%
\def\Pst{{\rm Pst}}%
\def\Nst{{\rm Nst}}%
\def\rad{{\rm rad}}%
\def\GL{{\rm GL}}%
\def\Tr{{\rm Tr}}%
\def\Pic{{\rm Pic}}%
\def\Hom{{\rm Hom}}%
\def\hom{{\rm hom}}%
\def\Mor{{\rm Mor}}%
\def\codim{{\rm codim}}%
\def\Supp{{\rm Supp}}%
\def\Spec{{\rm Spec}}%
\def\Proj{{\rm Proj}}%
\def\Maps{{\rm Maps}}%
\def\cone{{\rm cone}}%
\def\ord{{\rm ord}}%
\def\pr{{\rm pr}}%
\def\id{{\rm id}}%
\def\mult{{\rm mult}}%
\def\inv{{\rm inv}}%
\def\neut{{\rm neut}}%
%
\def\AAA{\mathcal{A}}
\def\BBB{\mathcal{B}}
\def\CCC{\mathcal{C}}
\def\EEE{\mathcal{E}}
\def\FFF{\mathcal{F}}

\def\CF{{\rm CF}}
\def\GCD{{\rm GCD}}
\def\Mat{{\rm Mat}}
\def\End{{\rm End}}
\def\cont{{\rm cont}}
\def\Kegel{{\rm Kegel}}
\def\Char{{\rm Char}}
\def\Der{{\rm Der}}
\def\signum{{\rm sg}}
\def\grad{{\rm grad}}
\def\Spur{{\rm Spur}}
\def\Sym{{\rm Sym}}
\def\Alt{{\rm Alt}}
\def\Abb{{\rm Abb}}
\def\Chi{{\mathbb X}}
\def\Tau{{\rm T}}
\def\Rho{{\rm R}}
\def\ad{{\rm ad}}
\def\Frob{{\rm Frob}}
\def\Rang{{\rm Rang}}
\def\SpRang{{\rm SpRang}}
\def\ZRang{{\rm ZRang}}
\def\ggT{{\rm ggT}}
\def\kgV{{\rm kgV}}
\def\Div{{\rm Div}}
\def\div{{\rm div}}
\def\quot{/\!\!/}
\def\mal{\! \cdot \!}
\def\add{{\rm add}}
\def\mult{{\rm mult}}
\def\smult{{\rm smult}}

\def\subgrpneq{\le}
\def\subgrp{\le}
\def\ideal#1{\unlhd_{#1}}
\def\submod#1{\le_{#1}}

\def\Bild{{\rm Bild}}
\def\Kern{{\rm Kern}}
\def\Kon{{\rm Kon}}
\def\bangle#1{{\langle #1 \rangle}}
\def\rq#1{\widehat{#1}}
\def\t#1{\widetilde{#1}}
\def\b#1{\overline{#1}}

\def\abs#1{{\vert #1 \vert}}
\def\norm#1#2{{\Vert #1 \Vert}_{#2}}
\def\PS#1#2{{\sum_{\nu=0}^{\infty} #1_{\nu} #2^{\nu}}}


\def\eins{{\mathbf 1}}

\def\ElM{{\rm ElM}}
\def\ZOp{{\rm ZOp}}
\def\SpOp{{\rm SpOp}}
\def\Gal{{\rm Gal}}
\def\Def{{\rm Def}}
\def\Fix{{\rm Fix}}
\def\ord{{\rm ord}}
\def\Aut{{\rm Aut}}
\def\Hol{{\rm Hol}}
\def\GL{{\rm GL}}
\def\SL{{\rm SL}}
\def\SO{{\rm SO}}
\def\Sp{{\rm Sp}}
\def\Spann{{\rm Spann}}
\def\Lin{{\rm Lin}}
\def\gl{\mathfrak{gl}}
\def\rg{{\rm rg}}
\def\sl{\mathfrak{sl}}
\def\so{\mathfrak{so}}
\def\sp{\mathfrak{sp}}
\def\gg{\mathfrak{g}}
\def\HDiv{{\rm HDiv}}
\def\CDiv{{\rm CDiv}}
\def\Res{{\rm Res}}
\def\Pst{{\rm Pst}}
\def\Nst{{\rm Nst}}
\def\WDiv{{\rm WDiv}}
\def\GL{{\rm GL}}
\def\Tr{{\rm Tr}}
\def\Pic{{\rm Pic}}
\def\Hom{{\rm Hom}}
\def\hom{{\rm hom}}
\def\Mor{{\rm Mor}}
\def\codim{{\rm codim}}
\def\Supp{{\rm Supp}}
\def\Spec{{\rm Spec}}
\def\Proj{{\rm Proj}}
\def\Maps{{\rm Maps}}
\def\cone{{\rm cone}}
\def\ord{{\rm ord}}
\def\pr{{\rm pr}}
\def\id{{\rm id}}
\def\mult{{\rm mult}}
\def\inv{{\rm inv}}
\def\neut{{\rm neut}}
\def\trdeg{{\rm trdeg}}
\def\sing{{\rm sing}}
\def\reg{{\rm reg}}


%%%%%%%%%%%%%%%%%%%%%%%%%%%

\newtheorem{theorem}{Теорема}
\newtheorem{proposition}{Предложение}
\newtheorem{lemma}{Лемма}
\newtheorem{corollary}{Следствие}
\theoremstyle{definition}
\newtheorem{definition}{Определение}
\newtheorem{problem}{Задача}
%
\theoremstyle{remark}
\newtheorem{exc}{Упражнение}
\newtheorem{remark}{Замечание}
\newtheorem{example}{Пример}

\renewcommand{\theenumi}{\textup{(\alph{enumi})}}
\renewcommand{\labelenumi}{\theenumi}
\newcounter{property}
\renewcommand{\theproperty}{\textup{(\arabic{property})}}
\newcommand{\property}{\refstepcounter{property}\item}
\newcounter{prooperty}
\renewcommand{\theprooperty}{\textup{(\arabic{prooperty})}}
\newcommand{\prooperty}{\refstepcounter{prooperty}\item}

\makeatletter
\def\keywords#1{{\def\@thefnmark{\relax}\@footnotetext{#1}}}
\let\subjclass\keywords
\makeatother
%
\begin{document}
%
\sloppy
%\thispagestyle{empty}
%
\centerline{\large \bf Лекции курса \guillemotleft
Алгебра\guillemotright{}, лектор Р.\,С.~Авдеев}

\smallskip

\centerline{\large ФКН НИУ ВШЭ, 1-й курс ОП ПМИ, 4-й модуль,
2015/2016 учебный год}


\bigskip

\section*{Лекция 4}

\medskip

{\it Теорема о согласованных базисах. Алгоритм
приведения целочисленной матрицы к диагональному виду. Строение конечно порождённых абелевых групп. Конечные абелевы группы. } % Экспонента конечной абелевой группы.}

В~теории абелевых групп операция прямого произведения конечного
числа групп обычно называется \textit{прямой суммой} и обозначается
символом~$\oplus$, так что пишут $A_1 \oplus A_2 \oplus \ldots
\oplus A_n$ вместо $A_1 \times A_2 \times \ldots \times A_n$.

Дадим более точное описание подгрупп свободных абелевых групп.

\smallskip

{\bf Теорема о согласованных базисах.}\ Для всякой подгруппы $N$
свободной абелевой группы $L$ ранга $n$ найдётся такой базис $e_1,
\ldots, e_n$ группы $L$ и такие натуральные числа $u_1, \ldots,
u_m$, $m \leqslant n$, что $u_1 e_1, \ldots, u_m e_m$~--- базис
группы $N$ и $u_i | u_{i+1}$ при $i = 1, \ldots, m-1$.

\smallskip

\begin{remark}
Числа $u_1, \ldots, u_p$, фигурирующие в теореме о согласованных
базисах, называются {\it инвариантными множителями} подгруппы $N
\subseteq L$. Можно показать, что они определены по подгруппе
однозначно.
\end{remark}

\begin{corollary}
В~условиях теоремы о согласованных базисах имеет место изоморфизм
$$
L / N \cong \ZZ_{u_1} \times \ldots \times \ZZ_{u_m} \times
\underbrace{\ZZ \times \ldots \times \ZZ}_{n - m}.
$$
\end{corollary}

\begin{proof}
Рассмотрим изоморфизм $L \cong \ZZ^n = \underbrace{\ZZ \times \ldots
\times \ZZ}_n$, сопоставляющий произвольному элементу $s_1 e_1 +
\ldots + s_n e_n \in L$ набор $(s_1, \ldots, s_n) \in \ZZ^n$. При
этом изоморфизме подгруппа $N \subseteq L$ отождествляется с
подгруппой
$$
u_1 \ZZ \times \ldots \times u_m \ZZ \times \underbrace{\lbrace 0
\rbrace \times \ldots \times \lbrace 0 \rbrace}_{n-m} \subseteq
\ZZ^n.
$$
Теперь требуемый результат получается применением теоремы о
факторизации по сомножителям.
\end{proof}

Теперь вернемся к доказательству теоремы о согласованных базисов.
Однако это требует некоторой подготовки.

\begin{definition}
{\it Целочисленными элементарными преобразованиями строк} матрицы
называются преобразования следующих трёх типов:

1) прибавление к одной строке другой, умноженной на целое число;

2) перестановка двух строк;

3) умножение одной строки на $-1$.

Аналогично определяются {\it целочисленные элементарные
преобразования столбцов} матрицы.
\end{definition}

Прямоугольную матрицу $C=(c_{ij})$ размера $n\times m$ назовём {\it
диагональной} и обозначим $\text{diag}(u_1,\ldots,u_p)$, если
$c_{ij}=0$ при $i\ne j$ и $c_{ii}=u_i$ при $i=1,\ldots,p$, где
$p=\text{min}(n,m)$.

\begin{proposition} \label{palg}
Всякую прямоугольную целочисленную матрицу $C=(c_{ij})$ с помощью
элементарных преобразований строк и столбцов можно привести к виду
$\text{diag}(u_1,\ldots,u_p)$, где $u_1,\ldots,u_p \geqslant 0$ и
$u_i|u_{i+1}$ при $i=1,\ldots,p-1$.
\end{proposition}

\begin{proof}
Если $C=0$, то доказывать нечего. Если $C\ne 0$, но $c_{11}=0$, то
переставим строки и столбцы и получим $c_{11}\ne 0$. Умножив, если
нужно, первую строку на $-1$, добьёмся условия $c_{11}>0$. Теперь
будем стремиться уменьшить~$c_{11}$.

Если какой-то элемент $c_{i1}$ не делится на $c_{11}$, то разделим с
остатком: $c_{i1}=qc_{11}+r$. Вычитая из $i$-й строки $1$-ю строку,
умноженную на~$q$, и затем переставляя $1$-ю и $i$-ю строки,
уменьшаем~$c_{11}$. Повторяя эту процедуру, в~итоге добиваемся, что
все элементы $1$-й строки и $1$-го столбца делятся на $c_{11}$.

Если какой-то $c_{ij}$ не делится на $c_{11}$, то поступаем
следующим образом. Вычтя из $i$-й строки $1$-ю строку с подходящим
коэффициентом, добьёмся $c_{i1}=0$. После этого прибавим к $1$-й
строке $i$-ю строку. При этом $c_{11}$ не изменится, а $c_{1j}$
перестанет делиться на $c_{11}$, и мы вновь сможем уменьшить
$c_{11}$.

В~итоге добьёмся того, что все элементы делятся на~$c_{11}$. После
этого обнулим все элементы $1$-й строки и $1$-го столбца, начиная со
вторых, и продолжим процесс с меньшей матрицей.
\end{proof}

Теперь мы готовы доказать теорему о согласованных базисах.

\begin{proof}[Доказательство теоремы о согласованных базисах]
Мы знаем, что $N$ является свободной абелевой группой ранга $m
\leqslant n$. Пусть $e_1, \ldots, e_n$~--- базис в $L$ и $f_1,
\ldots, f_m$~--- базис в~$N$. Тогда $(f_1, \ldots, f_m) = (e_1,
\ldots, e_n)C$, где $C$~--- целочисленная матрица размера $n \times
m$ и ранга~$m$. Покажем, что целочисленные элементарные
преобразования строк (столбцов) матрицы $C$~--- это в точности
элементарные преобразования над базисом в~$L$ (в~$N$). Для этого
рассмотрим сначала случай строк. Заметим, что каждое из
целочисленных элементарных преобразований строк реализуется при
помощи умножения матрицы $C$ слева на квадратную матрицу~$P$
порядка~$n$, определяемую следующим образом:

(1) в случае прибавления к $i$-й строке $j$-й, умноженной на целое
число~$z$, в матрице~$P$ на диагонали стоят единицы, на $(ij)$-м
месте~--- число~$z$, а на остальных местах~--- нули;

(2) в случае перестановки $i$-й и $j$-й строк имеем $p_{ij} = p_{ji}
= 1$, $p_{kk} = 1$ при $k \ne i,j$, а на остальных местах стоят
нули;

(3) в случае умножения $i$-й строки на $-1$ имеем $p_{ii} = -1$,
$p_{jj} = 1$ при $j \ne i$, а на остальных местах стоят нули.

Теперь заметим, что равенство $(f_1, \ldots, f_m) = (e_1, \ldots,
e_n)C$ эквивалентно равенству $(f_1, \ldots, f_m) = (e_1, \ldots,
e_n)P^{-1} PC$. Таким образом, базис $(f_1, \ldots, f_m)$ выражается
через новый базис $(e'_1, \ldots, e'_n) := (e_1, \ldots, e_n)P^{-1}$
при помощи матрицы~$PC$.

В случае столбцов всё аналогично: каждое из целочисленых
элементарных преобразований столбцов реализуется при помощи
умножения матрицы $C$ справа на некоторую квадратную матрицу $Q$
порядка~$m$ (определяемую почти так же, как~$P$). В~этом случае
имеем $(f_1, \ldots, f_m)Q = (e_1, \ldots, e_n)CQ$, так что новый
базис $(f'_1, \ldots, f'_m) := (f_1, \ldots, f_m)Q$ выражается через
$(e_1, \ldots, e_n)$ при помощи матрицы $CQ$.

Воспользовавшись предложением~\ref{palg}, мы можем привести матрицу
$C$ при помощи целочисленных элементарных преобразований строк и
столбцов к диагональному виду~$C'' = \text{diag}(u_1, \ldots, u_m)$,
где $u_i | u_{i+1}$ для всех $i = 1, \ldots, m-1$. С~учётом
сказанного выше это означает, что для некоторого базиса $e''_1,
\ldots, e''_n$ в~$L$ и некоторого базиса $f''_1, \ldots, f''_m$
в~$N$ справедливо соотношение $(f''_1, \ldots, f''_m) = (e''_1,
\ldots, e''_n) C''$. Иными словами, $f''_i = u_i e''_i$ для всех $i
= 1, \ldots, m$, а~это и требовалось.
\end{proof}

\begin{definition}
Конечная абелева группа $A$ называется {\it примарной}, если её
порядок равен $p^k$ для некоторого простого числа~$p$.
\end{definition}

\begin{remark}
В общем случае (когда группы не предполагаются коммутативными)
конечная группа $G$ с~условием $|G| = p^k$ ($p$~--- простое)
называется {\it $p$-группой}.
\end{remark}

Следствие~1 лекции~3 показывает, что каждая конечная циклическая
группа разлагается в прямую сумму примарных циклических подгрупп.

\begin{theorem} \label{traz}
Всякая конечно порождённая абелева группа $A$ разлагается в прямую
сумму примарных и бесконечных циклических подгрупп, т.\,е.
\begin{equation} \label{eqn}
A \cong \ZZ_{p_1^{k_1}} \oplus \ldots \oplus \ZZ_{p_s^{k_s}} \oplus
\ZZ \oplus \ldots \oplus \ZZ,
\end{equation}
где $p_1, \ldots, p_s$~--- простые числа \textup(не обязательно
попарно различные\textup) и $k_1, \ldots, k_s \in \NN$. Кроме того,
число бесконечных циклических слагаемых, а~также число и порядки
примарных циклических слагаемых определено однозначно.
\end{theorem}

Сразу выделим некоторые следствия из этой теоремы.

\begin{corollary}
Абелева группа $A$ является конечно порождённой тогда и только тогда, когда $A$ разлагается в прямую сумму циклических подгрупп. 
\end{corollary}

\begin{proof}
В одну сторону следует из теоремы. В другую сторону: пусть $A = A_1 \oplus \ldots \oplus A_m$, где $A_i$~--- циклическая подгруппа, то есть $A_i = \langle a_i \rangle$, $a_i \in A$. Тогда $\{a_1, \ldots, a_m \}$ --- набор порождающих элементов для группы $A$.
\end{proof}

\begin{corollary}
Всякая конечная абелева группа разлагается в прямую сумму примарных
циклических подгрупп, причём число и порядки примарных циклических
слагаемых определено однозначно.
\end{corollary}

Теперь преступим к доказательству самой теоремы.

\begin{proof}
Пусть $a_1,\ldots,a_n$~--- конечная система порождающих группы $A$.
Рассмотрим гомоморфизм
$$
\varphi \colon \ZZ^n \to A, \quad (s_1, \ldots, s_n) \mapsto s_1 a_1
+ \ldots + s_n a_n.
$$
Ясно, что $\varphi$ сюръективен. Тогда по теореме о гомоморфизме
получаем $A \cong \ZZ^n / N$, где $N = \Ker \varphi$. По теореме о
согласованных базисах существует такой базис $e_1, \ldots, e_n$
группы $\ZZ^n$ и такие натуральные числа $u_1, \ldots, u_m$, $m
\leqslant n$, что $u_1 e_1, \ldots, u_m e_m$~--- базис группы~$N$.
Тогда имеем
$$
\begin{array}{ccccccccccccc}
L &=& \langle e_1 \rangle &\oplus & \ldots & \oplus & \langle e_m
\rangle & \oplus & \langle e_{m+1} \rangle & \oplus & \ldots &
\oplus & \langle e_n \rangle, \\
N &=& \langle u_1e_1 \rangle & \oplus & \ldots & \oplus & \langle
u_m e_m \rangle &\oplus & \lbrace 0 \rbrace & \oplus & \ldots &
\oplus & \lbrace 0 \rbrace.
\end{array}
$$
Применяя теорему о факторизации по сомножителям, мы получаем
$$
\ZZ^n / N \cong \ZZ / u_1 \ZZ \oplus \ldots \oplus \ZZ / u_m \ZZ
\oplus \underbrace{\ZZ / \lbrace 0 \rbrace \oplus \ldots \oplus \ZZ
/ \lbrace 0 \rbrace}_{n-m} \cong \ZZ_{u_1} \oplus \ldots \oplus
\ZZ_{u_m} \oplus \underbrace{\ZZ \oplus \ldots \oplus \ZZ}_{n-m}.
$$
Чтобы добиться разложения~(\ref{eqn}), остаётся представить каждое
из циклических слагаемых $\ZZ_{u_i}$ в виде прямой суммы примарных
циклических подгрупп, воспользовавшись следствием~1 из лекции~3.

Перейдём к доказательству единственности разложения~(\ref{eqn}).
Пусть $\langle c \rangle_q$ обозначает циклическую группу порядка
$q$ с порождающей~$c$. Пусть имеется разложение
\begin{equation} \label{eqn2}
A = \langle c_1\rangle_{p_1^{k_1}} \oplus \ldots \oplus \langle c_s
\rangle_{p_s^{k_s}} \oplus \langle c_{s+1} \rangle_{\infty} \oplus
\ldots \oplus \langle c_{s+t} \rangle_{\infty}
\end{equation}
(заметьте, что мы просто переписали в другом виде правую часть
соотношения~(\ref{eqn})). Рассмотрим в~$A$ так называемую {\it
подгруппу кручения}
$$
\Tor A := \{ a \in A \mid ma=0 \ \text{для некоторого} \ m \in \NN
\}.
$$
Иными словами, $\Tor A$~--- это подгруппа в~$A$, состоящая из всех
элементов конечного порядка. Выделим эту подгруппу в
разложении~(\ref{eqn2}). Рассмотрим произвольный элемент $a \in A$.
Он представим в виде
$$
a = r_1c_1 + \ldots + r_m c_m + r_{m+1} c_{m+1} + \ldots + r_n c_n
$$
для некоторых целых чисел $r_1, \ldots, r_n$. Легко видеть, что $a$
имеет конечный порядок тогда и только тогда, когда $r_{m+1} = \ldots
= r_m = 0$. Отсюда получаем, что
\begin{equation} \label{eqn3}
\Tor A = \langle c_1 \rangle_{p_1^{k_1}} \oplus \ldots \oplus
\langle c_s \rangle_{p_s^{k_s}}.
\end{equation}
Применяя опять теорему о факторизации по сомножителям, мы получаем
$A / \Tor A \cong \ZZ^t$, где $t$ --- количество бесконечных 
циклических подгрупп в разложении~(\ref{eqn}). Отсюда следует, 
что число $t$ однозначно выражается в терминах самой группы~$A$ 
(как ранг свободной абелевой группы $A / \Tor A$). Значит, $t$ 
не зависит от разложения~(\ref{eqn2}).

Однозначность числа и порядков примарных циклических
групп будет доказана на следующей лекции.

\end{proof}

%Далее, для каждого простого числа $p$ определим в $A$ {\it подгруппу
%$p$-кручения}
%\begin{equation} \label{eqn4}
%\Tor_p A := \{ a\in A \mid p^ka=0 \ \text{для некоторого} \ k \in
%\NN \}.
%\end{equation}
%Ясно, что $\Tor_p A \subset \Tor A$. Выделим подгруппу $\Tor_p A$ в
%разложении~(\ref{eqn3}). Легко видеть, что $\langle c_i
%\rangle_{p_i^{k_i}} \subseteq \Tor_p A$ для всех $i$ с условием $p_i
%= p$. Если же $p_i \ne p$, то по следствию~2 из теоремы Лагранжа
%(см. лекцию~1) порядок любого ненулевого элемента $x \in \langle c_i
%\rangle_{p_i^{k_i}}$ является степенью числа~$p_i$, а~значит, $p^k x
%\ne 0$ для всех $k \in \NN$. Отсюда следует, что $\Tor_p A$ является
%суммой тех конечных слагаемых в разложении~(\ref{eqn3}), порядки
%которых суть степени~$p$. Поэтому доказательство теперь сводится к
%случаю, когда $A$~--- примарная группа.
%
%Пусть $|A|=p^k$ и
%$$
%A = \langle c_1\rangle_{p^{k_1}}\oplus\ldots\oplus\langle
%c_r\rangle_{p^{k_r}}, \quad k_1+\ldots+k_r=k.
%$$
%Докажем индукцией по~$k$, что набор чисел $k_1, \ldots, k_r$ не
%зависит от разложения.
%
%Если $k = 1$, то $|A| = p$, но тогда $A \cong \ZZ_p$ по следствию~5
%из теоремы Лагранжа (см. лекцию~1). Пусть теперь $k > 1$. Рассмотрим
%подгруппу $pA: = \{ pa \mid a \in A \}$. В~терминах
%равенства~(\ref{eqn4}) имеем
%$$
%pA = \langle pc_1 \rangle_{p^{k_1-1}} \oplus \ldots \oplus \langle
%pc_r\rangle_{p^{k_r-1}}.
%$$
%В частности, при $k_i = 1$ соответствующее слагаемое равно $\lbrace
%0 \rbrace$ (и тем самым исчезает). Так как $|pA| = p^{k - r} < p^k$,
%то по предположению индукции группа $pA$ разлагается в прямую сумму
%примарных циклических подгрупп однозначно с точностью до порядка
%слагаемых. Следовательно, ненулевые числа в наборе $k_1 - 1, \ldots,
%k_r-1$ определены однозначно (с точностью до перестановки). Отсюда
%мы находим значения $k_i$, отличные от~$1$. Количество тех~$k_i$,
%которые равны~$1$, однозначно восстанавливается из условия $k_1 +
%\ldots + k_r = k$.
%\end{proof}

%Заметим, что теорема о согласованных базисах даёт нам другое
%разложение конечной абелевой группы~$A$:
%\begin{equation} \label{eqn5}
%A=\ZZ_{u_1}\oplus\ldots\oplus\ZZ_{u_m}, \quad \text{где} \
%u_i|u_{i+1} \ \text{при} \ i = 1, \ldots, m-1.
%\end{equation}
%Числа $u_1, \ldots, u_m$ называют {\it инвариантными множителями}
%конечной абелевой группы~$A$.
%
%\begin{definition}
%{\it Экспонентой} конечной абелевой группы $A$ называется число
%$\exp A$, равное наименьшему общему кратному порядков элементов
%из~$A$.
%\end{definition}
%
%\begin{remark}
%Легко видеть, что $\exp A = \min \lbrace n \in \NN \mid ma = 0 \
%\text{для всех} \ a \in A \rbrace$.
%\end{remark}
%
%\begin{proposition}
%Экспонента конечной абелевой группы~$A$ равна её последнему
%инвариантному множителю~$u_m$.
%\end{proposition}
%
%\begin{proof}
%Обратимся к разложению~(\ref{eqn5}). Так как $u_i | u_m$ для всех $i
%= 1, \ldots, m$, то $u_ma=0$ для всех $a \in A$. Это означает, что
%$\exp A \leqslant u_m$ (и тем самым $\exp A \, | u_m$). С~другой
%стороны, в $A$ имеется циклическая подгруппа порядка $u_m$. Значит,
%$\exp A \geqslant u_m$.
%\end{proof}
%
%\begin{corollary}
%Конечная абелева группа $A$ является циклической тогда и только
%тогда, когда $\exp A =\nobreak |A|$.
%\end{corollary}
%
%\begin{proof}
%Группа $A$ является циклической тогда и только тогда, когда в
%разложении~(\ref{eqn5}) присутствует только одно слагаемое, т.\,е.
%$A = \ZZ_{u_m}$ и $|A| = u_m$.
%\end{proof}

\bigskip

\begin{thebibliography}{99}
\bibitem{Vi}
Э.\,Б.~Винберг. Курс алгебры. М.: Факториал Пресс, 2002 (глава~9,
$\S$~1)
\bibitem{Ko3}
А.\,И.~Кострикин. Введение в алгебру. Основные структуры алгебры.
М.: Наука. Физматлит, 2000 (глава~2, $\S$~3)
\bibitem{SZ}
Сборник задач по алгебре под редакцией А.\,И.~Кострикина. Новое
издание. М.: МЦНМО, 2009 (глава~13, $\S$~60)
\end{thebibliography}

\end{document}
