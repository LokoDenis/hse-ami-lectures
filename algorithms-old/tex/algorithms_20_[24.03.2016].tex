\documentclass[a4paper,12pt]{article}

%% Работа с русским языком
\usepackage{cmap}					% поиск в PDF
\usepackage{mathtext} 				% русские буквы в формулах
\usepackage[T2A]{fontenc}			% кодировка
\usepackage[utf8]{inputenc}			% кодировка исходного текста
\usepackage[english,russian]{babel}	% локализация и переносы

%% Отступы между абзацами и в начале абзаца 
\setlength{\parindent}{0pt}
\setlength{\parskip}{\medskipamount}

%% Изменяем размер полей
\usepackage[top=0.5in, bottom=0.75in, left=0.625in, right=0.625in]{geometry}

%% Графика
\usepackage[pdftex]{graphicx}
\graphicspath{{images/}}

%% Различные пакеты для работы с математикой
\usepackage{mathtools}				% Тот же amsmath, только с некоторыми поправками

\usepackage{amssymb}				% Математические символы
\usepackage{amsthm}					% Пакет для написания теорем
\usepackage{amstext}
\usepackage{array}
\usepackage{amsfonts}
\usepackage{icomma}					% "Умная" запятая: $0,2$ --- число, $0, 2$ --- перечисление
\usepackage{bbm}				    % Для красивого (!) \mathbb с  буквами и цифрами
\usepackage{enumitem}               % Для выравнивания itemise (\begin{itemize}[align=left])

% Номера формул
\mathtoolsset{showonlyrefs=true} % Показывать номера только у тех формул, на которые есть \eqref{} в тексте.

% Ссылки
\usepackage[colorlinks=true, urlcolor=blue]{hyperref}

% Шрифты
\usepackage{euscript}	 % Шрифт Евклид
\usepackage{mathrsfs}	 % Красивый матшрифт

% Свои команды\textbf{}
\DeclareMathOperator{\sgn}{\mathop{sgn}}

% Перенос знаков в формулах (по Львовскому)
\newcommand*{\hm}[1]{#1\nobreak\discretionary{}
{\hbox{$\mathsurround=0pt #1$}}{}}

% Графики
\usepackage{tikz}
\usepackage{pgfplots}
%\pgfplotsset{compat=1.12}

% Изменим формат \section и \subsection:
\usepackage{titlesec}
\titleformat{\section}
{\vspace{1cm}\centering\LARGE\bfseries}	% Стиль заголовка
{}										% префикс
{0pt}									% Расстояние между префиксом и заголовком
{} 										% Как отображается префикс
\titleformat{\subsection}				% Аналогично для \subsection
{\Large\bfseries}
{}
{0pt}
{}

% Информация об авторах
\author{Группа лектория ФКН ПМИ 2015-2016 \\
	Анастасия Иовлева \\
	Ксюша Закирова \\
	Руслан Хайдуров}
\title{Лекции по предмету \\
	\textbf{Линейная алгебра и геометрия}}
\date{2016 год}

\newtheorem*{Def}{Определение}
\newtheorem*{Lemma}{Лемма}
\newtheorem*{Suggestion}{Предложение}
\newtheorem*{Examples}{Пример}
\newtheorem*{Comment}{Замечание}
\newtheorem*{Consequence}{Следствие}
\newtheorem*{Theorem}{Теорема}
\newtheorem*{Statement}{Утверждение}
\newtheorem*{Task}{Упражнение}
\newtheorem*{Designation}{Обозначение}
\newtheorem*{Generalization}{Обобщение}
\newtheorem*{Thedream}{Предел мечтаний}
\newtheorem*{Properties}{Свойства}

\renewcommand{\mathbb}{\mathbbm}
\renewcommand{\Re}{\mathrm{Re\:}}
\renewcommand{\Im}{\mathrm{Im\:}}
\newcommand{\Arg}{\mathrm{Arg\:}}
\renewcommand{\arg}{\mathrm{arg\:}}
\newcommand{\Mat}{\mathrm{Mat}}
\newcommand{\id}{\mathrm{id}}
\newcommand{\isom}{\xrightarrow{\sim}} 
\newcommand{\leftisom}{\xleftarrow{\sim}}
\newcommand{\Hom}{\mathrm{Hom}}
\newcommand{\Ker}{\mathrm{Ker}\:}
\newcommand{\rk}{\mathrm{rk}\:}
\newcommand{\diag}{\mathrm{diag}}
\newcommand{\ort}{\mathrm{ort}}
\newcommand{\pr}{\mathrm{pr}}
\newcommand{\vol}{\mathrm{vol\:}}

\renewcommand{\epsilon}{\varepsilon}
\renewcommand{\phi}{\varphi}
\newcommand{\e}{\mathbb{e}}
\renewcommand{\l}{\lambda}
\renewcommand{\C}{\mathbb{C}}
\newcommand{\R}{\mathbb{R}}
\newcommand{\E}{\mathbb{E}}

\newcommand{\vvector}[1]{\begin{pmatrix}{#1}_1 \\\vdots\\{#1}_n\end{pmatrix}}
\renewcommand{\vector}[1]{({#1}_1, \ldots, {#1}_n)}

\begin{document}

\section*{Лекция 20 от 24.03.2016}

\subsection{Вставка красно-чёрные деревья}

В прошлый раз мы получили, что высота красно-чёрного дерева --- $O(\log n)$, где $n$ --- число ключей.

Когда мы вставили новую вершину, мы назначаем ей красный цвет. При этом может нарушиться одно из свойств --- у красного только чёрный родитель.

Править мы будем, используя только две операции:
\begin{enumerate}
    \item Перекрашивание --- смена цвета на противоположный.
    \item Поворот --- за картинкой полезайте в Кормена, в общем.
\end{enumerate}

Вообще, тут куча графов, я не успеваю перерисовывать, я сфоткал, если что.

Сложность такого алгоритма --- $O(\log n)$ --- на изначальную вставку уходит $O(\log n)$; ошибки двигаются вверх не больше чем за два поворота и ?? перекрашиваний --- тоже $O(\log n)$.

\subsection{Самоорганизующийся список}
Представьте, что вы готовитесь к экзамену. У вас есть стопка книг (все в белых обложках) и вы ищете нужную вам, а найдя --- кладёте вверх.

Формализуем: есть список $L$. Есть операции access$(x, L)$ за rank$(x)$; transpose$(x, y)$ --- за $O(1)$, но меняем только соседние элементы.

Решать будем для онлайн-алгоритма --- он не знает всех наших будущих запросов.

Пусть $|L| = n$, $S$ --- множество операций.

Понятно, что сложность в худшем случае не может быть меньше $\Theta(n|S|)$, то есть $c_A(S) = \Omega(n|S|)$

Средний случай: пусть $p(x)$ --- вероятность обращения к $x$.

$E(c_A(S)) = \sum\limits_{x\in L}\left(p(x)\rk(x)\right)$

Эвристика MTF --- после доступа помещаем элемент вверх списка. Оценим MTF:

Для начала введём понятие $\alpha$-конкуррентности; алгоритм $\alpha$-конкуррентен, если $C_A(S) \leqslant \alpha\cdot C_{OPT}(S)+k$.

Пусть $L_i^A$ --- список после $i$ операций по алгоритму $A$; $C_i^A$ --- стоимость $i$-ой операции по алгоритму $A$.

$\Phi(L_i^{MTF}) = 2|\{(x, y)\mid\rk_{L_i^{MTF}}(x) < \rk_{L_i^{MTF}}(y),\ \rk_{L_i^{OPT}}(x) > \rk_{L_i^{OPT}}(y) \}|$

При этом после одной транспозиции $\Phi$ меняется на $\pm 2$
\end{document}
