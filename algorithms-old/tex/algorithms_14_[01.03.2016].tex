\documentclass[a4paper,12pt]{article}

%% Работа с русским языком
\usepackage{cmap}					% поиск в PDF
\usepackage{mathtext} 				% русские буквы в формулах
\usepackage[T2A]{fontenc}			% кодировка
\usepackage[utf8]{inputenc}			% кодировка исходного текста
\usepackage[english,russian]{babel}	% локализация и переносы

%% Отступы между абзацами и в начале абзаца 
\setlength{\parindent}{0pt}
\setlength{\parskip}{\medskipamount}

%% Изменяем размер полей
\usepackage[top=0.5in, bottom=0.75in, left=0.625in, right=0.625in]{geometry}

%% Графика
\usepackage[pdftex]{graphicx}
\graphicspath{{images/}}

%% Различные пакеты для работы с математикой
\usepackage{mathtools}				% Тот же amsmath, только с некоторыми поправками

\usepackage{amssymb}				% Математические символы
\usepackage{amsthm}					% Пакет для написания теорем
\usepackage{amstext}
\usepackage{array}
\usepackage{amsfonts}
\usepackage{icomma}					% "Умная" запятая: $0,2$ --- число, $0, 2$ --- перечисление
\usepackage{bbm}				    % Для красивого (!) \mathbb с  буквами и цифрами
\usepackage{enumitem}               % Для выравнивания itemise (\begin{itemize}[align=left])

% Номера формул
\mathtoolsset{showonlyrefs=true} % Показывать номера только у тех формул, на которые есть \eqref{} в тексте.

% Ссылки
\usepackage[colorlinks=true, urlcolor=blue]{hyperref}

% Шрифты
\usepackage{euscript}	 % Шрифт Евклид
\usepackage{mathrsfs}	 % Красивый матшрифт

% Свои команды\textbf{}
\DeclareMathOperator{\sgn}{\mathop{sgn}}

% Перенос знаков в формулах (по Львовскому)
\newcommand*{\hm}[1]{#1\nobreak\discretionary{}
{\hbox{$\mathsurround=0pt #1$}}{}}

% Графики
\usepackage{tikz}
\usepackage{pgfplots}
%\pgfplotsset{compat=1.12}

% Изменим формат \section и \subsection:
\usepackage{titlesec}
\titleformat{\section}
{\vspace{1cm}\centering\LARGE\bfseries}	% Стиль заголовка
{}										% префикс
{0pt}									% Расстояние между префиксом и заголовком
{} 										% Как отображается префикс
\titleformat{\subsection}				% Аналогично для \subsection
{\Large\bfseries}
{}
{0pt}
{}

% Информация об авторах
\author{Группа лектория ФКН ПМИ 2015-2016 \\
	Анастасия Иовлева \\
	Ксюша Закирова \\
	Руслан Хайдуров}
\title{Лекции по предмету \\
	\textbf{Линейная алгебра и геометрия}}
\date{2016 год}

\newtheorem*{Def}{Определение}
\newtheorem*{Lemma}{Лемма}
\newtheorem*{Suggestion}{Предложение}
\newtheorem*{Examples}{Пример}
\newtheorem*{Comment}{Замечание}
\newtheorem*{Consequence}{Следствие}
\newtheorem*{Theorem}{Теорема}
\newtheorem*{Statement}{Утверждение}
\newtheorem*{Task}{Упражнение}
\newtheorem*{Designation}{Обозначение}
\newtheorem*{Generalization}{Обобщение}
\newtheorem*{Thedream}{Предел мечтаний}
\newtheorem*{Properties}{Свойства}

\renewcommand{\mathbb}{\mathbbm}
\renewcommand{\Re}{\mathrm{Re\:}}
\renewcommand{\Im}{\mathrm{Im\:}}
\newcommand{\Arg}{\mathrm{Arg\:}}
\renewcommand{\arg}{\mathrm{arg\:}}
\newcommand{\Mat}{\mathrm{Mat}}
\newcommand{\id}{\mathrm{id}}
\newcommand{\isom}{\xrightarrow{\sim}} 
\newcommand{\leftisom}{\xleftarrow{\sim}}
\newcommand{\Hom}{\mathrm{Hom}}
\newcommand{\Ker}{\mathrm{Ker}\:}
\newcommand{\rk}{\mathrm{rk}\:}
\newcommand{\diag}{\mathrm{diag}}
\newcommand{\ort}{\mathrm{ort}}
\newcommand{\pr}{\mathrm{pr}}
\newcommand{\vol}{\mathrm{vol\:}}

\renewcommand{\epsilon}{\varepsilon}
\renewcommand{\phi}{\varphi}
\newcommand{\e}{\mathbb{e}}
\renewcommand{\l}{\lambda}
\renewcommand{\C}{\mathbb{C}}
\newcommand{\R}{\mathbb{R}}
\newcommand{\E}{\mathbb{E}}

\newcommand{\vvector}[1]{\begin{pmatrix}{#1}_1 \\\vdots\\{#1}_n\end{pmatrix}}
\renewcommand{\vector}[1]{({#1}_1, \ldots, {#1}_n)}

\begin{document}

\section*{Лекция 14 от 01.03.2016}

\subsection{Поиск компонент связности в неориентированном графе}
(третий аргумент DFS --- то,  чем помечаются вершины)
\begin{lstlisting}
i := 1
for v in V do
    if v is not marked then
        DFS(G, v, i)
        i := i + 1
\end{lstlisting}

Сколько времени, если в графе $m$ вершин и $n$ рёбер? $O(m_n)$. Почему? Потому что в DFS мы проходим по каждому ребру в компоненте связности не более двух раз.

А для ориентированного графа?

Проверка сильной связности в ориентированном графе. Очевидно, это сложнее, потому что, если мы рассмотрим граф из двух вершин и одного ребра, то из одной DFS до другой доберётся, а в обратную сторону --- уже нет.

\begin{lstlisting}
DFS(G, s, 1)
if exists u in V: u is not marked then
    return false
return DFS(Grev, s, 2)
\end{lstlisting}

Если $G = (V, E)$, то $G^{rev} = (V, E^{rev})$

Вот как мы определяем $DFS$, кстати:

\begin{lstlisting}
DFS(G, s, i)
    mark s with i
    for (s, v) in E do
        if v is not marked
            DFS(G, v, i)
    t := t+1
    f[s] = t
\end{lstlisting}

Поиск компонент сильной связности:

\begin{lstlisting}
Explore(Grev)
unmark all
Explore(G)  // for v in V v poryadke ubyvuaniya f[v]
\end{lstlisting}

Пусть $C_1$ и $C_2$ --- две компоненты сильной связности; $i\in C_1,\ j\in C_2,\ (i,j)\in E$. Тогда $\max\limits_{u\in C_1} f[u] < \max\limits_{v\in C_2} f[v]$

Каждый вызов DFS внутри Explore(G) обходит некоторую компоненту сильной связности.

S --- множество вершин, обойдённых в Explore(G).

Базис: $S = \varnothing$.
Шаг: Пусть $S = C_1 \cup\ldots\cup C_k$, где $C_i$ --- компоненты сильной связности.

Пусть $C_v$ --- к.с.с., содержащая $v$ и $S_v$ --- множество вершин, обойдённых в DFS(G, v, i). Тогда $C_v \subseteq S_v$; $u\not\in C_v \implies \forall w \in C_u: f[v] < f[w]$.

\subsection{Матрицы смежности}
Пусть $A$ --- матрица смежности $G$; тогда $(A^t)_{ij}$ --- число путей из $i$ в $j$ длины $t$.

Существует ли путь из $i$ в $j$? Возведём $E+A$ в степень $n-1$ и проверим, что $ij$-ый элемент больше нуля. Используя быстрое возведение в степень можно обойтись $\lceil\log_2n\rceil$ перемножениями матриц.

\begin{lstlisting}
B := E + A
for x := 1 to log2 n do
    b_ij := \Lor (b_ik\land b_kj)
\end{lstlisting}

\subsection{Кратчайшие пути}
Вход: $G, W$;
Выход: $B$ такое, что $B_{ij}$ --- длина кратччайшего пути из $i$ в $j$.

Алгоритм Флойда-Уоршелла:

\begin{lstlisting}
B := W
for k in V do
    for i in V do
        for j in V do
        B_{ij}:= min(B_{ij}, B_{ik} + B_{kj})
\end{lstlisting}
\end{document}
