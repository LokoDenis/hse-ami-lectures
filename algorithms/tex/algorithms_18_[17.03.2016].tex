\documentclass[a4paper,12pt]{article}

%% Работа с русским языком
\usepackage{cmap}					% поиск в PDF
\usepackage{mathtext} 				% русские буквы в формулах
\usepackage[T2A]{fontenc}			% кодировка
\usepackage[utf8]{inputenc}			% кодировка исходного текста
\usepackage[english,russian]{babel}	% локализация и переносы

%% Отступы между абзацами и в начале абзаца 
\setlength{\parindent}{0pt}
\setlength{\parskip}{\medskipamount}

%% Изменяем размер полей
\usepackage[top=0.5in, bottom=0.75in, left=0.625in, right=0.625in]{geometry}

%% Графика
\usepackage[pdftex]{graphicx}
\graphicspath{{images/}}

%% Различные пакеты для работы с математикой
\usepackage{mathtools}				% Тот же amsmath, только с некоторыми поправками

\usepackage{amssymb}				% Математические символы
\usepackage{amsthm}					% Пакет для написания теорем
\usepackage{amstext}
\usepackage{array}
\usepackage{amsfonts}
\usepackage{icomma}					% "Умная" запятая: $0,2$ --- число, $0, 2$ --- перечисление
\usepackage{bbm}				    % Для красивого (!) \mathbb с  буквами и цифрами
\usepackage{enumitem}               % Для выравнивания itemise (\begin{itemize}[align=left])

% Номера формул
\mathtoolsset{showonlyrefs=true} % Показывать номера только у тех формул, на которые есть \eqref{} в тексте.

% Ссылки
\usepackage[colorlinks=true, urlcolor=blue]{hyperref}

% Шрифты
\usepackage{euscript}	 % Шрифт Евклид
\usepackage{mathrsfs}	 % Красивый матшрифт

% Свои команды\textbf{}
\DeclareMathOperator{\sgn}{\mathop{sgn}}

% Перенос знаков в формулах (по Львовскому)
\newcommand*{\hm}[1]{#1\nobreak\discretionary{}
{\hbox{$\mathsurround=0pt #1$}}{}}

% Графики
\usepackage{tikz}
\usepackage{pgfplots}
%\pgfplotsset{compat=1.12}

% Изменим формат \section и \subsection:
\usepackage{titlesec}
\titleformat{\section}
{\vspace{1cm}\centering\LARGE\bfseries}	% Стиль заголовка
{}										% префикс
{0pt}									% Расстояние между префиксом и заголовком
{} 										% Как отображается префикс
\titleformat{\subsection}				% Аналогично для \subsection
{\Large\bfseries}
{}
{0pt}
{}

% Информация об авторах
\author{Группа лектория ФКН ПМИ 2015-2016 \\
	Анастасия Иовлева \\
	Ксюша Закирова \\
	Руслан Хайдуров}
\title{Лекции по предмету \\
	\textbf{Линейная алгебра и геометрия}}
\date{2016 год}

\newtheorem*{Def}{Определение}
\newtheorem*{Lemma}{Лемма}
\newtheorem*{Suggestion}{Предложение}
\newtheorem*{Examples}{Пример}
\newtheorem*{Comment}{Замечание}
\newtheorem*{Consequence}{Следствие}
\newtheorem*{Theorem}{Теорема}
\newtheorem*{Statement}{Утверждение}
\newtheorem*{Task}{Упражнение}
\newtheorem*{Designation}{Обозначение}
\newtheorem*{Generalization}{Обобщение}
\newtheorem*{Thedream}{Предел мечтаний}
\newtheorem*{Properties}{Свойства}

\renewcommand{\mathbb}{\mathbbm}
\renewcommand{\Re}{\mathrm{Re\:}}
\renewcommand{\Im}{\mathrm{Im\:}}
\newcommand{\Arg}{\mathrm{Arg\:}}
\renewcommand{\arg}{\mathrm{arg\:}}
\newcommand{\Mat}{\mathrm{Mat}}
\newcommand{\id}{\mathrm{id}}
\newcommand{\isom}{\xrightarrow{\sim}} 
\newcommand{\leftisom}{\xleftarrow{\sim}}
\newcommand{\Hom}{\mathrm{Hom}}
\newcommand{\Ker}{\mathrm{Ker}\:}
\newcommand{\rk}{\mathrm{rk}\:}
\newcommand{\diag}{\mathrm{diag}}
\newcommand{\ort}{\mathrm{ort}}
\newcommand{\pr}{\mathrm{pr}}
\newcommand{\vol}{\mathrm{vol\:}}

\renewcommand{\epsilon}{\varepsilon}
\renewcommand{\phi}{\varphi}
\newcommand{\e}{\mathbb{e}}
\renewcommand{\l}{\lambda}
\renewcommand{\C}{\mathbb{C}}
\newcommand{\R}{\mathbb{R}}
\newcommand{\E}{\mathbb{E}}

\newcommand{\vvector}[1]{\begin{pmatrix}{#1}_1 \\\vdots\\{#1}_n\end{pmatrix}}
\renewcommand{\vector}[1]{({#1}_1, \ldots, {#1}_n)}

\begin{document}

\section*{Лекция ?? от 15.03.2016}

\subsection{Ассоциативный массив}

Продолжаем говорить про структуры данных. Ассоциативный массив (он же map в C++, он же словарь в Python). Это структура данных такая, что с каждой записью ассоциирован уникальный ключ и реализованы следующие операции:

Insert(S, x)

Delete(S, x)

Find(S, k)

Возможные способы:

Таблица с прямой адресацией:

Ключи берутся из $U = \{0,\ldots, m-1\}$; данные будем хранить в массиве размера m; все операции быстры, ожнако если мы хотим, чтобы ключи были, например, long'ами, то у нас такая таблица в память не поместится вообще никак. А что делать, если $U$ большое, бесконечное, например?

Хеш-функция:

Пусть есть функция $h: U\to \{0, \ldots, m-1\}$. При этом, записывать элемент будем в ячейку $h(k)$. Возможно, что двум ключам соответствует один хеш; тогда возникает \emph{коллизия}, и их можно решать разными способами. Например, можно хранить не элементы, а списки элементов с соответствующим хешем; необходимо только модифицировать наши функции.

\begin{lstlisting}
Find(T, k)
    list := T[h(k)]
    return k in list
\end{lstlisting}

Проблема этого подхода --- случай, когда хеши всех ключей равны. Тогда таблица вырождается в список, а он очень неэффективен. Худший случай для Find --- $O(n)$, в отличие от $O(1)$ в лучшем. А давайте рассмотрим средний случай?

Будем считать, что $h$ хорошая и распределяет $n$ ключей по $m$ хешам примерно равномерно. Пусть $\alpha = \frac{n}{m}$ --- коэффициент заполнения.

Ожидаемое время поиска отсутствующего элемента --- $O(1+\alpha)$. При этом, если $n = O(m)$, то поиск занимает $O()$

Но нам нужна хорошая функция. Где её взять?

Метод деления:

$h(x) = x \pmod{m}$.

Однако, если у нас, например, $m$ чётное, а так сложилось, что мы работаем с чётными числами, то половина ячеек нашей хеш-таблицы будет пустой. (Вообще, обычно $m$ берут простым и всё хорошо)

Метод умножения:

$h(k) = (ak\pmod{2^w}) >> (w-r)$, где $w$ --- длина слова.


А теперь снова к коллизиям. Можно использовать не списки, а таблицу с открытой адресацией, где используются по очереди несколько ($m$) хеш-функций, пока очередной хеш от ключа не окажется незанят. При этом мы хотим хоть какой-то эффективности, поэтому иметь больше чем $m$ функций --- излишне; значит, мы хотим, чтобы для всех $k,\ i\neq j$ выполнялось $h(k, i) \neq h(k, j)$.

Самый простой способ --- имея одну функцию $h'(k)$ определить $h(k, i) = (h'(k) + i) \pmod{m}$. Этот вариант не очень хорош --- будут образовываться большие подряд занятые блоки и они будут замедлять работу.

Лучше сделать так: $h(k, i) = (h'(k) + ih''(k)) \pmod{m}$.

А теперь поговорим об эффективности поиска:

Для каждого ключа все $m!$ перестановок равновероятны.

$\alpha = \frac{n}{m} < 1$

\[
    1+\frac{n}{m}\left(1+\frac{n-1}{m-1}\left(1+\frac{n-2}{m-2}\cdots\right)\right) \leqslant 1+\alpha(1+\alpha(1+\alpha\cdots)) \leqslant \frac{1}{1-\alpha}
\]<++>
\end{document}
