\documentclass[a4paper,12pt]{article}

%% Работа с русским языком
\usepackage{cmap}					% поиск в PDF
\usepackage{mathtext} 				% русские буквы в формулах
\usepackage[T2A]{fontenc}			% кодировка
\usepackage[utf8]{inputenc}			% кодировка исходного текста
\usepackage[english,russian]{babel}	% локализация и переносы

%% Отступы между абзацами и в начале абзаца 
\setlength{\parindent}{0pt}
\setlength{\parskip}{\medskipamount}

%% Изменяем размер полей
\usepackage[top=0.5in, bottom=0.75in, left=0.625in, right=0.625in]{geometry}

%% Графика
\usepackage[pdftex]{graphicx}
\graphicspath{{images/}}

%% Различные пакеты для работы с математикой
\usepackage{mathtools}				% Тот же amsmath, только с некоторыми поправками

\usepackage{amssymb}				% Математические символы
\usepackage{amsthm}					% Пакет для написания теорем
\usepackage{amstext}
\usepackage{array}
\usepackage{amsfonts}
\usepackage{icomma}					% "Умная" запятая: $0,2$ --- число, $0, 2$ --- перечисление
\usepackage{bbm}				    % Для красивого (!) \mathbb с  буквами и цифрами
\usepackage{enumitem}               % Для выравнивания itemise (\begin{itemize}[align=left])

% Номера формул
\mathtoolsset{showonlyrefs=true} % Показывать номера только у тех формул, на которые есть \eqref{} в тексте.

% Ссылки
\usepackage[colorlinks=true, urlcolor=blue]{hyperref}

% Шрифты
\usepackage{euscript}	 % Шрифт Евклид
\usepackage{mathrsfs}	 % Красивый матшрифт

% Свои команды\textbf{}
\DeclareMathOperator{\sgn}{\mathop{sgn}}

% Перенос знаков в формулах (по Львовскому)
\newcommand*{\hm}[1]{#1\nobreak\discretionary{}
{\hbox{$\mathsurround=0pt #1$}}{}}

% Графики
\usepackage{tikz}
\usepackage{pgfplots}
%\pgfplotsset{compat=1.12}

% Изменим формат \section и \subsection:
\usepackage{titlesec}
\titleformat{\section}
{\vspace{1cm}\centering\LARGE\bfseries}	% Стиль заголовка
{}										% префикс
{0pt}									% Расстояние между префиксом и заголовком
{} 										% Как отображается префикс
\titleformat{\subsection}				% Аналогично для \subsection
{\Large\bfseries}
{}
{0pt}
{}

% Информация об авторах
\author{Группа лектория ФКН ПМИ 2015-2016 \\
	Анастасия Иовлева \\
	Ксюша Закирова \\
	Руслан Хайдуров}
\title{Лекции по предмету \\
	\textbf{Линейная алгебра и геометрия}}
\date{2016 год}

\newtheorem*{Def}{Определение}
\newtheorem*{Lemma}{Лемма}
\newtheorem*{Suggestion}{Предложение}
\newtheorem*{Examples}{Пример}
\newtheorem*{Comment}{Замечание}
\newtheorem*{Consequence}{Следствие}
\newtheorem*{Theorem}{Теорема}
\newtheorem*{Statement}{Утверждение}
\newtheorem*{Task}{Упражнение}
\newtheorem*{Designation}{Обозначение}
\newtheorem*{Generalization}{Обобщение}
\newtheorem*{Thedream}{Предел мечтаний}
\newtheorem*{Properties}{Свойства}

\renewcommand{\mathbb}{\mathbbm}
\renewcommand{\Re}{\mathrm{Re\:}}
\renewcommand{\Im}{\mathrm{Im\:}}
\newcommand{\Arg}{\mathrm{Arg\:}}
\renewcommand{\arg}{\mathrm{arg\:}}
\newcommand{\Mat}{\mathrm{Mat}}
\newcommand{\id}{\mathrm{id}}
\newcommand{\isom}{\xrightarrow{\sim}} 
\newcommand{\leftisom}{\xleftarrow{\sim}}
\newcommand{\Hom}{\mathrm{Hom}}
\newcommand{\Ker}{\mathrm{Ker}\:}
\newcommand{\rk}{\mathrm{rk}\:}
\newcommand{\diag}{\mathrm{diag}}
\newcommand{\ort}{\mathrm{ort}}
\newcommand{\pr}{\mathrm{pr}}
\newcommand{\vol}{\mathrm{vol\:}}

\renewcommand{\epsilon}{\varepsilon}
\renewcommand{\phi}{\varphi}
\newcommand{\e}{\mathbb{e}}
\renewcommand{\l}{\lambda}
\renewcommand{\C}{\mathbb{C}}
\newcommand{\R}{\mathbb{R}}
\newcommand{\E}{\mathbb{E}}

\newcommand{\vvector}[1]{\begin{pmatrix}{#1}_1 \\\vdots\\{#1}_n\end{pmatrix}}
\renewcommand{\vector}[1]{({#1}_1, \ldots, {#1}_n)}

\begin{document}

\section*{Лекция ?? от 15.03.2016}

\subsection{?}

Пусть у нас есть некоторая функция Init$(n)$, создающая массив из $n$ нулей;
Read$(i)$;
Write$(a, i)$

Также будем исходить из предположения, что памядь под $n$ элементов можно выделить за время $O(1)$; чтение и запись тоже будем считать константными.

Соответственно Init$(n)$ работает за линейное время --- выделение за константу и $n$ присваиваний. А если мы хотим инициализацию тоже за константу?

Давайте сдеаем так --- не будем ничего хранить, пока не надо, а когда будет запрос в ранее незанятую ячейку, возвращать ноль.

Перепишем наш Read:

\begin{lstlisting}
Read(i)
    if IsInitialized(i) then
        return A[i]
    else return 0
\end{lstlisting}

\begin{lstlisting}
IsInitialized(i)
    if B[i] > k then 
        return False
    if C[B[i]] = i then
        return True
    else return False
\end{lstlisting}

\begin{lstlisting}
Write(a, i)
    A[i] := a
    if not IsInit(i)
        k := k+1
        B[i] := k
        C[k] := 1
\end{lstlisting}

\begin{lstlisting}
Init(m)
    A := Malloc(n)
    B := Malloc(n)
    C := Malloc(n)
    k := 0
\end{lstlisting}

\subsection{??}

Пусть у нас есть восьмибитное число 0; будем прибавлять к нему единицу. Заметим, что в первый раз понадобится одна операция, потом две, потом одна, потом три\ldots

Получается, что инкремент имеет сложность $O(k)$; логично предположить, что $n$ инкрементов --- $O(nk)$. Это действительно так, но это грубая оценка --- $O(k)$ ведь только в худшем случае.

\begin{lstlisting}
Increment(A)
    for i := 0 to k-1 do
        if A[i] = 0 then
            A[i] := 1
            break
        A[i] = 0
\end{lstlisting}

Заметим, что при $n$ инкрементах последний знак числа меняется $n$ раз, предпоследний --- $\frac{n}{2}$ и так далее.В сумме это составит не более $2n$ изменений ---$O(n)$.

\subsection{Банковский метод}.

$c_i$ --- реальная стоимость

$\hat c_i$ --- учётная стоимость.

$\sum \hat c_i \geqslant \sum c_i$ --- вчёт всегда больше 0.

Если $\hat c_i > c_i$ то разность поступает на счёт, иначе списывается.

При этом учётная стоимость является оценкой сверху для реальности.

В нашей задаче с инкрементом введём такие стоимости:

$\hat c (0\to 1) = 2;\ c (1\to 0) = 0$ 

Учётная стоимость одного инкремента не больше двух.

\subsection{Метод потенциалов}

$D_i$ --- структура данных после $i$ операций.

$\Phi(D_0) = 0$

$\Phi(D_i) \geqslant 0$

$\hat c_i := c_i + \Phi(D_i) - \Phi(D_{i-1})$

$\sum\limits_{i=1}^n\hat c_i = \sum\limits_{i=1}^n c_i +\Phi(D_n) - \Phi(D_0)$

Пусть $\Phi(D_i)$ --- количество единиц.

$\hat c_i = c_i + \Phi(D_i) - \Phi(D_{i-1}) \leqslant t+1+1-t = 2$

\end{document}
