\documentclass[a4paper,12pt]{article}

%% Работа с русским языком
\usepackage{cmap}					% поиск в PDF
\usepackage{mathtext} 				% русские буквы в формулах
\usepackage[T2A]{fontenc}			% кодировка
\usepackage[utf8]{inputenc}			% кодировка исходного текста
\usepackage[english,russian]{babel}	% локализация и переносы

%% Отступы между абзацами и в начале абзаца 
\setlength{\parindent}{0pt}
\setlength{\parskip}{\medskipamount}

%% Изменяем размер полей
\usepackage[top=0.5in, bottom=0.75in, left=0.625in, right=0.625in]{geometry}

%% Графика
\usepackage[pdftex]{graphicx}
\graphicspath{{images/}}

%% Различные пакеты для работы с математикой
\usepackage{mathtools}				% Тот же amsmath, только с некоторыми поправками

\usepackage{amssymb}				% Математические символы
\usepackage{amsthm}					% Пакет для написания теорем
\usepackage{amstext}
\usepackage{array}
\usepackage{amsfonts}
\usepackage{icomma}					% "Умная" запятая: $0,2$ --- число, $0, 2$ --- перечисление
\usepackage{bbm}				    % Для красивого (!) \mathbb с  буквами и цифрами
\usepackage{enumitem}               % Для выравнивания itemise (\begin{itemize}[align=left])

% Номера формул
\mathtoolsset{showonlyrefs=true} % Показывать номера только у тех формул, на которые есть \eqref{} в тексте.

% Ссылки
\usepackage[colorlinks=true, urlcolor=blue]{hyperref}

% Шрифты
\usepackage{euscript}	 % Шрифт Евклид
\usepackage{mathrsfs}	 % Красивый матшрифт

% Свои команды\textbf{}
\DeclareMathOperator{\sgn}{\mathop{sgn}}

% Перенос знаков в формулах (по Львовскому)
\newcommand*{\hm}[1]{#1\nobreak\discretionary{}
{\hbox{$\mathsurround=0pt #1$}}{}}

% Графики
\usepackage{tikz}
\usepackage{pgfplots}
%\pgfplotsset{compat=1.12}

% Изменим формат \section и \subsection:
\usepackage{titlesec}
\titleformat{\section}
{\vspace{1cm}\centering\LARGE\bfseries}	% Стиль заголовка
{}										% префикс
{0pt}									% Расстояние между префиксом и заголовком
{} 										% Как отображается префикс
\titleformat{\subsection}				% Аналогично для \subsection
{\Large\bfseries}
{}
{0pt}
{}

% Информация об авторах
\author{Группа лектория ФКН ПМИ 2015-2016 \\
	Анастасия Иовлева \\
	Ксюша Закирова \\
	Руслан Хайдуров}
\title{Лекции по предмету \\
	\textbf{Линейная алгебра и геометрия}}
\date{2016 год}

\newtheorem*{Def}{Определение}
\newtheorem*{Lemma}{Лемма}
\newtheorem*{Suggestion}{Предложение}
\newtheorem*{Examples}{Пример}
\newtheorem*{Comment}{Замечание}
\newtheorem*{Consequence}{Следствие}
\newtheorem*{Theorem}{Теорема}
\newtheorem*{Statement}{Утверждение}
\newtheorem*{Task}{Упражнение}
\newtheorem*{Designation}{Обозначение}
\newtheorem*{Generalization}{Обобщение}
\newtheorem*{Thedream}{Предел мечтаний}
\newtheorem*{Properties}{Свойства}

\renewcommand{\mathbb}{\mathbbm}
\renewcommand{\Re}{\mathrm{Re\:}}
\renewcommand{\Im}{\mathrm{Im\:}}
\newcommand{\Arg}{\mathrm{Arg\:}}
\renewcommand{\arg}{\mathrm{arg\:}}
\newcommand{\Mat}{\mathrm{Mat}}
\newcommand{\id}{\mathrm{id}}
\newcommand{\isom}{\xrightarrow{\sim}} 
\newcommand{\leftisom}{\xleftarrow{\sim}}
\newcommand{\Hom}{\mathrm{Hom}}
\newcommand{\Ker}{\mathrm{Ker}\:}
\newcommand{\rk}{\mathrm{rk}\:}
\newcommand{\diag}{\mathrm{diag}}
\newcommand{\ort}{\mathrm{ort}}
\newcommand{\pr}{\mathrm{pr}}
\newcommand{\vol}{\mathrm{vol\:}}

\renewcommand{\epsilon}{\varepsilon}
\renewcommand{\phi}{\varphi}
\newcommand{\e}{\mathbb{e}}
\renewcommand{\l}{\lambda}
\renewcommand{\C}{\mathbb{C}}
\newcommand{\R}{\mathbb{R}}
\newcommand{\E}{\mathbb{E}}

\newcommand{\vvector}[1]{\begin{pmatrix}{#1}_1 \\\vdots\\{#1}_n\end{pmatrix}}
\renewcommand{\vector}[1]{({#1}_1, \ldots, {#1}_n)}

\begin{document}

\section*{Лекция 5 от 26.01.2016}

\subsection{Умножение чисел}

Запишем умножение столбиком. На получение одной строки нужно $O(n)$ операций, на сложение --- тоже $O(n)$; итого --- $O(n^2)$. Можно ли быстрее? Колмогоров считал нельзя, оказалось, что можно.

Воспользуемся стратегией ``Разделяй и властвуй''. 

\begin{align*}
    x &= 10^{\frac{n}{2}}a + b\\
    y &= 10^{\frac{n}{2}}c + d\\
    \hline
    xy &= 10^nac + 10^{\frac{n}{2}}(ad+bc)+bd
\end{align*}

$T(n) = 4T\left( \frac{n}{2} \right) + \Theta(n)$

$T(n) \leqslant aT\left( \frac{n}{b} \right) + n^d$

$a = 4;\ b = 2;\ d = 1$

$a > b^d$

$T(n) = O\left( n^{\log_ba} \right) = O(a^2)$

А если сведём к трем подзадачам? Тогда получится вот так:

$T(n) = O\left( n^{\log_23} \right) \approx O(n^{1.58})$

Давайте перемножим:

$(a+b)(c+d) = ac+(ad+bc) + bd$

То есть $ad+bc$ из формулы --- это $(a+b)(c+d) - ac - bd$

$xy = 10^nac + 10^{\frac{n}{2}}z+bd$

И получили \emph{алгоритм Карацубы}.

\subsection{Перемножение матриц}

Пусть у нас есть квадратные матрицы $A = (a_{ij})$ и $B = (b_{ij})$. Пусть $C = A\times B$. Действуя строго по определению, умножение матриц займёт $O(n^3)$ --- для каждого жлемента матрицы нужно $n$ умножений и $n-1$ сложение, а элементов всего --- $n^2$.

И снова --- ``Разделяй и властвуй''. Попробуем делить матрицы на четыре подматрицы. Тогда пусть $A = \begin{pmatrix}
    A_{11}&A_{12}\\
    A_{21}&A_{22}
\end{pmatrix}$, 
$B = \begin{pmatrix}
    B_{11}&B_{12}\\
    B_{21}&B_{22}
\end{pmatrix}$.
Тогда $C = A\times B =
\begin{pmatrix}
    A_{11}B_{11}+A_{12}B_{21}&\ldots\\
    \vdots&\ddots
\end{pmatrix}$.

$T(n) = 8T\left( \frac{n}{2} \right) + O(n^2)$

Основная теорема: $8 > 2^2$

$O\left( n^{\log_28} \right) = O(n^2)$.

А если свести к семи подзадачам?

Алгоритм довольно простой, но как до него додуматься --- не вполне понятно.

\begin{align*}
    M_1 &= (A_{11} + A_{22})(B_{11}+B_{22}); \\
    M_2 &= (A_{21} + A_{22})B_{11}; \\
    M_3 &= A_{11}(B_{12}-B_{22}); \\
    M_4 &= A_{22}(B_{21}+B_{11}); \\
    M_5 &= (A_{11} + A_{12})B_{22}; \\
    M_6 &= (A_{21} - A_{11})(B_{11}+B_{12}); \\
    M_7 &= (A_{12} - A_{22})(B_{21}+B_{22}); \\
\end{align*}

\begin{align*}
    C_1 &= M_1+M_4-M_5+M_7; \\
    C_2 &= M_3+M_5; \\
    C_3 &= M_2+M_4; \\
    C_4 &= M_1-M_2+M_5+M_6; \\
\end{align*}

Можно проверить что всё верно (оставим это как упражнение читателю).

$T(n) = 7T\left( \frac{n}{2} \right) + O(n^2)$

$O\left( n^{\log_27} \right)$
\end{document}
