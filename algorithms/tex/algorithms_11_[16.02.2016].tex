\documentclass[a4paper,12pt]{article}

%% Работа с русским языком
\usepackage{cmap}					% поиск в PDF
\usepackage{mathtext} 				% русские буквы в формулах
\usepackage[T2A]{fontenc}			% кодировка
\usepackage[utf8]{inputenc}			% кодировка исходного текста
\usepackage[english,russian]{babel}	% локализация и переносы

%% Отступы между абзацами и в начале абзаца 
\setlength{\parindent}{0pt}
\setlength{\parskip}{\medskipamount}

%% Изменяем размер полей
\usepackage[top=0.5in, bottom=0.75in, left=0.625in, right=0.625in]{geometry}

%% Графика
\usepackage[pdftex]{graphicx}
\graphicspath{{images/}}

%% Различные пакеты для работы с математикой
\usepackage{mathtools}				% Тот же amsmath, только с некоторыми поправками

\usepackage{amssymb}				% Математические символы
\usepackage{amsthm}					% Пакет для написания теорем
\usepackage{amstext}
\usepackage{array}
\usepackage{amsfonts}
\usepackage{icomma}					% "Умная" запятая: $0,2$ --- число, $0, 2$ --- перечисление
\usepackage{bbm}				    % Для красивого (!) \mathbb с  буквами и цифрами
\usepackage{enumitem}               % Для выравнивания itemise (\begin{itemize}[align=left])

% Номера формул
\mathtoolsset{showonlyrefs=true} % Показывать номера только у тех формул, на которые есть \eqref{} в тексте.

% Ссылки
\usepackage[colorlinks=true, urlcolor=blue]{hyperref}

% Шрифты
\usepackage{euscript}	 % Шрифт Евклид
\usepackage{mathrsfs}	 % Красивый матшрифт

% Свои команды\textbf{}
\DeclareMathOperator{\sgn}{\mathop{sgn}}

% Перенос знаков в формулах (по Львовскому)
\newcommand*{\hm}[1]{#1\nobreak\discretionary{}
{\hbox{$\mathsurround=0pt #1$}}{}}

% Графики
\usepackage{tikz}
\usepackage{pgfplots}
%\pgfplotsset{compat=1.12}

% Изменим формат \section и \subsection:
\usepackage{titlesec}
\titleformat{\section}
{\vspace{1cm}\centering\LARGE\bfseries}	% Стиль заголовка
{}										% префикс
{0pt}									% Расстояние между префиксом и заголовком
{} 										% Как отображается префикс
\titleformat{\subsection}				% Аналогично для \subsection
{\Large\bfseries}
{}
{0pt}
{}

% Информация об авторах
\author{Группа лектория ФКН ПМИ 2015-2016 \\
	Анастасия Иовлева \\
	Ксюша Закирова \\
	Руслан Хайдуров}
\title{Лекции по предмету \\
	\textbf{Линейная алгебра и геометрия}}
\date{2016 год}

\newtheorem*{Def}{Определение}
\newtheorem*{Lemma}{Лемма}
\newtheorem*{Suggestion}{Предложение}
\newtheorem*{Examples}{Пример}
\newtheorem*{Comment}{Замечание}
\newtheorem*{Consequence}{Следствие}
\newtheorem*{Theorem}{Теорема}
\newtheorem*{Statement}{Утверждение}
\newtheorem*{Task}{Упражнение}
\newtheorem*{Designation}{Обозначение}
\newtheorem*{Generalization}{Обобщение}
\newtheorem*{Thedream}{Предел мечтаний}
\newtheorem*{Properties}{Свойства}

\renewcommand{\mathbb}{\mathbbm}
\renewcommand{\Re}{\mathrm{Re\:}}
\renewcommand{\Im}{\mathrm{Im\:}}
\newcommand{\Arg}{\mathrm{Arg\:}}
\renewcommand{\arg}{\mathrm{arg\:}}
\newcommand{\Mat}{\mathrm{Mat}}
\newcommand{\id}{\mathrm{id}}
\newcommand{\isom}{\xrightarrow{\sim}} 
\newcommand{\leftisom}{\xleftarrow{\sim}}
\newcommand{\Hom}{\mathrm{Hom}}
\newcommand{\Ker}{\mathrm{Ker}\:}
\newcommand{\rk}{\mathrm{rk}\:}
\newcommand{\diag}{\mathrm{diag}}
\newcommand{\ort}{\mathrm{ort}}
\newcommand{\pr}{\mathrm{pr}}
\newcommand{\vol}{\mathrm{vol\:}}

\renewcommand{\epsilon}{\varepsilon}
\renewcommand{\phi}{\varphi}
\newcommand{\e}{\mathbb{e}}
\renewcommand{\l}{\lambda}
\renewcommand{\C}{\mathbb{C}}
\newcommand{\R}{\mathbb{R}}
\newcommand{\E}{\mathbb{E}}

\newcommand{\vvector}[1]{\begin{pmatrix}{#1}_1 \\\vdots\\{#1}_n\end{pmatrix}}
\renewcommand{\vector}[1]{({#1}_1, \ldots, {#1}_n)}

\begin{document}

\section*{Лекция 11 от 16.02.2016}

\subsection{Алгоритмы на графах}
Графы бывают ориентированными и неориентированными, при этом неориентированные --- частный случай ориентированных. Сегодня мы будем говорить исключительно о неориентированных.

\subsection{Достижимость}
Вход: $G(V, E); s, t \in V$.
Вопрос: есть ли путь из $s$ в $t$ в $G$?

Небольшое отступление: ещё бывают взвешенные графы, где каждому ребру сопоставлен его вес (например, длина).

А можно поставить задачу поиска \emph{кратчайшего пути}:
Вход: $G(V, E); s, t \in V, W_i$.
Выход: длина кратчайшего пути из $s$ в $t$.

Но о взвешенных графах мы тоде говорит сегодня не будем.

Вообще говоря, многие задачи на графах не сразу бросаются в глаза как задачи, собственно, на графах --- например, та же задача о расстоянии редактирования; если рассмотреть слова как вершины графа, а рёбра провести между теми вершинами, которые можно перевести одну в другую одой операцией; а хотим мы найти кратчайшее расстояние межды dthibyjq $s$ и $t$. Понятно, что такое решение не очень эффективно просто потому, что граф получается бесконечный, но тем не менее, как решение вполне годится.

Итак, задача достижимости. Что можно сделать в самом начале? Посмотреть на соседей $s$. Если среди них есть $t$, то мы выиграли; если нет, то посмотрим на соседей этих соседей и так далее. Запишем алгоритм, который обойдёт всевершины, достижимые из $s$ и пометит их: 

Пусть $Q$ --- множество тех, кого мы уже нашли, но чьих соседей ещё не проверили;
\begin{lstlisting}
Explore(G, s)
    Q := {s}
    while Q != {} do
        extract u from Q
        mark u
        for v such that (u, v) in E do
            if v is not marked then
             add v to Q
\end{lstlisting}

Однако, алгоритм ещё не идеален; мы не уточнили, какую именно вершину мы извлекаем из $Q$; проверим, что будет, если $Q$ работает как стек.

(тут неплохо бы как-то это описать)

Заметим, что такой алгоритм будет ``углубляться'' в граф на каждой итерации, двигаясь по неокоторому ациклическому пути пока ему есть кда идти; как только идти стало некуда, он начнёт ``отступать'', пока не окажется в вершине, из которой можно попасть в некоторую ещё не исследованную. То есть, попав в вершину $f$, мы не вернёмся назад, пока не обойдём всех соседей $f$. Такой алгоритм называется \emph{поиск в глубину} (\emph{depth-first search}).

Можно записать его рекурсивно:

\begin{lstlisting}
DFS(G, S)
    mark s
    for v such that (s, v) in E do
        if v is not marked
            DSF(G, v)
\end{lstlisting}

А теперь попробуем сделать $Q$ очередью. Пройдясь по графу вручную, становится понятно, что для улучшения алгоритма стоит помечать вершины перед их добавлением в очередь. Порядок обхода это не меняет, впрочем; наш алгоритм обходит алгоритм как бы ``по слоям'':

Пусть $L_0 = \left\{ s \right\}$. определим слои рекуррентно: $L_{j+1} = \left\{ v\mid \forall i \leqslant j: v\not\in L_i, \exists u \in L_j:(u, v) \in S \right\}$. Заметим, что номер слоя, в который входит вершина, есть длина кратчайшего пути из $s$ в неё, то есть \emph{поиск в ширину} (\emph{breadth-first search}) можно использовать для поиска кратчайшего пути. Перепишем его для этого:

Пусть $d$ --- список расстояний.
\begin{lstlisting}
BFS(G, s)
    for v in V do
        d[v] := infinity
    d[s] := 0
    Q := [s]
    while Q is not empty do
         u := dequeue(Q)
         for v such that (u, v) in E do
         if d[v] = infinity then
            enqueue(v)
            d[v] = d[u]+1
\end{lstlisting}

При этом, мы получим расстояния, но не получим пути; для этого, впрочем, достаточно завести массив $P$ и сохранять в него родителя текущей вершины.
\end{document}
