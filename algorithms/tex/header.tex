% Размер страницы и шрифта
\documentclass[12pt,a4paper]{article}

% Подключение русского языка
\usepackage[T2A]{fontenc}
\usepackage[utf8]{inputenc}
\usepackage[english,russian]{babel}

% Изменяем размер полей
\usepackage[top=0.5in, bottom=0.75in, left=0.625in, right=0.625in]{geometry}

% Различные пакеты
\usepackage{mathtools}				% Тот же amsmath, только с некоторыми поправками
\usepackage{amssymb}				% Математические символы
\usepackage{amsthm}					% Пакет для написания теорем
\usepackage{graphicx}				% Пакет для вставки картинок
\usepackage{listings}				% Пакет для написания кода на каком-то языке программирования
\usepackage{algorithm}				% Пакет для написания алгоритмов
\usepackage[noend]{algpseudocode}	% Подключает псевдокод, отключает end if и иже с ними
\usepackage{indentfirst}			% Начало текста с красной строки
\usepackage[colorlinks=true, urlcolor=blue]{hyperref}	% Ссылки
\usepackage{pgfplots}				% Графики
\pgfplotsset{compat=1.12}
\usepackage{forest}					% Деревья
\usepackage{titlesec}				% Изменение формата заголовков

% Изменим формат \section и \subsection:
\titleformat{\section}
	{\vspace{1cm}\centering\LARGE\bfseries}	% Стиль заголовка
	{}										% префикс
	{0pt}									% Расстояние между префиксом и заголовком
	{} 										% Как отображается префикс
\titleformat{\subsection}					% Аналогично для \subsection
	{\Large\bfseries}
	{}
	{0pt}
	{}

% Теоремы и утверждения. В комменте указываем номер лекции, в которой это используется.
\newtheorem*{minimal_steps}{Утверждение}	% Лекция 1

% Информация об авторах
\author{Группа лектория ФКН ПМИ 2015-2016 \\
	Никита Попов \\
	Тамерлан Таболов \\
	Лёша Хачиянц}
\title{Лекции по предмету \\
	\textbf{Алгоритмы и структуры данных}}
\date{2016 год}