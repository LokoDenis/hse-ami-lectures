\documentclass[11pt,a4paper]{article}
\usepackage{listings}
\usepackage{amssymb}
\usepackage{amsmath}
\usepackage[utf8]{inputenc}
\usepackage[russian]{babel}
\usepackage[margin=0.5in]{geometry}
\usepackage{graphicx}
\usepackage{listings}
\allowdisplaybreaks

\title{Лекция по АиСД №6}

\date{\today}

\author{Попов Никита}


\begin{document}
\maketitle

Вход --- множество точек $(x_i, y_i)$. Выход --- $i, j: d((x_i, y_i), (x_j, y_j))$ --- минимально.

``Разделяй и властвуй'':

\begin{lstlisting}
closest_pair_rec(P_x, P_y)
    if n < 4 then
        solve directly
    L_x := P_x[1..ceil(n/2)]. L_y = ...
    R_x := P_x[ceil(n/2)+1..n], R_y = ...
    (l_1, l_2) := closest_pair_rec(L_x, L_y)
    (r_1, r_2) := closest_pair_rec(R_x, R_y)
    delta := min(d(l_1, l_2), d(r_1, r_2))
    
    S_x^l := {(x, y) \in L_x | x^*-x < delta}
    S_x^r := {(x, y) \in L_x | x-x^* < delta}

    S_y := {(x, y) \in P_y | |x-x^*| \leqslant delta}

    for i := 1 to |S_y| do
        c_i := argmin_{i-15\leqslant j\leqslant i+15}(d(S_y[i], S_y[j]))
        if d(S_y[i], s_y[c]) < delta then

        WAAAAAAAAT
\end{lstlisting}

\begin{lstlisting}
closest_pair(P)
    P_x := sort P by x
    P_y := sort P by y
    closest_pair_rec(P_x, P_y)
\end{lstlisting}

Теперь нужно объединить. Сколько времени у нас есть?

Для $n\log n$ $T(n) = 2T(\frac{n}{2}) + O(n)$, то есть у нас остаётся $O(n)$

Теперь нам нужно проверить, нет ли пары ближе, не по рзные стороны от границы. Нам интересна ближайшая такая пара, но только если расстояние меньше delta.

Пусть $x^* = max\left\{ x\mid(x, y)\in L_x \right\}$

\begin{lstlisting}
\end{lstlisting}

Тут картинка

Рассмотрим точки, попавшие в полосу. Для всех точек в $L_x$ нам не обязательно рассматривать все точки в полосе в $R_y$. Рассмотрим только те, что по $y$ лежат в delta-окрестности


Возьмём полосу и поделим на квадраты со стороной $\delta/2$. Сколько точек в каждом таком квадрате? Если там есть две точки, то расстояние между ними меньше $\delta$, а это невозможно.

Понятно, что между точками на расстоянии $<\delta$ не более трёх рядов квадратов. Значит, всегда будет достаточно рассмотреть 15 ближайших точек.
\end{document}
