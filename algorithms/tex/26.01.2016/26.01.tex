\documentclass[11pt,a4paper]{article}
\usepackage{listings}
\usepackage{amssymb}
\usepackage{amsmath}
\usepackage[utf8]{inputenc}
\usepackage[russian]{babel}
\usepackage[margin=0.5in]{geometry}
\usepackage{graphicx}
\usepackage{listings}
\allowdisplaybreaks

\title{Лекция по АиСД №2}

\date{\today}

\author{Попов Никита}


\begin{document}
\maketitle

Мы рассматривали случаи, когда все элементы различны, однако, если есть равные, алгоритм может сломаться.

$\begin{array}{|c|c|c|c|}
    \hline
    <x&>x&?&=x\\
    \hline
\end{array}$

Индексы границ --- $i$, $j$, $k$.
\begin{lstlisting}
Partition3(a)
    i:= 1
    j:= 1
    k:= n
    while j < k do
    if a[j] = a[n] then
        k -= 1
        a[j], a[k] := a[k], a[j]
\end{lstlisting}
\hspace{1.8cm}Однако, мы не сдвинули $j$ --- на $j$-ом месте теперь нерассмотренный ранее элемент
\begin{lstlisting}
    else if a[j] < a[n] then
        a[i], a[j] := a[j], a[i]
        i += 1
        j += 1
    else
        j += 1
\end{lstlisting}

После выхода из while получим

$\begin{array}{|c|c|c|c|}
    \hline
    <x&>x&=x\\
    \hline
\end{array}$

При этом $j = k$.

Остаётся только переставить части массива.

\begin{lstlisting}
    while i < k and j <= n do
        a[i], a[j] := a[j], a[i]
        i += 1
        j += 1
\end{lstlisting}

Самая быстрая из наших сортировок --- $n \log n$. А можно ли быстрее?

На основе только сравнений --- нет. Но если мы знаем, что числа только целые и не больше некоторой константы $C$, то можно сделать так:

Тут вставлю описание потом. (сортировка подсчётом)

Рассмотрим общий алгоритм --- через сравнения.

$n = 3$.

*картинка из тетради*

$n$ элементов; не меньше $n!$ листьев.

Дерево бинарное, так что $2^h \geqslant n!$

$h \geqslant \log_2 n! \geqslant log_2\left( \frac{n}{2} \right)^{\frac{n}{2}} = \frac{n}{2}\log_2\frac{n}{2} = \Omega(n\log n)$

\section*{Поиск медианы}

Медиана --- такой элемент массива, что не меньше половины элементов меньше неё, и не меньше половины --- больше.

Очевидно, что можно отсортировать и взять средний --- $\Theta(n)$.

Можно ли за линейное время? Можно.


Поиск $k$-ой порядковой статистики:
\begin{lstlisting}
Select(a, k)
    choose pivot a[p] = x
    i := Partition(a, p)
    if i = k then
        return a[i]
    if i > k then
        return Select(a[:i-1, k])
    return Select(a[i+1:, k-i])
\end{lstlisting}

$\begin{array}{|c|c|c|}
    \hline
    <=x&x&>=x\\
    \hline
    &i&
\end{array}$

Как и в быстрой сортировке, неправильно выбранный опорный элемент портит скорость до $n^2$.

$j$-подзадача размер $n'$.

$\left( \frac{3}{4} \right)^{j+1}n < n' \leqslant \left( \frac{3}{4} \right)^{j}n$

Среднее время с одним $j$ --- 2;

Максимальное $j$ --- $O(\log n)$

$T(n) \leqslant \sum\limits_{j=0}^{\log_{\frac{4}{3}}n} 2\cdot c\cdot \left( \frac{3}{4} \right)^jn = 2cn\sum\left( \frac{3}{4} \right)^j \leqslant 2cn$

\section*{Медиана медиан}
 
Divide a into groups of 5

Choose medians $m_{1}, \ldots, m_{\frac{n}{5}}$

m:= Select ($[m_i], \frac{n}{10}$)

choose x as pivot



$T(n) \leqslant cn + T\left(\frac{n}{5}\right) + T\left( \frac{7}{10}n \right)$

$T(n) \leqslant ln$ для некоторого $l$

$T(n) \leqslant cn + \frac{ln}{5} + \frac{7}{10}ln$
\end{document}
